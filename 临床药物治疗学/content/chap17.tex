\chapter{恶性肿瘤的药物治疗}

\section{概述}

\subsection{概述}

肿瘤是指机体在各种致瘤因素作用下,局部组织的细胞异常增生而形成的局部肿块。恶性肿瘤又称癌症,近年来的发病率呈上升的趋势,是一种严重危害人类生命与健康的疾病。

早期,肿瘤细胞处于活跃的增殖分裂阶段,这时进入增殖周期的细胞数量较多,生长率高,对化疗药物敏感,随着瘤体的增大,生长率降低,对化疗药物的敏感度下降。

肿瘤诊断时的分期和治疗后的再分期对治疗和预后均有重要影响,分期越低,治疗效果越好,预后也越好。临床上多用4期分类法:Ⅰ期为局限性肿瘤,Ⅱ期和Ⅲ期分别为局部和区域性扩散性肿瘤,Ⅳ期表示远距离转移的肿瘤。

目前治疗恶性肿瘤的主要方法有手术治疗、放射治疗(放疗)和化学治疗三种。药物化疗的目的在于杀灭癌细胞,在手术前后应用可缩小瘤体,抑制肿瘤细胞的扩散和杀灭残存的肿瘤细胞。对于不能手术切除者,化疗作为临床治疗的主要手段,可起到控制肿瘤生长、缓解症状,以及延长患者生命和改善患者生活质量的作用。

\subsection{常用化疗药物}

\subsubsection{烷化剂}

烷化剂属于细胞毒类药物,在体内能形成碳正离子或其他具有活泼的亲电性基团的化合物,进而与细胞中的生物大分子(DNA、RNA、酶)中含有丰富电子的基团(如氨基、巯基、羟基、羧基、磷酸基等)发生共价结合,使其丧失活性或使DNA分子发生断裂,导致肿瘤细胞死亡。常用烷化剂有环磷酰胺(Cyclophosphamide,CTX)、异环磷酰胺(Ifosfamide,IFO)、噻替哌(Thiotepa)、卡莫司汀(Carmustine,BCNU)、洛莫司汀(Lomustine)等。

\subsubsection{抗代谢药}

抗代谢药是指能与体内代谢物发生特异性结合,通过抑制DNA合成叶酸、嘌呤、嘧啶及嘧啶核苷的合成途径,从而抑制肿瘤细胞的生成和复制所必需的代谢途径,导致肿瘤细胞死亡的抗肿瘤药物。常用抗代谢药物有氟尿嘧啶(5-Fluorouracil,5-FU)、卡培他滨(Capecitabine)、阿糖胞苷(Cytarabine,Ara-C)、吉西他滨(Gemcitabine)、MTX等。

\subsubsection{抗生素类}

抗生素类抗肿瘤药主要通过抑制细胞DNA或RNA的合成,从而导致肿瘤细胞死亡。常用抗生素类有丝裂霉素C(Mitomycin
C,MMC)、多柔比星(Doxorubicin,ADM)、表柔比星(Epirubicin,EPI)、柔红霉素(Daunorubicin,DNR)、米托蒽醌(Mitoxantrone,MIT)等。

\subsubsection{植物碱类}

植物碱类是指从天然植物中发现的具有抗肿瘤活性的药物及其衍生物。常用的有长春碱(Vinblastine,VLB)、长春新碱(Vincristine,VCR)、长春地辛(Vindesine,VDS)、长春瑞滨(Navelbine,NVB)、伊立替康(Irinotecan,CPT-11)、托泊替康(Topotecan,TPT)、喜树碱(Camptothecine,CPT)、羟基喜树碱(Hydroxycamptothecin,HCPT)、紫杉醇(Paclitaxel,TAX)、多西他赛(Docetaxel,TXT)、高三尖杉酯碱(Homobarringtoni,HHRT)、依托泊苷(Etoposide,VP-16)、替尼泊苷(Teniposide,VM-26)等。

\subsubsection{铂类配合物}

铂类配合物为金属铂的配合物,可作用于DNA链间和链内交链,干扰DNA的复制,或可与核蛋白和胞质蛋白结合从而导致肿瘤细胞的死亡。常用铂类配合物有顺铂(Cisplatin,DDP)、卡铂(Carboplatin,CBP)、奥沙利铂(Oxaliplatin,L-OHP)等。

\section{肺癌}

\subsection{概述}

肺癌是肺部的恶性肿瘤。近50年来,许多国家都报道肺癌的发病率和病死率均明显增高,男性肺癌发病率和病死率均占所有恶性肿瘤的第一位;女性发病率占第二位,病死率占第二位,严重威胁人们的健康和生命。

肺癌的病因与发病机制至今尚未明确,流行病学研究表明肺癌的发病与吸烟、职业和环境因素、电离辐射、既往肺部感染、遗传等因素有关。

肺癌的临床表现比较复杂,症状和体征的有无、轻重以及出现的早晚,取决于肿瘤发生部位、病理类型、有无转移和并发症,以及患者的反应程度和耐受性的差异。肺癌早期症状常较轻微,甚至可无任何不适。中央型肺癌症状出现早且重,周围型肺癌症状出现晚且较轻,甚至无症状,常在体检时被发现。肺癌的症状大致分为局部症状、全身症状、肺外症状等。局部症状有咳嗽、痰血或咯血、胸痛、胸闷气短、声音嘶哑等;全身症状有发热、消瘦和恶病质等;肺外症状以肺源性骨关节增生症较多见,其次还有内分泌紊乱和神经肌肉病变等。

\subsection{肺癌的治疗}

肺癌的治疗方法主要有手术治疗、放射治疗和药物治疗三种。肺癌的治疗要根据患者的身体状况、肿瘤的病理类型、侵犯的范围和病程早晚合理运用治疗手段,制定治疗方案,提高患者的治愈率和生活质量。

根据肺癌的生物学特点,临床多将肺癌分为非小细胞肺癌和小细胞肺癌两大类,其中非小细胞肺癌占80%以上,小细胞肺癌占10%~15%,两者的治疗原则和方案不同。

\subsection{肺癌的药物治疗}

由于肺癌患者在诊断时有2/3已经超过了手术切除的范围,1/2已经有了临床或潜在的播散,因此化疗在肺癌的治疗上占有较重要的地位。化疗对小细胞肺癌的疗效无论早期或晚期均较肯定,甚至有约1%的早期小细胞肺癌通过化疗治愈。化疗也是治疗非小细胞肺癌的主要手段,化疗治疗非小细胞肺癌的肿瘤缓解率为40%~50%。

\subsubsection{常用药物}

常用的治疗非小细胞肺癌的药物有长春瑞滨(NVB)、紫杉醇(TAX)、多西他赛(TXT)、吉西他滨(Gemcitabine)、伊立替康(CPT-11)、托泊替康(TPT)、奥沙利铂(L-OPH)等;治疗小细胞肺癌的药物有异环磷酰胺(IFO)、卡铂(CBP)、替尼泊苷(VM-26)、长春地辛(VDS)、顺铂(DDP)、表柔比星(EPI)等。

\subsubsection{小细胞肺癌的化疗}

小细胞肺癌的治疗主要是全身化疗,目前多采用先化疗再手术,术后再辅助化疗的手段。局限期小细胞肺癌和广泛期小细胞肺癌的常用化疗方案如表\ref{tab17-1}和表\ref{tab17-2}所示。

\begin{longtable}[]{lp{7cm}}
    \caption{局限期小细胞肺癌的常用化疗方案}
    \label{tab17-1}\\
\toprule
方案 & 药物、剂量、给药方法和疗程\tabularnewline
\midrule
\endhead
EP & DDP 20mg/m{2} ,静脉滴注,d1~5或者50mg/m{2} ,d1~3;VP-16
100mg/d,静脉滴注,d1~5,第3周重复1次,至少3个周期\tabularnewline
CE & CBP 300mg/m{2} ,静脉滴注,d1;VP-16
100mg/d,静脉滴注,d1~5,每3或4周重复1次,至少连用3次\tabularnewline
COMVP(OME) & CTX 500mg/m{2} ,静脉滴注,d1,d8;VCR
1~2mg,静脉滴注,d1,d8;MTX 10~20mg,静脉滴注,d1,d8;VP-16
100mg/d,静脉滴注,d1~5,第3周重复1次,至少连用3个周期\tabularnewline
\bottomrule
\end{longtable}

\begin{longtable}[]{lp{7cm}}
    \caption{广泛期小细胞肺癌的常用化疗方案}
    \label{tab17-2}\\
\toprule
方案 & 药物、剂量、给药方法和疗程\tabularnewline
\midrule
\endhead
VIP & VP-16 75mg/m{2} ,静脉滴注,d1~5;IFO 1.29g/m{2}
,静脉滴注,d1~5;DDP 20mg/m{2}
,静脉滴注,d1~5,第3周重复1次,至少连用3个周期\tabularnewline
IME & IFO 1.29g/m{2} ,静脉滴注,d1~4;MTX
20mg,静脉滴注,d1,d8;VP-16
100mg/d,静脉滴注,d1~5。第3周重复1次,至少连用3个周期\tabularnewline
CTE & CBP 300mg/m{2} ,静脉滴注,d1;TAX 135mg/m{2}
,静脉滴注,d1;VP-16 100mg/m{2} ,静脉滴注,d1~5\tabularnewline
\bottomrule
\end{longtable}

\subsubsection{非小细胞肺癌的化疗}

Ⅰ-Ⅱ期非小细胞肺癌的治疗以手术为主,化疗一般作为术后辅助治疗应用,ⅢA期可采用术前化疗,Ⅳ期则以化疗为主,适当进行放疗。常用方案见表\ref{tab17-3}所示。
 
\begin{longtable}[]{p{5cm}p{5cm}p{5cm}}
    \caption{非小细胞肺癌的常用化疗方案}
    \label{tab17-3}\\
\toprule
适用范围&方案&药物、剂量、给药方法和疗程\tabularnewline
\midrule
\endhead
术前或术后辅助&CAP&CTX 500mg/m{2} ,静脉滴注,d1;ADM 40mg/m{2} ,静脉滴注,d1;DDP
40mg/m{2} ,静脉滴注,d1~3,每4周重复1次\\
术前或术后辅助&NP&NVB 35mg/m{2} ,静脉滴注(8~10min),d1;DDP 40mg/m{2}
,静脉滴注,d1~3,每3周重复1次\\
晚期非小细胞肺癌治疗&TP&TAX  135mg/m{2} ,静脉滴注,d1;DDP 60mg/m{2}
,静脉滴注,d3,每3周重复1次\\
晚期非小细胞肺癌治疗&CT&TAX 135mg/m{2} ,静脉滴注,d1;CBP 300mg/m{2}
,静脉滴注,d2,每3周重复1次\\
晚期非小细胞肺癌治疗&TXT+P&TAX 60~75mg/m{2} ,静脉滴注,d1;DDP 60mg/m{2}
,静脉滴注,d2,每3周重复1次\\
\bottomrule
\end{longtable}

\section{胃癌}

\subsection{概述}

胃癌是最常见的消化道恶性肿瘤,是发病率最高的恶性肿瘤之一。其中,男性的发病率是女性的1.5~2.5倍,发病高峰年龄为50~80岁,30岁以下年轻人发病率近年来有所升高。

胃癌的病因可能与{Hp}
感染、摄取含硝酸盐和亚硝酸盐食物、环境因素、家族遗传等因素有关。

胃癌早期无明显症状,随着病情的发展,可逐渐出现非特异性的酷似胃炎或胃溃疡的症状,进展期胃癌常见上腹痛,同时伴有食欲缺乏、食无味和体重减轻,当肿瘤侵袭及较大血管或黏膜下层血管受到广泛浸润破坏时,可发生大量呕血或黑便。

\subsection{胃癌的治疗}

手术治疗是唯一有可能根治胃癌的治疗方法,胃癌的诊断一经建立,应尽早施行根治性手术,因各种原因不能进行手术者,亦要争取原发灶的姑息性切除,并配合术后化疗。进展期胃癌即使进行根治手术,仍有较高的复发和转移率,术前术后均应进行辅助放疗、化疗和免疫疗法。

\subsection{胃癌的药物治疗}

常用于治疗胃癌的药物有5-FU、顺铂(DDP)、多柔比星(ADM)、依托泊苷(VP-16)等,常用的联合化疗方案如表\ref{tab17-4}所示。

\begin{longtable}[]{lp{8cm}}
    \caption{胃癌的常用化疗方案}
    \label{tab17-4}\\
\toprule
方案 & 药物、剂量、给药方法和疗程\tabularnewline
\midrule
\endhead
FAM & 5-FU 600mg/m$^2$ ,静脉滴注,d1、d8、d39、d36;ADM 30mg/m$^2$
,静脉滴注,d1、d29;MMC 10mg/m$^2$
,静脉滴注,d1,每6周为1疗程\tabularnewline
EAP &  ADM 20mg/m$^2$ ,静脉滴注,d1、d7;VP-16
120mg/m$^2$ (60岁以上,100mg/m$^2$
),静脉滴注,d4~d6;DDP 40mg/m$^2$
,静脉滴注,d2、d8,水化;每4周重复,3周期为1疗程\tabularnewline
ELF &  CF(亚叶酸钙)200mg/m$^2$
,静脉滴注(10min),d1~3;5-FU 500mg/m$^2$
,静脉滴注(10min),d1~3; VP-16 120mg/m$^2$
,静脉滴注(50min),d1~d3,每4周重复\tabularnewline
\bottomrule
\end{longtable}

\section{肝癌}

\subsection{概述}

肝癌即肝脏恶性肿瘤,是外科疾病中的常见病和多发病,是我国高发的恶性肿瘤。肝癌可分为原发性和继发性两大类。原发性肝癌起源于肝脏的上皮或间叶组织;继发性或称转移性肝癌系指全身多个器官起源的恶性肿瘤侵犯至肝脏,一般多见于胃、胆道、胰腺、结直肠、卵巢、子宫、肺、乳腺等器官恶性肿瘤的肝转移。自1990年代以来,我国肝癌的发病率已升至恶性肿瘤发病率的第二位,我国每年肝癌死亡病例占全球肝癌病死率的42%,肝癌的高发年龄为35~45岁,其中男性发病率高于女性,比率为2~5∶1。

原发性肝癌的病因及确切分子机制尚不完全清楚,目前认为其发病是多因素、多步骤的复杂过程,受环境和遗传双重因素影响。乙型肝炎病毒(HBV)和丙型肝炎病毒(HCV)感染、黄曲霉素、饮水污染、乙醇、肝硬化、亚硝胺类物质等都与肝癌发病相关。在我国,HBV感染是主要的致癌因素,而继发性肝癌(转移性肝癌)可通过不同途径,如随血液、淋巴液转移或直接浸润肝脏而形成疾病。

早期肝癌缺乏典型症状,中晚期肝癌常见的临床表现有肝区疼痛、肝肿大、黄疸、脾肿大、腹水、肝硬化征象,如发生转移还会出现相关症状与体征。

\subsection{肝癌的治疗}

手术是治疗肝癌的首选,包括根治性肝切除、姑息性肝切除等。对不能切除的肝癌可根据具体情况,采用术中肝动脉结扎、肝动脉化疗栓塞等手段。

\subsection{肝癌的药物治疗}

常用的药物是多柔比星(ADM)和5-FU,其他药物有奥沙利铂(L-OPH),吉西他滨、卡培他滨等,常用的化疗方案如表\ref{tab17-5}所示。

\begin{longtable}[]{lp{10cm}}
    \caption{肺癌的常用化疗方案}\\
    \label{tab17-5}\\
\toprule
方案 & 药物、剂量、给药方法和疗程\tabularnewline
\midrule
\endhead
PDF & DDP 20mg/m$^2$ ,静脉滴注,d1~4;ADM 40mg/m$^2$
,静脉滴注,d1;5-FU 400mg/m$^2$ ,静脉滴注,d1~4;干扰素50万/m$^2$
,sc(结膜下注射)d1~4;3~4周重复\tabularnewline
AFP & DDP 60mg/m$^2$ ;ADM 30mg/m$^2$ ;5-FU 600~800mg/m$^2$
,分别加0.9%生理盐水40~80mL,依次经肝动脉灌注,3~4周1次\tabularnewline
FAM & MMC 8~12mg/m$^2$ ;ADM 30mg/m$^2$ ;5-FU 600~800mg/m$^2$
,分别加0.9%生理盐水40~80mL,依次经肝动脉灌注,3~4周1次\tabularnewline
EMP & 5-FU 450mg/m$^2$ /d,静脉滴注,d1~5;MIT 6mg/m$^2$
,静脉滴注,d1;DDP 80mg/m$^2$ ,静脉滴注,d1;每4周重复\tabularnewline
\bottomrule
\end{longtable}