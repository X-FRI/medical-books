\chapter{合理用药}

长久以来,合理用药都是临床需要解决又无法圆满解决的一大难题,这主要是因为合理用药无客观准确的评价标准,完全取决于医师的学识、经验和道德。药物数量的激增,诊疗任务的繁重,临床医师很难有足够的时间去全面深入地了解每一种治疗药物,同时,复杂的临床情况,如患者的个体差异、病理生理的改变、心理因素及依从性等,无时不在干扰着医师的治疗决策和用药效果,从而使得药物疗效下降,药品不良反应和药物滥用现象越来越多。因此,在明确诊断的前提下,详细了解患者的个体情况,做到安全、有效、经济、方便地选用药物,对提高药物疗效、减少药品不良反应、减轻患者经济负担、节约社会资源,具有非常重要的现实意义。

\subsection{临床用药不合理的常见表现}

\subsubsection{忽视循证医学证实的用药方法}

为了追求用药的“新”与“贵”,临床上常忽视一些已经循证医学证明有价值的用药方法,使其得不到充分应用。例如,急性心肌梗死患者选用β受体阻断药防止病情恶化,急性支气管哮喘患者选用吸入性糖皮质激素治疗,儿童急性腹泻时使用有效、安全、经济的口服补盐液以防失水等。

\subsubsection{无明显用药指征时使用抗生素及激素类药物}

据调查,国内抗生素的合理应用率不到40%。滥用抗生素的情况无处不在,主要表现为:①无细菌或抗生素敏感病原体感染的疾病,如发热未明的患者,或已明确为病毒感染而又无合并细菌感染的患者,使用抗生素。②无感染指征的预防性应用抗生素。如外科病例几乎常规地将抗生素用于无菌手术前,甚至开始于手术前的好几天。③局部使用抗生素等。

\subsubsection{应用药物种类过多或过杂}

目前,临床治疗中合并应用多种药物日益普遍。合并用药的目的是提高疗效、扩大治疗范围或减少不良反应,因此,合并用药不当,可使药效减弱、毒性增高或出现严重不良反应,甚至引起药源性死亡。例如,磺酰脲类降糖药与氯噻酮合用时,可使降血糖的效果降低。然而,临床上合并用药更常见的是药品不良反应的增加。据调查,合并应用药物的种类愈多,不良反应的发生率也愈高。

\subsubsection{选药对患者缺乏安全性}

医师选用药物时,不仅要考虑用药的适应证,同时也要注意药物的禁忌证及易引起不良反应的生理或病理因素等。例如,新生儿易发生药物性溶血性贫血,因而不宜使用磺胺及呋喃类抗菌药物;老年人因生理性肾功能减退,肾小球滤过率降低,连续反复应用氨基糖苷类或与第一代头孢菌素合用,则易发生听力损害及肾功能衰竭;妊娠妇女如选药不当可导致畸胎等,这些现象都是因年龄不同或生理变化而选药不当导致的结果。病理状态下,特别是肝、肾疾患时的选药问题更不容忽视。此外,患者的用药史、药物过敏史等都是选药时必须注意的问题,否则将会引起药物的过敏反应或其他不良反应。

\subsubsection{给药方案的不合理}

给药方案是指给药途径、给药剂量和用药间隔时间的方案。许多医师认为,疾病一旦确诊,治疗用药只需“按章”执行而已,因此,“协定处方”等就应运而生。据了解,不合理用药产生的不良后果中,不合理的用药方案仍占重要比例。例如:①药物配伍不合理,产生体外的药物相互作用,使药物产生理化性质变化或疗效降低。如庆大霉素与青霉素类药物混合进行静脉滴注时,庆大霉素可被灭活。②剂量未个体化。患者的生理或病理状态不同,药物在体内的药动学参数可能发生变化,医师必须根据患者的情况来调整剂量或用药间隔时间,这对于治疗范围较窄的药物来说尤为重要,如地高辛、苯妥英钠等;否则,将会出现药理作用过强,甚至严重中毒。③忽视用药途径的药动学特征。不同的给药途径能使剂量相同的药物达到不同的血药浓度,甚至达到不同的治疗目的。例如,硫酸镁口服给药时,因为不吸收而仅作为容积性泻剂使用;但注射用药时,则可使神经-肌肉传导阻滞而具有抗惊厥效果。因此,正确选择给药途径,对保证疗效非常重要。

\subsection{合理用药的益处}

\subsubsection{合理用药可提高药物疗效}

临床治疗中常可以看到,对某一疾病,若选用的药物不同,或所选药物的剂量、给药方法及疗程不同,其疗效会出现明显差异。例如,使用青霉素G时,现主张分次给药,即在每日总给药量不变的情况下,分次将青霉素G加入到50~100mL葡萄糖溶液中,并于30min内静脉滴注完毕。这一给药方案更符合青霉素的杀菌作用特点,即在短时间内大量杀灭进入繁殖期的细菌,从而提高了疗效。又如,对需要中长期用药的患者,糖皮质激素(泼尼松、泼尼松龙)采用隔日早晨一次顿服法,更符合人体内分泌激素的时辰规律,既提高了疗效,又降低了药品不良反应。

\subsubsection{合理用药可降低药品不良反应,减少药源性疾病的发生}

临床医师不合理用药导致药品不良反应增加,主要源于以下两个方面:①对疾病的认识不正确,以致盲目选药。如对病毒感染者使用抗生素,或对发热未明者进行抗菌治疗,结果易引起细菌耐药性或产生其他不良反应。②对药物的药理学知识了解不够,不能正确合理地使用药物。如对肝、肾功能减退患者选用具有肝、肾毒性的药物,必然会加重患者的肝肾功能损害。因此,如果医师具有扎实的专业知识和药理学与药物治疗学知识,以及丰富的临床治疗经验,就能较好地做到合理用药,大大降低药品不良反应的发生率或不良反应的严重度。

\subsubsection{合理用药可降低医疗费用,减轻患者经济负担,节约社会资源}

随着大量新药的不断上市,临床治疗可供选择的药物品种越来越多。但新药因研究和开发费用高,故价格一般比老药贵。因此,所选药物除了安全有效外,还应考虑药物价格,后者对低收入患者来说不能忽视。例如,对可选用青霉素或广谱青霉素治疗有效的感染,就不必使用价格昂贵的第三代头孢菌素;大多数细菌感染用一种抗菌药物即可控制,就无须采用二联或三联疗法,这样,不仅降低了医疗费用,避免不必要的资源浪费,又可降低药品不良反应及细菌耐药性。同时,合理用药可加速疾病愈合,缩短住院或治疗时间,大大减轻了患者的经济负担,节约了社会资源。

\subsection{合理用药的基本原则}

合理用药的重要性如此明确,然而临床医师要真正做到合理用药却并非易事。特别是在病情诊断不明确,用药指征不足而又必须迅速做出决定的情况下,医师用药往往要冒极大的风险,这需要医师的学识、经验和智慧。近年来,循证医学的兴起以及临床药学服务的开展,为临床合理用药提高了依据和保障,极大地促进了临床治疗水平的提高。尽管目前合理用药没有具体客观的评价指标,但一般应符合以下一些基本原则。

\subsubsection{合理选药}

合理选药是根据患者的具体情况,安全、有效、经济、方便、及时地选用药物。药物都会有不良反应,特别在某些情况下,如年老体弱者、婴幼儿、病情严重者,药品不良反应会更严重。因此,临床用药时优先考虑的是药物的安全性。对同时患有多种疾病的患者选用药物时需要慎重考虑,所选药物既要对该病有效,又要不影响其他疾病的疗效。选用药物时,还应考虑到所选药物使用是否方便。使用方便的药物,患者的依从性好,其疗效也会较好,特别是对同时服用多种药物的患者尤为重要。药物价格高低也是衡量选药是否合理的一个重要指标。

\subsubsection{联合用药}

联合用药在临床治疗中已十分普遍,合理的联合可提高药物疗效,减少不良反应的发生。例如,在抗高血压、抗肿瘤及抗感染治疗中,有很多有效的联合用药方案,取得了较为满意的临床效果。反之,不合理的联合用药就会降低药物疗效,增加药品不良反应。因此,在联合用药时一定要注意以下几点:①详细了解病情,确定联合用药指征;②应熟知所用药物的作用特点及不良反应;③联合用药时各药物剂量、用法、疗程是否正确合理。

\subsubsection{用药个体化}

由于生物个体存在着差异性,一般的用药原则并不总是合适某些特殊的个体。有时即使是患有相同疾病的患者,因个人的条件、习性或病理生理的差异,对药物的反应就可能不同。所以,在具体治疗每一位患者时,应根据患者的具体情况用药。用药个体化主要是根据医师的治疗经验和治疗药物监测来决定。用药个体化的含义很广,如药物的选择以及药物的剂量、用法、疗程、联合用药等。

\subsubsection{合理停药}

在临床治疗中,根据病情的发展变化需要更换药物时,或治疗目的已达到而需停药时,合理停药就显得非常重要。若停药不合理,可引起停药后不良反应,出现所谓“反跳现象”,如糖皮质激素长期治疗后突然停药。有些药物的毒性作用在停药后仍可产生,如氨基糖苷类药物的耳、肾毒性。在联合用药中,由于药物之间的相互协同或相互制约关系,停用某药时,就必须相应减少或增加另一药物的剂量,建立新的平衡关系。例如,苯巴比妥的酶促作用可加速安乃近的代谢,两药合用时就应增加安乃近的用量;而当突然停用苯巴比妥时,酶促作用消失,安乃近就须相应减少用量。