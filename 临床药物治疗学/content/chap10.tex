\chapter{精神障碍的药物治疗}

\section{精神分裂症}

\subsection{定义}

精神分裂症(schizophrenia)是一类常见的精神症状复杂、至今未明确病理基础的重性精神障碍,多起病于青年或成年早期,具有知觉、思维、情感、认知、行为及社会功能多方面的障碍和精神活动不协调,一般没有意识障碍。自然病程多迁延,导致衰退和残疾。部分患者可痊愈或基本痊愈。

\subsection{流行病学}

根据世界卫生组织1992年所公布的资料,精神分裂症的年发病率为0.07‰~0.14‰。各国以及同一国家的不同地区精神分裂症的年发病率差异很大。如来自欧洲的资料表明,丹麦发病率较低,而英国发病率则较高。

\subsection{病因和发病机制}

精神分裂症的病因受到遗传因素、大脑结构异常、神经生化异常、环境及心理社会因素的影响,脑部影像学等研究资料显示,该病的多数患者脑发育还存在缺陷。生物化学和精神药理学的一些研究结果提示,精神分裂症的发病与脑内神经递质异常有关,如DA能活动过强、5-羟色胺(serotonin,5-HT)能功能紊乱、γ-氨基丁酸(GABA)和谷氨酸能系统也可能参与其生理病理过程。

\subsection{临床表现}

精神分裂症的临床症状复杂多样。不同个体、不同疾病类型、处于疾病的不同阶段,其临床表现可有很大差异。不过,这类患者均具有感知、思维、情感、意志及行为的不协调和脱离现实环境的特点。

\subsubsection{前驱期症状}

前驱期症状是指明显的精神症状出现前,患者所出现的一些非特异性的症状。这些症状不具有特异性,在青少年中并不少见,但更多见于发病前。最常见的前驱期症状可以概括为以下几个方面:①情绪改变:抑郁、焦虑、情绪波动、易激惹等;②认知改变:出现一些古怪或异常观念,学习或工作水平下降等;③对自我和外界的感知改变;④行为改变:如社会活动退缩或丧失兴趣,多疑敏感,社会功能水平下降等;⑤躯体改变:睡眠和食欲改变,乏力,活动和动机下降等。由于此时的患者在其他方面基本保持正常,且常常对这些症状有较为合理化的解释,故处于疾病前驱期的这些表现常不为家人重视。

\subsubsection{显症期症状}
\paragraph{阳性症状}

阳性症状是指异常心理过程的出现,普遍公认的阳性症状包括幻觉、妄想及紊乱的言语和行为(瓦解症状)。
\paragraph{阴性症状}

阴性症状是指正常心理功能的缺失,涉及情感、社会及认知方面的缺陷。最近,由美国国立精神卫生研究所组织的专家共识会议建议以下五条为精神分裂症的阴性症状条目,包括意志减退、快感缺乏、情感迟钝、社会退缩和言语贫乏。

\subsection{诊断}

目前,尚无对精神分裂症的权威诊断方法。一般主要根据病史及临床表现综合评定做出诊断,临床表现中主要根据精神检查的发现。按照ICD-10诊断标准,精神分裂症存在一些对诊断有特殊意义,并常常同时出现的症状群,常见的有以下几种。

(1)思维鸣响、思维插入或思维被撤走以及思维广播。

(2)明确涉及躯体或四肢运动,或特殊思维、行动或感觉的被影响、被控制或被动妄想;妄想性知觉。

(3)对患者的行为进行跟踪性评论,或彼此对患者加以讨论的幻听,或来源于身体某一部分的其他类型的听幻觉。

(4)与文化不相称且根本不可能的其他类型的持续性妄想,如具有某种宗教或政治身份,或超人的力量和能力(例如能控制天气,或与另一世界的外来者进行交流)。

(5)伴有转瞬即逝的或未充分形成的无明显情感内容的妄想,或伴有持久的超价观念,或连续数周或数月每日均出现的任何感官的幻觉。

(6)思潮断裂或无关的插入语,导致言语不连贯,或不中肯或语词新作。

(7)紧张性行为,如兴奋、摆姿势,或蜡样屈曲、违拗、缄默及木僵。

(8)“阴性”症状,如显著的情感淡漠、言语贫乏、情感反应迟钝或不协调,常导致社会退缩及社会功能下降,但必须澄清这些症状并非由抑郁症或神经阻滞剂治疗所致。

(9)个人行为的某些方面发生显著而持久的总体性质的改变,表现为丧失兴趣、缺乏目的、懒散、自我专注及社会退缩。

诊断精神分裂症通常要求在1个月或以上时期的大部分时间内确实存在属于上述(1)~(4)中至少一个(如不甚明确常需2个或多个症状)或(5)~(9)中来自至少两组症状群中的十分明确的症状才可诊断为精神分裂症。

\subsection{治疗}

\subsubsection{药物治疗}

抗精神病药物,又称神经松弛药,是指能够控制精神运动性兴奋,针对一些临床精神病症状具有治疗作用的一类药物。随着临床精神药理学研究的深入,提出按主要药理作用分类。目前,常用药物主要分为两大类:第一代抗精神病药物(多受体作用,D{2}
受体为主)及第二代抗精神病药物(双受体作用,5-HT/D{2}
)。药物治疗应系统而规范,强调早期、足量、足疗程,注意单一用药原则和个体化用药原则。一般首选第二代抗精神病药作为一线用药,第一代抗精神病药及氯氮平作为二线用药。
\paragraph{第一代抗精神病药}

主要包括吩噻嗪类、硫杂蒽类、丁酰苯类、苯甲酰胺类。常用药物如下。

(1)氯丙嗪(Chlorpromazine):是第一代抗精神病药最常用的代表药之一,隶属于吩噻嗪类抗精神病药。氯丙嗪有较强的镇静作用和抗幻觉、妄想、思维形式障碍作用。对淡漠退缩和焦虑抑郁的作用弱。嗜睡和锥体外系不良反应较明显,也可引起肝功能异常和使催乳素升高、泌乳、月经改变。氯丙嗪可引起过敏反应,一般身体健康的患者均可选用。口服:精神分裂症开始25~50mg,每日2次,酌情加量至300~600mg,每次增幅50~100mg;维持量100~300mg。肌注:兴奋不合作者用25~50mg,加东莨菪碱0.3mg肌内注射,每日1~3次。

(2)奋乃静(Perphenazine):奋乃静作用与氯丙嗪相似,镇静作用稍弱,对内脏不良反应较少,易引起锥体外系症状,常用于幻觉较明显的患者和有躯体疾患的患者。

(3)氟哌啶醇(Haloperidol):属丁酰苯类的代表药物,化学结构与氯丙嗪完全不同,能选择性拮抗D{2}
受体,抗精神病作用与氯丙嗪相似,但对自主神经及心、肝功能影响较小,锥体外系不良反应较明显且常见。口服:治疗精神分裂症从每日2~4mg开始,渐增至治疗量每日10~40mg,分2~3次口服。肌注:常用于兴奋躁动,每次5~10mg。每日1~3次。为避免发生急性肌张力障碍,常预防性使用东莨菪碱每次0.3mg,同时注射。

(4)舒必利:有较强的抗精神病作用。该药的特点是几乎没有镇静作用,所以没有嗜睡乏力等不良反应,适应于以淡漠退缩等阴性症状为主和木僵违怮等紧张症状为主的患者。锥体外系不良反应较少;神经内分泌不良反应较常见,如溢乳、月经异常等;增量过快可导致心电图改变;有很强的镇吐作用。口服治疗量为每日400~1000mg,分2~3次服用。紧张症状患者可用200~600mg稀释于500mL补液中静脉滴注,每日1次。
\paragraph{第二代抗精神病药}

5-羟色胺和多巴胺受体拮抗剂、多受体作用药、选择性D{2} /D{3}
受体拮抗剂或D{2} 、5-HT{1A} 受体部分激动剂和5-HT{2A}
受体拮抗剂,统称第二代抗精神病药。第二代抗精神病药除了阻断中枢多巴胺D{2}
受体外,对5-羟色胺2A受体也有较强的阻断作用,同时对阳性症状和阴性症状都有较好的疗效。治疗剂量下,较少或不产生锥体外系症状和催乳素水平升高相关症状。目前多数研究认为,第二代抗精神病药阻断D{2}
和5-HT{2A} 受体的高比率特性和依从性方面总体优于经典抗精神病药。

常用药物:

(1)氯氮平(clozapine):属二苯氧氮杂䓬
类的代表药物,是具有三环类结构的化合物,结构特征为中央七面环。氯氮平于1958年首先在瑞士合成,1960年代开始临床应用,此药为第二代抗精神病药,目前是我国许多地区治疗精神分裂症的首选药。

氯氮平同时拮抗DA受体和5-HT{2}
受体,有较强的抗精神病作用,锥体外系不良反应较少且轻;有很强的镇静作用;对精神分裂症阳性和阴性症状均有效,也可用于躁狂和兴奋状态。其他抗精神病药物治疗无效的难治性精神分裂症可换用此类第二代抗精神病药物治疗。主要不良反应是粒细胞缺乏症,需要监测血象,尤其是在使用初期,与剂量无关。常规为第1个月每周查1次粒细胞计数,第2个月每2周查1次,第3个月以后每月查1次,6个月后每3个月查1次。常见不良反应为抗胆碱能作用引起的流涎、便秘、心动过速,体重增加较明显,部分患者可出现心电图改变。口服易吸收,治疗剂量从25~50mg每日2次开始。治疗精神分裂症可增至每日200~600mg。

(2)利培酮:属于苯丙异噁唑衍生物,是继氯氮平之后的第二个非典型抗精神病药DA受体和5-HT{2}
受体平衡拮抗剂,药理作用与氟哌啶醇相似,但优于氟哌啶醇,对精神分裂症阳性、阴性症状和情感症状均有效。有锥体外系不良反应,但较氯丙嗪和氟哌啶醇轻;较突出的不良反应是高催乳素血症,女性可出现月经异常。口服易吸收,治疗剂量每日2~6mg。5或6d达稳态血药浓度。

(3)奥氮平:DA受体和5-HT{2}
受体平衡拮抗剂,与氯氮平作用相似,对精神分裂症阳性、阴性症状都有效,兼有抗焦虑作用。本药初始剂量为每日5~10mg,一般治疗剂量为每日10~20mg,有时需增至每日30mg,其半衰期长,可每日1次给药,治疗依从性较好。极少出现锥体外系反应和粒细胞缺乏症,对心血管系统影响也较轻。常见的不良反应有镇静、体位性低血压和体重增加。

(4)喹硫平:喹硫平为二苯并噻氮䓬
类衍生物。对5-HT{2}
受体的阻滞作用大于DA受体阻滞作用,对精神分裂症阳性、阴性症状都有效。其特点是锥体外系不良反应发生率很低,几乎没有抗胆碱能不良反应,无粒细胞缺乏不良反应。主要不良反应是嗜睡、头晕和体位性低血压,可引起催乳素水平暂时升高。

(5)齐拉西酮:是5-HT{2A} 和多巴胺D{2}
受体的强拮抗剂,对精神分裂症的阳性、阴性症状和情感症状有治疗效果。锥体外系不良反应少,对外周抗胆碱作用不明显,对糖、脂代谢影响小。适用于催乳素水平升高、体重明显增加、锥体外系反应阈值较低及迟发性运动障碍的患者。齐拉西酮重要的不良反应是{Q-T}
间期延长,可能与药物相互作用有关。故治疗期间,应避免与其他可能导致{Q-T}
间期延长的药物合用,并纠正可能增加心律失常风险的电解质紊乱等情况。其他不良反应有嗜睡、头晕、恶心和头重脚轻,偶有心动过速、便秘和体位性低血压。

(6)阿立哌唑:其药理作用与第一代和第二代抗精神病药均有所不同,为5-羟色胺-多巴胺系统稳定剂,既可以治疗精神分裂症的阳性症状,也可以治疗阴性症状和认知功能损害,与其他抗精神病药的疗效相当。常见不良反应是头痛、焦虑、困倦、兴奋、静坐不能和恶心等,对催乳素水平和体重无明显影响。

\subsubsection{心理治疗及精神康复}

对于精神分裂症患者,心理治疗主要应用于急性期发作以后。急性期患者精神症状活跃,受其影响,患者对疾病缺乏自知力,不能领悟心理治疗的要求,亦不能配合治疗的过程。但在巩固期和维持期,患者精神症状逐渐消失,自知力逐步恢复,接触改善,能进行交流学习,有了解疾病性质、提高识别能力的需要,也有学习应对社会歧视、改善人际交往和伴发情绪和行为问题的需要。

\subsubsection{其他治疗}

(1)改良电抽搐治疗:主要用以部分难治性精神分裂症患者,起效较快,急性症状控制后仍用药物治疗。其不良反应和并发症主要为可恢复的短期记忆受损、窦性心动过速、罕见呼吸窘迫、窒息等。实施要点:以阿托品静脉给药以降低迷走神经张力,减少呼吸道分泌;以速效麻醉剂行静脉诱导麻醉;以肌肉松弛剂静脉给药,以解除骨骼肌抽搐,再按常规ECT通电。

(2)精神外科治疗:指脑额叶白质切断术,只适用于少数经各种治疗无效而又难于管理的患者。

\section{心境障碍}

\subsection{定义}

心境障碍(mood
disorder)也称情感性精神障碍,是指由各种原因引起的因显著而持久的情感或心境改变为主要特征的一组疾病。临床上主要表现为情感高涨或低落,伴有相应的认知和行为改变,可有幻觉、妄想等精神病性症状。多数患者有反复发作的倾向,每次发作多可缓解,部分可有残留症状或转为慢性。还包括环性心境障碍和恶劣心境两种持续性心境障碍。

\subsection{病因和发病机制}

目前研究表明,发病与生物因素(遗传和生化异常-DA、5-HT异常)、心理社会因素(生活事件)和躯体疾病因素有关。患者回顾病情,多数人格都具有抑郁或躁狂素质特征。

通过对情感性精神障碍患者脑部的神经递质进行研究发现,这类患者出现不同程度的单胺类代谢紊乱。单胺类物质在机体中发挥重要生理作用,已知的物质主要包括儿茶酚胺和吲哚烷基胺,前者包括去甲肾上腺素(NE)和多巴胺(DA),后者主要是指5-羟色胺。这两类单胺类物质在脑组织内浓度高低变化与躁狂抑郁症有一定的关系。NE浓度过低因而不能兴奋脑内肾上腺素能受纳器时,就会出现抑郁症状;反之便出现躁狂表现。神经化学和药理学发现抑郁症患者脑内DA功能降低,而躁狂症患者脑内DA功能提高。由于脑内各种神经递质系统与神经内分泌系统和神经免疫系统之间存在十分复杂的联系,至今精神性精神障碍发病机制还在进一步探索之中。

\subsection{临床表现}

\subsubsection{躁狂发作}

躁狂发作的典型临床症状是心境高涨、思维奔逸和活动增多。
\paragraph{心境高涨}

患者主观体验特别愉快,自我感觉良好,整天兴高采烈,得意扬扬,笑逐颜开,洋溢着欢乐的风趣和神态,甚至感到天空格外晴朗,周围事物的色彩格外绚丽,自己亦感到无比快乐和幸福。患者这种高涨的心境具有一定的感染力,常博得周围人的共鸣,引起阵阵的欢笑。
\paragraph{思维奔逸}

表现为联想过程明显加速,自觉思维非常敏捷,思维内容丰富多变,思潮犹如大海中的汹涌波涛,有时感到自己舌头在和思想赛跑,言语跟不上思维的速度,常表现为言语增多,口若悬河,眉飞色舞,即使口干舌燥,声音嘶哑,仍要讲个不停。但讲话的内容较肤浅且凌乱不切实际,常给人以信口开河之感。
\paragraph{活动增多}

表现精力旺盛,兴趣范围广,动作快速敏捷,活动明显增多,且爱管闲事,整天忙忙碌碌,但做事常常虎头蛇尾,一事无成。对自己行为缺乏正确判断,常常是随心所欲。不考虑后果,随意挥霍钱财,注意打扮装饰,但并不得体。
\paragraph{躯体症状}

由于患者自我感觉良好,故很少有躯体不适体诉,常表现为面色红润,两眼有神,体格检查可发现瞳孔轻度扩大,心率加快,且有交感神经亢进的症状(如便秘)。因患者极度兴奋,体力过度消耗,容易引起失水、体重减轻等。
\paragraph{其他症状}

患者的主动和被动注意力均有增强,但不能持久,易为周围事物所吸引,急性期这种随境转移的症状最为明显。部分患者有记忆力增强,且无法抑制,多变动,常常充满许多细节琐事,对记忆的时间常失去正确的分界,以至与过去的记忆混为一谈而无连贯。

\subsubsection{抑郁发作}

抑郁发作临床上是以心境低落、思维迟缓、认知功能损害、意志活动减退和躯体症状为主。
\paragraph{心境低落}

主要表现为显著而持久的情感低落,抑郁悲观。患者终日忧心忡忡,郁郁寡欢、愁眉苦脸、长吁短叹。程度轻的患者感到闷闷不乐,无愉快感,凡事缺乏兴趣,任何事都提不起劲,感到“高兴不起来”;程度重的可痛不欲生,悲观绝望,有度日如年、生不如死之感,患者常诉说“活着没有意思”“心里难受”等。部分患者可伴有焦虑、激越症状,特别是更年期和老年抑郁症患者更明显。典型的抑郁心境病例具有晨重夜轻节律改变的特点,即情绪低落在早晨较为严重,而傍晚时可有所减轻,如出现则有助于诊断。
\paragraph{思维迟缓}

患者思维联想速度缓慢,反应迟钝,思路闭塞,自觉“脑子好像是生了锈的机器”或“脑子像涂了一层糨糊一样”。临床上可见主动言语减少,语速明显减慢,声音低沉,对答困难,严重者交流无法顺利进行。
\paragraph{认知功能损害}

研究认为抑郁症患者存在认知功能损害。主要表现为近事记忆力下降,注意力障碍(反应时间延长),警觉性增高,抽象思维能力差,学习困难,语言流畅性差,空间知觉、眼手协调及思维灵活性等能力减退。认知功能损害导致患者社会功能障碍,而且影响患者远期预后。
\paragraph{意志活动减退}

患者意志活动呈显著持久的抑制。临床表现行为缓慢,生活被动、疏懒,不想做事,不愿和周围人接触交往,常独坐一旁,或整日卧床,不想去上班,不愿外出,不愿参加平常喜欢的活动和业余爱好,常闭门独居、回避社交。严重时,连吃、喝、个人卫生都不顾,蓬头垢面、不修边幅,甚至发展为不语、不动、不食,可达木僵状态,称为“抑郁性木僵”,但仔细精神检查,患者仍流露痛苦抑郁情绪。伴有焦虑的患者,可有坐立不安、手指抓握、搓手顿足或踱来踱去等症状。严重的患者常伴有消极自杀的观念或行为。
\paragraph{躯体症状}

在抑郁发作时很常见。主要有睡眠障碍、乏力、食欲减退、体重下降、便秘、身体任何部位的疼痛、性欲减退、阳痿、闭经等。躯体不适的可涉及各器官,如恶心、呕吐、心慌、胸闷、出汗等。自主神经功能失调的症状也较常见。
\paragraph{其他}

抑郁发作时也可出现人格解体、现实解体及强迫症状。

\subsection{诊断}

\subsubsection{躁狂发作}

ICD-10对轻躁狂发作和躁狂发作的诊断标准分别进行了描述。
\paragraph{轻躁狂}

心境高涨或易激惹,对个体来讲已达到肯定异常程度,且至少持续4d。必须具备以下3条,且对个人日常的工作及生活有一定的影响:①活动增加或坐卧不宁;②语量增多;③注意力集中困难或随境转移;④随眠需要减少;⑤性功能增强;⑥轻度挥霍或行为轻率、不负责任;⑦社交活动增多或过分亲昵。
\paragraph{躁狂发作}

心境明显高涨,易激惹,与个体所处环境不协调。至少具有以下3条(若仅为易激惹,需4条):①活动增加,丧失社会约束力以至行为出格;②言语增多;③意念飘忽或思维奔逸(语速增快、言语追促)的主观体验;④注意力不集中或随境转移;⑤自我评价过高或夸大;⑥睡眠需要减少;⑦鲁莽行为(如挥霍、不负责任,或不计后果的行为等);⑧性欲亢进。严重者可出现幻觉、妄想等精神病性症状。

严重损害社会功能,或给别人造成危险或不良后果。病程至少已持续1周。排除器质性精神障碍,或精神活性物质和非成瘾物质所致的类躁狂发作。

\subsubsection{抑郁发作}

在ICD-10中,患者通常具有心境低落、兴趣和愉快感丧失、精力不济或疲劳感等典型症状。其他常见的症状是:①集中注意和注意的能力降低;②自我评价降低;③自罪观念和无价值感(即使在轻度发作中也有);④认为前途暗淡悲观;⑤自伤自杀的观念或行为;⑥睡眠障碍;⑦食欲下降。病程持续至少2周。

\subsubsection{双相情感障碍}

双相情感障碍的诊断以躁狂发作和抑郁发作为基础,在整个病程中,曾经出现过典型的躁狂发作和抑郁发作,并且目前的发作情况符合某一型躁狂发作或抑郁发作的表现形式。符合这类诊断标准的患者诊断为双相情感障碍。

\subsubsection{环性心境障碍}

环性心境障碍是指反复出现轻度心境高涨或低落,但不符合躁狂或抑郁发作症状标准。心境不稳定至少2年,其间有轻度躁狂或轻度抑郁的周期,可伴有或不伴有心境正常间歇期;社会功能受损较轻。

\subsubsection{恶劣心境障碍}

恶劣心境障碍是慢性的心境低落,无论从严重程度还是一次发作的持续时间,目前均不符合轻度或中度复发性抑郁标准,同时无躁狂症状。至少2年内抑郁心境持续存在或反复出现,其间的正常心境很少持续几周;社会功能受损较轻,自知力完整或较完整。

\subsection{治疗}

\subsubsection{药物治疗}

情感性精神障碍最主要的治疗药物是心境稳定剂和抗抑郁药。对于躁狂发作以及双相情感障碍躁狂相的患者,使用心境稳定剂为主;有明显兴奋躁动的患者,可以合并抗精神病药物,包括经典抗精神病药氟哌啶醇、氯丙嗪和非典型抗精神病药奥氮平、喹硫平、利培酮、齐拉西酮、阿立哌唑等。对于双相情感障碍抑郁相的患者,原则上不主张使用抗抑郁药物,因其容易诱发躁狂发作、快速循环发作或导致抑郁症状慢性化,对于抑郁发作比较严重甚至伴有明显消极行为者、抑郁发作在整个病程中占据绝大多数者以及伴有严重焦虑、强迫症状者可以考虑在心境稳定剂足量治疗的基础上,短期合并应用抗抑郁药,一旦上述症状缓解,应尽早减少或停用抗抑郁药。
\paragraph{心境稳定剂}

心境稳定剂是指对躁狂或抑郁具有治疗或预防复发作用且不引起躁狂或抑郁转相,或引起频繁发作或躁狂与抑郁混合发作的药物。这类药物用于治疗躁狂发作和双相情感障碍,包括对躁狂相和抑郁相的治疗以及预防复发。

常用药物:

(1)碳酸锂:是最经典、疗效最可靠的心境稳定剂。碳酸锂被FDA批准的适应证有躁狂发作、双相障碍患者的维持治疗。其他还有双相障碍、抑郁症发作(辅助用药)、血管性头痛和中性粒细胞减少症。起效时间1~3周。治疗开始前应检查肾功能并确定是否肥胖。治疗过程中要监测血锂浓度和体重,对于体重增加超过5%者,要注意是否发生糖尿病、血脂蛋白异常,或考虑换用其他药物。不良反应有共济失调、构音困难、谵妄、震颤、记忆力问题、多尿、烦渴、腹泻、恶心、体重增加、皮疹、白细胞计数增多等。严重的不良反应有肾功能损害、肾源性糖尿病、心律不齐、心血管改变、心动过缓、低血压、T波低平或倒置,罕见癫痫
发作。出现震颤时可加用普萘洛尔20~30mg,每日2或3次。急性期常用剂量为每日1800mg,分次服用。维持治疗剂量为每日900~1200mg,分次服用。但因个体差异大,最好测定患者血锂浓度来确定治疗量。急性期血锂浓度维持在0.6~1.2mmol/L;维持期为0.4~0.8mmol/L;老年患者治疗血锂浓度不超过1.0mmol/L为宜。起始剂量为250mg,每日2或3次,根据血药浓度逐渐增加剂量。药物的治疗剂量与中毒剂量接近,容易发生中毒。中毒症状包括震颤、共济失调、腹泻、恶心、过度镇静。严重锂中毒可引起昏迷和死亡。不推荐用于严重肾功能损害和心脏疾病的患者。服排钠利尿剂者及大量出汗者可增加锂盐的毒性。老年患者和器质性疾病患者在治疗剂量就可能出现神经毒性反应,如谵妄和其他精神状态改变,应低剂量治疗。该药的优势是治疗欣快性躁狂、难治性抑郁,减少自杀的危险性,与新型抗精神病药物和(或)心境稳定剂如抗癫痫
药丙戊酸盐合用效果好。缺点是用于治疗烦躁性躁狂、混合性躁狂和快速循环型躁狂、双相障碍抑郁相效果较差。对于预防躁狂发作的效果比预防抑郁发作的效果好。

(2)丙戊酸盐(Valproate):为抗癫痫
药和心境稳定剂。经FDA批准的适应证有躁狂、单独出现的或与其他类型的癫痫
相关的复杂性部分发作、预防偏头痛,其他还有双相障碍的维持治疗、双相抑郁、精神病、精神分裂症。靶症状是情绪不稳定、预防偏头痛、颇痈部分发作。对于急性躁狂,数日内起效;作为心境稳定剂,需数周到数月发挥最佳作用。治疗前必须进行血小板计数和肝功能检查,测体重。在治疗期间要监测体重,若体重增加超过5%,应评估是否有糖尿病、脂蛋白异常或考虑换药。不良反应有镇静、震颤、头晕、共济失调、衰弱、头痛、腹痛、恶心、呕吐、腹泻、食欲降低、便秘、消化不良、体重增加、秃头症、脂质调节异常等。严重的不良反应有罕见的中毒性肝炎和胰腺炎,有时可致死。合用普萘洛尔20~30mg,每日2或3次,可减少震颤。剂量范围:躁狂发作为每日1200~1500mg。治疗非急性躁狂时,起始剂量为每日250~500mg,缓慢加量。治疗急性躁狂时,起始剂量为每日1000mg,快速加量。肝功能损害患者禁用;老年患者应减量,加量应缓慢;用于儿童时应严密监测;妇女在怀孕前3个月服用可致胎儿畸形;哺乳期妇女使用安全。该药的优势是治疗双相障碍的躁狂相,与锂盐和(或)新型抗精神病药物合用效果好;缺点是不适用于双相障碍的抑郁相,会导致镇静和体重增加,与多种药物有相互作用,引起多种不良反应,不适用于孕妇。

(3)卡马西平(Carbamazepine):为抗癫痫
药和心境稳定剂,是电压敏感的钠通道拮抗剂。经FDA批准的适应证有精神运动性癫痫
、癫痫
大发作、混合型癫痫
,其他还有双相情感障碍、精神分裂症(辅助用药)。对于急性躁狂,需数周才能起效。心境稳定剂的作用需数周到数月才能达到最佳效果。靶症状除了癫痫
,还有不稳定的情绪,特别是躁狂。治疗开始需查血常规、肝肾功能和甲状腺功能。常见的不良反应有过度镇静、头晕、意识障碍、头痛、恶心、呕吐、腹泻、视力模糊、良性白细胞减少症及皮疹。严重不良反应有罕见的再生障碍性贫血、粒细胞缺乏症、严重的皮肤病反应、心脏问题、诱发精神病性症状或躁狂、抗利尿激素分泌失调综合征、癫痫
大发作频率增加(在不典型意识丧失的癫痫
患者中)。治疗期间必须监测有无异常出血或瘀肿、口周疼痛、感染、发热或咽部疼痛。有骨髓抑制的患者禁用,不能与单胺氧化酶抑制剂(MAOIs)合用。有肾脏疾病的患者必须减量,肝功能损害和心脏功能损害的患者慎用。老年人和儿童减量。常用剂量为每日400~1200mg,6岁以下的儿童为每日10~20mg/kg。停药过快会使双相障碍的症状复发。卡马西平主要经肝脏CYP450酶3A4亚型代谢,肾脏排泄,其活性代谢产物为10,11-环氧-卡马西平。由于卡马西平会激活CYP450酶3A4亚型活性,可加快自身代谢,经常需要增加剂量。此药与奈法唑酮、氟伏沙明、氟西汀合用时可增加卡马西平的血药浓度。卡马西平可增加苯妥英和扑米酮的血药浓度,降低氯氮平、华法林和氟哌啶醇等药的浓度。用于妊娠前3个月时可能导致胎儿先天性异常。建议哺乳期妇女停药。对锂盐或其他心境稳定剂治疗无效的患者可能有效,作为治疗躁狂的二线或三线用药。
\paragraph{抗抑郁药}

抗抑郁药包括三环类抗抑郁药、单胺氧化酶抑制剂、四环类抗抑郁药以及新型抗抑郁药几种。

1)三环类抗抑郁药

(1)阿米替林(Amitriptyline):属于三环类抗抑郁药。适应证有抑郁症、更年期忧郁症、恶劣心境以及器质性精神障碍伴发的抑郁症状,特别是对伴有失眠的抑郁症效果较好。不良反应:抗胆碱能作用如口干、便秘、视力模糊、排尿困难;心血管方面有心动过速、直立性低血压等;其他如头昏、躁狂样兴奋、激动、肝功能异常。有严重心脏病、青光眼、尿滞留、前列腺肥大者禁用;不宜与单胺氧化酶抑制剂合用;不宜与抗胆碱能药物合用。半衰期为14~46h。口服应从小剂量开始,逐渐加量。治疗剂量为每日100~300mg,分2~3次服,剂量小时也可每晚1次服用。

(2)氯米帕明(Clomipramine):属于三环类抗抑郁药。有较强地抑制中枢神经系统5-HT再吸收的作用。适应证有抑郁症、强迫症和恐惧症和焦虑症,能够消除抑郁情绪,唤起工作及社交活动的兴趣,振奋情绪,恢复活力;还可用于治疗慢性疼痛。不良反应有轻微乏力、困倦、头晕、口干、口苦、便秘、食欲不振、视力模糊、排尿困难、直立性低血压等,偶有皮肤过敏反应及肝功能异常。高龄、青光眼、前列腺肥大患者慎用,不宜与单胺氧化酶抑制剂和抗胆碱能药物合用。半衰期为20h,口服,开始剂量为25mg,每日3次,治疗剂量为每日100~300mg,分2或3次服用。

2)单胺氧化酶抑制剂(MAOIs)

吗氯贝胺(Moclobermide):属于MAOIs。曾用于治疗非典型的抑郁症,由于会引起对肝实质损害的严重不良反应,目前已极少使用。常见的不良反应为失眠、口干、便秘、排尿困难、低血压等。与富含酪胺的食物如奶酪、酵母、鸡肝、酒类等合用时可发生高血压危象,一般不与三环类抗抑郁药合用。

3)四环类抗抑郁药

马普替林(Maprotiline):在结构上比经典的三环类抗抑郁药物增加了一个内环,也被称为四环类抗抑郁药物。作用机制以抑制NE在突触前神经元的再摄取为主。抗抑郁作用强,为广谱抗抑郁药,适用于迟滞性抑郁症、激越性抑郁症和其他各类抑郁症。能够提高情绪,缓解焦虑、激动和精神运动阻滞。不良反应有口干、便秘、视力模糊、心动过速、头晕、震颤、睡眠障碍、皮肤过敏,偶可诱发躁狂。青光眼、前列腺肥大、癫痫
及心、肝、肾功能不良者慎用;不宜与单胺氧化酶抑制剂和抗胆碱能药物合用;降低胍乙啶的降压作用;孕妇及哺乳期妇女禁用。老年患者剂量酌减。半衰期51h。口服,从小剂量开始,逐渐加量,治疗剂量为每日100~300mg,分3次服,剂量小时也可以每晚1次服用。

4)新型抗抑郁药

(1)氟西汀(Fluoxetine):为新型抗抑郁药,属于SSRI类抗抑郁药。靶症状为抑郁情绪、动力和兴趣缺乏、焦虑、睡眠障碍。氟西汀与奥氮平合用可以治疗双相抑郁,难治性单相抑郁和精神病性抑郁。通常需要3~4周起效,不良反应有性功能障碍、胃肠道反应(食欲降低、恶心、腹泻、便秘、口干)、失眠、镇静、激越、震颤、头痛、头晕、出汗、出血等。严重的不良反应有罕见的癫痫
发作、诱发躁狂和激活自杀观念。治疗抑郁症和焦虑症时常用剂量为每日20~80mg,治疗贪食症时为每日60~80mg。母药半衰期为2或3天,活性代谢产物去甲氟西汀为2周。与三环类抗抑郁药合用时增加三环类抗抑郁药的血浆水平,因此应减少后者剂量。不能与单胺氧化酶抑制剂合用。肝脏损害和老年患者要减量。该药的优点是用于不典型抑郁症(睡眠过多、食欲增加)、疲乏和精力差的患者,合并进食和情绪障碍的患者,患有强迫症或抑郁症的儿童。缺点是不适用于治疗厌食、激越及失眠的患者;起效相对较慢。

(2)帕罗西汀(Paroxetine):为新型抗抑郁药,属于SSRI类抗抑郁药。失眠或焦虑在治疗早期就可缓解。治疗抑郁作用需2~4周才出现,若治疗6~8周仍然无效,需要增加剂量或判定无效。不良反应有性功能障碍、胃肠道反应(食欲降低、恶心、腹泻、便秘、口干)、失眠、镇静、激越、震颤、头痛、头晕、出汗等。严重的不良反应有罕见的癫痫
发作、诱发躁狂和激活自杀观念。剂量范围每日20~60mg,起始剂量为每日10~20mg,需等待数周才能决定是否有效,每周加量10mg。停药时应缓慢,以免出现撤药反应。肝肾功能损害和老年患者应减少剂量。该药的优势是治疗时伴有焦虑和失眠的患者,以及焦虑抑郁混合的患者。缺点是不适用于睡眠过多的患者、AD患者、认知障碍患者以及伴有精神运动性迟滞、疲乏、精力差的患者。

(3)氟伏沙明(Fluvoxamine):为新型抗抑郁药,属于SSRI类抗抑郁药。靶症状是抑郁情绪和焦虑。通常需要2~4周起效,部分患者在使用早期就可改善睡眠或焦虑。不良反应类似于帕罗西汀。严重的不良反应有罕见的癫痫
发作、诱发躁狂和激活自杀观念。治疗强迫症常用剂量为每日100~300mg,治疗抑郁症为每日100~200mg。起始剂量为每日50mg,4~7d每日增加50mg,直至获得最佳疗效。最高剂量为每日300mg,与三环类抗抑郁药、卡马西平和苯二氮䓬
类药物合用时增加合用药物的血浆水平,应减少合用药物的剂量。不应与MAOIs合用。用于肝脏损害的患者时应减小剂量。该药的优势是治疗抑郁焦虑混合的患者,可以快速出现抗焦虑和治疗失眠的作用,还可用于治疗精神病性抑郁和妄想性抑郁;缺点是不能用于有肠易激综合征和多种胃肠道不适的患者。剂量需要滴定以及每日服药2次。

(4)舍曲林(Sertraline):为新型抗抑郁药,属于SSRI类抗抑郁药。靶症状是抑郁情绪、焦虑、睡眠障碍、惊恐发作、回避行为、再经历、警觉性增高。在治疗早期,部分患者可出现精力和活动增加。治疗作用需2~4周后才可出现;若治疗6~8周仍然无效,需要增加剂量或判定无效。不良反应类似于氟西汀和帕罗西汀。严重的不良反应有罕见的癫痫
发作、诱发躁狂和激活自杀观念。剂量范围为每日50~200mg。有肝脏损害的患者应减量。老年患者剂量要小,加药应慢。该药的优势是治疗不典型抑郁(睡眠过多、食欲增加),可用于老年患者,对疲乏和精力差的患者效果较好。缺点是不宜用于伴有失眠、肠易激综合征的患者。剂量需要滴定。

(5)西酞普兰(Citalopram):为新型抗抑郁药,属于SSRI类抗抑郁药。靶症状有抑郁情绪、焦虑、惊恐发作、回避行为、再经历以及警觉性增高,其他还有睡眠障碍,包括失眠或睡眠过多。不良反应类似于氟西汀和帕罗西汀。严重的不良反应有罕见的癫痫
和诱发躁狂。常用剂量为每日20~60mg,起始剂量为每日20mg,缓慢加量。该药的优点是较其他抗抑郁药更易耐受,可用于老年患者及使用其他SSRIS过度激活或镇静的患者,药物较少对肝脏CYP450酶各种亚型起抑制作用。具有量效关系,剂量需要滴定以达到最佳疗效。

(6)文拉法辛(Venlafaxine):为新型抗抑郁药,属于SNRI类抗抑郁药。靶症状是抑郁情绪,精力、动力和兴趣降低,睡眠障碍,焦虑。起效时间通常需要2~4周;如治疗6~8周后仍然无效,需要增加剂量或判定无效。不良反应随着剂量的增加而增加,常见有头痛、神经质、失眠、镇静、恶心、腹泻、食欲减退、性功能障碍、衰弱、出汗等,还可见抗利尿激素分泌异常综合征、剂量依赖性高血压,严重罕见的不良反应有癫痫
及诱发躁狂和激活自杀观念。常用剂量范围:治疗抑郁症时为每日75~225mg,缓释剂为顿服,非缓释剂分成2或3次服用。治疗GAD时剂量为每日150~225mg,起始剂量为75mg(缓释剂)或25~50mg(非缓释剂),每4d的加药量不应超过每日75mg,直至出现最佳效果。最大剂量可达每日375mg,应缓慢停用。该药的优势是治疗迟滞性抑郁,不典型抑郁及伴焦虑的患者,其治疗抑郁症的缓解率较SSRI高;缺点是不能用于高血压或临界高血压患者。

(7)曲唑酮(Trazodone):为新型抗抑郁药,属于SARI类抗抑郁药。靶症状是抑郁、焦虑、睡眠障碍。治疗失眠作用起效快,治疗抑郁的作用需2~4周。若治疗6~8周仍然无效,需要增加剂量或判定无效。治疗失眠时可长期使用,因无证据表明会产生耐受性、依赖或撤药症状。不良反应有恶心、呕吐、水肿、视力模糊、便秘、口干、头晕、镇静、疲乏、头痛、共济失调、震颤、低血压、晕厥。长期治疗时罕见慢性心律过缓及皮疹。严重的罕见不良反应有阴茎持续勃起、癫痫
、诱发躁狂或激活自杀观念。剂量范围为每日150~600mg。单药治疗抑郁症时,起始剂量为每日150mg,分次服用,每3或4日加药量为每日50mg,分2次服用。慎用于有肝脏损害的患者和儿童;不推荐用于心肌梗死恢复期患者;老年患者应减量;孕妇妊娠前3个月避免使用,哺乳期应停止服药。该药的优势是治疗失眠时不会产生依赖,辅助其他抗抑郁药治疗残留的失眠和焦虑症状、伴焦虑的抑郁症患者,极少引起性功能障碍;缺点是不适用于乏力、睡眠过多的患者和难以忍受镇静作用的患者。

(8)米氮平(Mitrazapine):为新型抗抑郁药,属于NaSSA类抗抑郁药。该药对重度抑郁和明显焦虑、激越的患者疗效明显且起效较快,对患者的食欲和睡眠改善明显;过度镇静和引起体重增加是较为突出的不良反应。起效时间:对失眠和焦虑的作用可短期内见效,但对抑郁的治疗作用通常需要2~4周;若6~8周内无效,应增加剂量或判定无效。在治疗过程中应监测体重。不良反应有口干、便秘、食欲增加、体重增加、镇静、头晕、多梦、意识障碍、流感样症状(可能是由于白细胞或粒细胞计数低)、低血压。严重的不良反应有罕见的癫痫
、诱发躁狂或激活自杀观念。剂量范围为每日15~45mg,晚上服用。起始剂量为每日15mg,1或2周增加剂量直至出现最佳效果,最高剂量为每日45mg。慎用于心、肝、肾功能损害的患者,老年患者要减量。该药的优势是治疗特别担心性功能障碍患者、症状性焦虑患者和合并使用药物的患者;可作为增效剂增加其他抗抑郁药的效果;缺点是不宜用于担心体重增加的患者和精力差的患者。

(9)瑞波西汀(Reboxetine):为新型抗抑郁药,属于NRI类抗抑郁药。靶症状为抑郁情绪、精力差、动力缺乏和兴趣降低、自杀观念、认知障碍、精神运动性迟滞。起效时间通常需要2~4周;若治疗6~8周,抑郁情绪仍无改善,应增加剂量或判定无效。不良反应有失眠、头晕、焦虑、激越、口干、便秘、尿潴留、性功能障碍以及剂量依赖性低血压。严重的不良反应有罕见的癫痫
和诱发躁狂,激活自杀观念。常用剂量范围为每日8mg,分2次服用,最大剂量为每日10mg;起始剂量为每日2mg,分2次服用,1周后加至每日4mg。心脏疾病患者慎用;肝肾疾病及老年、儿童患者慎用;不推荐用于孕妇和哺乳期妇女。该药的优点是用于治疗疲倦、无动力的患者、有认知障碍患者和精神运动性迟滞患者,其改善社会功能和职业功能的效果较SSRI好;缺点是每日需要服药2次。

\subsubsection{电抽搐治疗或改良电抽搐治疗}

对于有严重消极自杀言行或抑郁性木僵的患者,应首选电抽搐或改良电抽搐治疗;对使用抗抑郁药治疗无效的患者也可采用电抽搐治疗。电抽搐治疗见效快、疗效好,6~12次为一疗程。电抽搐治疗后仍需用药物维持治疗。

\subsubsection{重复经颅磁刺激治疗}

重复经颅磁刺激治疗是20世纪90年代初应用于精神科临床研究的物理治疗方法,其基本原理是磁场穿过皮肤、软组织和颅骨,在大脑神经中产生电流和引起神经元的去极化,从而产生生理效应。

\subsubsection{脑深部电刺激}

脑深部电刺激是一种神经外科手术疗法,刺激器是如同起搏器样的装置,或者将刺激电极植入基底神经核区,或背侧丘脑,或底丘脑核区,以高频电刺激打断神经、精神疾病的异常神经活动。

\subsubsection{心理治疗}

在药物治疗的同时常合并心理治疗,尤其是有明显心理社会因素作用的抑郁发作患者及轻度抑郁或恢复期患者。

\section{神经症}

\subsection{定义}

神经症性障碍是指病因、发病机制和临床表现颇不一致的一组精神障碍。包括恐惧性障碍、焦虑性障碍、强迫症、分离(转换性)障碍、躯体形式障碍、神经衰弱。

\subsection{流行病学}

WHO领导下的精神卫生调查协作组于2001---2003年对美洲、欧洲、中东、非洲和亚洲的14个国家的60463名成人进行了精神障碍相关调查,发表了世界精神卫生报告,其中焦虑性障碍的年患病率如下表\ref{tab10-1}\footnote{*表示不包括强迫性障碍;\#表示不包括特殊恐惧症。}。

\begin{table}
    \centering
\caption{不同国家和地区的年患病率}
\label{tab10-1}
\begin{tabular}{cccc}
\toprule
国家和地区 & 年患病率 & 国家和地区& 年患病率\\
\midrule
哥伦比亚&10.0&西班牙&5.9\\
墨西哥&6.8*&乌克兰&7.1*#\\
美国&18.2&黎巴嫩&11.2\\
比利时&6.9&尼日利亚&3.3\\
法国&12.0&日本&5.3*\\
德国&6.2&中国北京&3.2*\\
意大利&5.8&中国上海&2.4*\\
荷兰&8.8&&\\
\bottomrule
\end{tabular}
\end{table}

这次调查采用DSM-IV诊断标准,其中焦虑性障碍包括广场恐惧症、广泛性焦虑障碍、强迫性障碍、惊恐障碍、创伤后应激障碍、社交恐惧症和特殊恐惧症。

\subsection{病因和发病机制}

神经症的病因受到遗传因素、素质因素、生理因素和心理社会因素的影响,双生子研究表明神经症呈明显的家族聚集性;同时患者在发病前有相应的人格倾向;并且存在明显的生理和心理社会因素。

\subsection{临床表现}

\subsubsection{恐惧性障碍}

恐惧症的临床表现很多,恐惧对象已达数百种之多,而且多以恐惧对象作为疾病名称,如飞行恐惧症、医牙恐惧症等,ICD-10将其归纳为以下三类。
\paragraph{场所恐惧症}

主要临床表现:①不敢进入商店、公共汽车、剧院、教室等公共场所和人群集聚的地方,如果患者冒险去尝试面对这些恐惧场景,就会出现各种焦虑症状,甚至惊恐发作。②焦虑症状和惊恐发作常导致患者产生回避行为,甚至根本不敢出门,对配偶和亲属的依赖突出。恐惧发作时还可伴有抑郁、强迫及人格解体等症状。这种表现形式在西方最常见,妇女患者尤多,多在20~30岁起病。
\paragraph{社交恐惧症}

主要表现在社交场合下感到害羞、局促不安、尴尬、笨拙,怕成为人们耻笑的对象。他们不敢在人们的注视下操作、书写或进食;他们害怕聚会,害怕与人近距离的相处,更害怕组织以自己为中心的活动;他们不敢当众演讲,不敢与重要人物谈话,担心届时会脸红,被称为赤面恐惧。有的患者看着别人眼睛时,害怕并回避与别人的视线相遇,此称对视恐惧。他们没有牵连观念,对周围现实的判断并无错误,只是不能控制自己不合理的情感反应和回避行为,并因此苦恼。患者恐惧的对象可以是生人,也可以是熟人,甚至是自己的亲属、配偶。较常见的恐惧对象是异性、严厉的上司和未婚夫(妻)的父亲等。大多在17~30岁期间发病,男女发病率相近。患者若被迫进入社交场合时,便产生严重的焦虑反应,惶然不知所措。
\paragraph{特定恐惧症}

也称单一恐惧症,指患者对某一具体的物体、动物有一种不合理的恐惧。特定恐惧症常起始于童年,如恐惧某一小动物,在儿童中很普遍,只是这种恐惧通常随着年龄增长而消失。为何少数人一直持续到成人,目前尚无法解释。不祥物恐惧(如棺材、坟堆、血污)在正常人中也不少见,不同的只是没有患者那种典型的回避行为及强烈的情绪和自主神经反应。单一恐惧症的症状恒定,多只限于某一特殊对象,如恐惧昆虫、老鼠或刀剪等物品,既不改变,也不泛化。但在部分患者,却可能在消除了对某一物体的恐惧之后,又出现新的恐惧对象。

\subsubsection{焦虑性障碍}

焦虑性障碍主要症状为焦虑的情绪体验、自主神经功能失调及运动性不安。临床上常见有急性焦虑和慢性焦虑两种表现形式。
\paragraph{急性焦虑(惊恐障碍)}

(1)惊恐发作:是一种突如其来的惊恐体验,仿佛窒息将至、疯狂将至、死亡将至。患者宛如濒临末日,或奔走,或惊叫,惊恐万状、四处呼救。惊恐发作时伴有严重的自主神经功能失调,通常起病急陡,终止也迅速。一般持续数十分钟便自发缓解。发作过后患者仍心有余悸,不过焦虑的情绪体验不再突出,而代之以虚弱无力,需经若干天才能逐渐恢复。

(2)预期焦虑:大多数患者在反复出现惊恐发作之后的间歇期,常担心再次发病,因而惴惴不安,也可出现一些自主神经功能活动亢进的症状。

(3)求助和回避行为:惊恐发作时,由于强烈的恐惧体验,患者常立即要求紧急帮助。在发作间歇期,常担心发病时得不到帮助,而主动回避一些活动,如不愿单独出门,不愿到人多的热闹场所等。
\paragraph{慢性焦虑(又称广泛性焦虑或浮游性焦虑)}

(1)精神性焦虑:精神上的过度担心是该病的核心,且其担心与关注涉及不同方面。表现为未来可能发生的、难以预料的某种危险或不幸事件经常担心。可表现为自由浮动性焦虑、预期焦虑,并伴有觉醒性增高(如对外界刺激敏感、易于出现惊跳反应)、注意力难于集中、难以入睡和(或)易惊醒、易激惹等。

(2)躯体性焦虑:主要表现为运动不安与肌肉紧张。运动不安可表现为搓手顿足、静坐不能、无目的的小动作增多等。肌肉紧张可表现为主观上的一组或多组肌肉不舒服的紧张感,严重时有肌肉酸痛,多见于胸部、颈部及肩背部肌肉,紧张性头痛也很常见,有的患者可出现肢体的震颤。

(3)自主神经功能紊乱:表现为心悸、心慌、胸闷、呼吸迫促,出汗,口干,便秘、腹泻,尿频、尿急,皮肤潮红或苍白,阳痿、早泄、月经紊乱等症状。

(4)其他症状:可合并疲劳、抑郁、强迫、恐惧、惊恐发作及人格解体等症状,但这些症状常不是疾病的主要临床相。

\subsubsection{强迫症}
\paragraph{强迫观念}

(1)强迫怀疑:对已完成某件事的可靠性有不确定感,如门窗是否关紧?钱物是否失落?不管患者怀疑什么,事实上他自己都清楚,这种怀疑是没有必要的。

(2)强迫回忆:不由自主地反复回忆以往经历,无法摆脱。

(3)强迫性穷思竭虑:对一些毫无意义或与己无关的事反复思索、刨根究底,如一个会计师苦苦思索了10年:眉毛为什么长在眼睛的上面而不是眼睛的下面?欲罢不能。
\paragraph{强迫情绪}

主要指一种担心,如某患者坐公共汽车时总是双手放在头顶上,担心万一车上有人丢失钱包以免涉嫌自己;一个男孩与女孩说话时要把双手放在背后,用一只手紧紧握住一只手,说是怕自己做出不文明的举动来等。
\paragraph{强迫意向}

患者感到有一种冲动要去做某种违背自己心愿的事。如某工人见到电插座就想去触电、站在阳台上就想往下跳,抱着自己的婴孩就想往地上摔。患者不会真的去做,也知道这种想法是非理性,但这种冲动不止,欲罢不能。
\paragraph{强迫行为}

(1)强迫检查:如反复检查门是否锁紧、煤气是否关好、账目或稿件是否有错,严重时检查数十遍也不放心。

(2)强迫洗涤:如反复洗手、反复洗涤衣物,明知过分,但无法自控。

(3)强迫计数:反复数高楼大厦的窗、数楼梯、数电杆、数路面砖。某患者嗜清点门牌号码,常因某个门牌未点到而串街走巷,直到如愿方才罢休,为此常误了正事,因而痛苦不堪。

(4)强迫性仪式动作:患者经常重复某些动作,久而久之程序化。如某同学进寝室时要在门口站一下,再走进去。某次因同学们相拥而入,他没来得及站立一下,遂焦虑不安,直到后来借故出来,在门口站立一下之后,方才平静下来。

\subsubsection{分离(转换)性障碍}
\paragraph{分离性遗忘}

分离性遗忘并非由器质性因素引起的记忆缺失。患者单单遗忘了某一阶段的经历或某一性质的事件,而那一段经历或那一类事件对患者来说往往是创伤性的,是令患者痛苦的。
\paragraph{分离性漫游}

分离性漫游又称神游症。此症发生在白天觉醒时,患者离开住所或工作单位,外出漫游。在漫游过程中患者能保持基本的自我料理,如饮食、个人卫生等,并能进行简单的社会交往,如购票、乘车等。短暂肤浅的接触看不出患者有明显的失常。此种漫游事先无任何目的和构想,开始和结束都是突然的,一般历时数小时至数天,清醒后对发病经过不能完全回忆。
\paragraph{分离性木僵}

患者的行为符合木僵的标准,但检查和询问找不到躯体原因的证据。此外,如同其他分离性障碍一样,有证据支持心理原因的存在,近期或是有应激性事件,或是有突出的人际或社会问题。

木僵诊断的依据是自发运动以及对声、光、触等外界刺激的反应消失或极度减少,患者在长时间里几乎一动不动地坐着或躺着。完全或几乎没有言语及自发的有目的运动。虽可存在一定程度的意识紊乱,但肌张力、姿势、呼吸、有时睁眼、协调的眼部运动均表明患者既非处于熟睡之中,也不是无意识状态。
\paragraph{出神与附体障碍}

出神表现为暂时性的同时丧失个人身份感和对周围环境的完全意识。在某些病例,患者的举动就像是已被另一种人格、精灵、神、或“力量”所代替。注意和意识仅局限于或集中在密切接触环境的一两个侧面,常有局限且重复的一系列运动、姿势、发音。本处包含的出神状态是指不由自主、非人所愿的,以及发生于宗教或其他文化上认可的外在处境下(或这类处境的延续)的妨碍日常活动者。
\paragraph{分离性运动障碍}

(1)肢体瘫痪:可表现为单瘫、偏瘫或截瘫。伴有肌张力增强者常固定于某种姿势,被动时出现明显抵抗。病程持久者可能出现失用性肌萎缩。

(2)行走不能:坐时、躺时双下肢活动正常,但不能站立行走,站立时无力支撑,则缓缓倒地。

(3)缄默症、失音症:不用语言而用书写或手势与人交流称为缄默症;想说话,但发不出声音,或仅发出嘶哑的、含糊的、细微的声音,称为失音症,检查声带发现功能正常,可正常咳嗽。
\paragraph{分离性抽搐}

分离性抽搐(假性抽搐)在运动方面可与癫痫
的抽搐十分近似,但咬舌、严重摔伤、小便失禁等表现在分离性抽搐中很罕见。不存在意识丧失,而代之以木僵或出神状态。

(1)痉挛发作:受到精神刺激或暗示时发生,缓慢倒地,呼之不理、全身僵直或肢体抖动,或呈角弓反张姿势。患者表情痛苦,眼角含泪,一般持续数十分钟。

(2)局部肌肉的抽动或阵挛:可表现为肢体的粗大颤动或某一群肌肉的抽动,或是声响很大的呃逆,症状可持续数分钟至数十分钟,或中间停顿片刻,不久又可持续。
\paragraph{分离性感觉麻木和感觉丧失}

(1)感觉过敏:对一般的声、光刺激均难以忍受,轻微的抚摸可引起剧烈疼痛。

(2)感觉缺失:表现为局部或全身的感觉缺失,缺失的感觉可为痛觉、触觉、温觉、冷觉或振动觉,缺失的范围与神经分布不一致。

(3)感觉异常:如果感觉咽部有梗阻感或异物感,称为癔症球;头部紧箍感、沉重感,称为病症盔;精神因素引起的头痛或其他躯体部位的疼痛,称为心因性疼痛。

(4)视觉障碍:可表现为失明、管状视野、单眼复视。

(5)听觉障碍:表现为突然失聪,或选择性耳聋,即对某一类声音辨听能力缺失。

\subsubsection{躯体形式障碍}
\paragraph{疑病症}

本病突出的表现是:患者对自身的身体状况过分关注,认为自己患某种严重的躯体疾病。主诉与症状可只限于某一部位、器官或系统,也可涉及全身。症状表现的形式多种多样,有的患者对症状的感知极为具体,描述的病象鲜明、逼真,表现为定位清楚的病感。如肝脏肿胀的感觉、胃肠扭转的体验、脑部充血的感受、咽喉异物堵塞感等。有的患者则体验到定位不清楚的病感,性质模糊,难以言表,只知道自己体虚有病,状态不佳。

疼痛是本病最常见的症状,有一半以上的患者主诉疼痛,常见部位为头部、腰部和胸部,有时感觉全身疼痛。其次是躯体症状,可涉及许多不同器官,表现多样,如感到恶心、吞咽困难、反酸、胀气、心悸;有的患者则觉得有体臭或五官不正、身体畸形。虽查无实据,仍要四处求医、反复检查。
\paragraph{躯体化障碍}

临床表现除了符合躯体形式障碍的诊断概念以外,还必须以多种多样、反复出现、经常变化的躯体症状为主,以下列4组症状之中,至少有2组共6项症状。

(1)胃肠道症状:如腹痛,恶心,腹胀或胀气,嘴里无味或舌苔过厚,呕吐或反胃,大便次数多、稀便,或水样便。

(2)呼吸循环系症状:如气短,胸痛。

(3)泌尿生殖系症状:如排尿困难或尿频,生殖器或其周围不适感,异常或大量的阴道分泌物。

(4)皮肤症状或疼痛症状:如瘢痕、肢体或关节疼痛、麻木或刺痛感。

(5)体检和实验室检查不能发现躯体障碍的证据,能对症状的严重性、变异性、持续性或继发的社会功能损害做出合理解释。

(6)对上述症状的优势观念使患者痛苦,不断求诊,或要求进行各种检查,但检查结果阴性和医师的合理解释均不能打消其疑虑。

(7)如存在自主神经活动亢进的症状,但不占主导地位。

而且体格检查和实验室检查未发现与这些症状相关的躯体疾病的证据。尽管如此,患者仍深感痛苦,不断求医。各种医学检查的正常结果和医师的合理解释均不能打消患者的疑虑,且病程必须持续2年以上。
\paragraph{躯体形式自主神经紊乱}

躯体形式自主神经紊乱是一种主要受自主神经支配的器官系统(如心血管、胃肠道、呼吸系统)发生躯体障碍所致的神经症样综合征。患者在自主神经兴奋症状(如心悸、出汗、脸红、震颤)基础上,又发生了非特异但更有个体特征和主观性的症状,如部位不定的疼痛、烧灼感、沉重感、紧束感、肿胀感,经检查这些症状都不能证明有关器官和系统发生了躯体障碍。故本障碍的特征在于明显的自主神经受累,非特异性的症状附加了主观的主诉,以及坚持将症状归于某一特定的器官或系统。
\paragraph{躯体形式疼痛障碍}

躯体形式疼痛障碍是一种不能用生理过程或躯体障碍予以合理解释的持续、严重的疼痛。情绪冲突或心理社会问题直接导致了疼痛的发生,经过检查未发现相应主诉的躯体病变。患者声称疼痛剧烈,但可能缺少器质性疼痛时所伴有的那些生理反应。躯体形式疼痛障碍的患者主诉最多的是头痛、腰背痛及不典型大的面部疼痛,疼痛的时间、性质、部位常常变化,镇痛剂、镇静剂往往无效,而抗抑郁药物可能获意外之功效。

\subsubsection{神经衰弱}
\paragraph{精神易兴奋、脑力和体力易疲劳}

患者的精神活动极易发动。由于兴奋阈值低,周围一些轻微的,甚至是无关的刺激也能引起患者较强烈的或较持久的反应,因而患者的注意力涣散,不由自主的联想和回忆增多,精力很难集中。引起兴奋反应的刺激并不都很强烈,也不一定都是不愉快的事情,但无法平息的无谓联想却令人痛苦。

由于患者的非指向性思维长期处于活跃、兴奋状态,大脑无法得到必要的、充分的松弛和休息,于是脑力容易疲劳。感到脑子反应迟钝,记忆力减退、思维不清晰、思考效率下降。同时,患者也感到疲乏、困倦、全身无力等躯体疲劳症状,即使适当休息或消遣娱乐之后仍难以恢复。
\paragraph{情绪症状}

患者可能会出现一些焦虑或抑郁症状,但不突出,也不持久。神经衰弱突出的情绪症状是易激惹、易烦恼和易紧张。由于情绪启动阈值降低,再加上情绪自制力减弱,患者显得易激惹,包括稍微受到刺激便易发怒,发怒之后又易后悔;看书看戏易伤感,动不动就热泪盈眶;易感委屈;遇到不平之事、不正之风易愤慨,是可忍孰不可忍。易烦恼,即思绪剪不断、理还乱,觉得人人都不顺眼,事事都不顺心。易紧张是指不必要的担心和不安,总觉得处境不妙,形势紧迫、咄咄逼人。
\paragraph{心理生理症状}

心理生理症状指心理因素引起的某些生理障碍,如紧张性疼痛:患者觉得头重、头胀、头痛、头部紧箍感;或颈部、腰背部的不适和酸痛。睡眠障碍:入睡困难、入睡不深、自觉多梦、睡后不解乏。或者本来睡眠尚可,但总担心失眠或总觉得没有入睡。常常是睡眠节律倒错,该醒时昏昏欲睡,该睡时则头脑清醒。其他的心理生理症状还包括耳鸣、心慌、胸闷、消化不良、尿频、多汗、阳痿或月经不调等。

\subsection{诊断}

ICD-10中关于神经症性障碍的诊断标准如下:神经症是一组主要表现为焦虑、抑郁、恐惧、强迫、疑病症状,或神经衰弱症状的精神障碍。本障碍有一定的人格基础,起病常受心理社会(环境)因素影响。症状没有可证实的器质性病变作基础,与患者的现实处境不相称,但患者对存在的症状感到痛苦和无能为力,自知力完整或基本完整,病程多迁延。各种神经症性症状或其组合可见于感染、中毒、内脏、内分泌或代谢和脑器质性疾病,称为神经症样综合征。

(1)症状标准(至少有下列1项):①恐惧;②强迫症状;③惊恐发作;④焦虑;⑤躯体形式症状;⑥躯体化症状;⑦疑病症状;⑧神经衰弱症状。

(2)严重标准:社会功能受损或无法摆脱的精神痛苦,促使其主动求医。

(3)病程标准:符合症状标准至少已3个月,惊恐障碍另有规定。

(4)排除标准:排除器质性精神障碍、精神活性物质与非成瘾物质所致精神障碍、各种精神病性障碍,如精神分裂症、偏执性精神病以及心境障碍等。

\subsection{治疗}

\subsubsection{恐惧性障碍的治疗}
\paragraph{心理治疗}

行为疗法是目前恐惧症的首选疗法,如暴露冲击疗法、系统脱敏疗法、放松训练等。除了行为治疗外,可合并进行其他心理治疗,如认知疗法、精神分析治疗、人际间心理治疗等。
\paragraph{药物治疗}

严格地说并无一种消除恐惧情绪的药物。研究证据证明下述药物对恐惧症可能有效。

(1)苯二氮䓬 类药物。

(2)β受体阻滞剂,如普萘洛尔。

(3)抗抑郁药:三环类抗抑郁剂(如多塞平、阿米替林、丙米嗪),选择性5-HT再摄取抑制剂(SSRI)、5-HT和去甲肾上腺素再摄取抑制剂(SNRI,如文拉法辛)和其他抗抑郁药(如奈法唑酮、米氮平)。

(4)其他药物:有限的资料提示下述药物可能对恐惧症有效,抗癫痫
药(如托吡酯、加巴喷丁、丙戊酸盐)、抗精神病药物奥氮平等。

\subsubsection{焦虑性障碍的治疗}
\paragraph{心理治疗}

心理治疗如放松疗法,不论是对广泛性焦虑症或急性焦虑发作均是有益的。当个体全身松弛时,生理觉醒水平全面降低,心率、呼吸、脉搏、血压、肌电、皮电等生理指标出现与焦虑状态逆向的变化。许多研究证实,松弛不仅有如此生理效果,而且有相应的心理效果。生物反馈疗法、音乐疗法、瑜伽、静气功的原理与之接近,疗效也相仿。
\paragraph{药物治疗}

1)急性焦虑发作(惊恐障碍)

(1)抗抑郁药。目前已经证实SSRI类药物对惊恐发作有效,且多被作为一线药物。但惊恐发作患者往往对药物引起的激活作用十分敏感,因此在用药初期宜缓慢增加剂量或合并使用苯二氮䓬
类药物。SNRI类及NaSSA类抗抑郁药,目前也被推荐作为惊恐障碍的治疗药物。而TCA类抗抑郁药和MAOIs类药物,如氯米帕明、丙咪嗪、苯乙肼、吗氯贝胺等,对惊恐障碍的疗效较肯定,但由于不良反应大等原因,目前一般不作为一线用药。

(2)苯二氮䓬
类药物。阿普唑仑、氯硝西泮、劳拉西泮已经被证实对惊恐发作有效,但由于具有镇静、肌肉松弛作用及可能的滥用和撤药反应,目前主要作为二线用药。

(3)其他药物。目前尚有临床研究证实下列药物可能对惊恐发作有效,如β肾上腺素阻断剂、丁螺环酮、噻加宾、丙戊酸等。

2)慢性焦虑(广泛性焦虑)

(1)抗焦虑药。苯二氮䓬
类药物:曾经是临床上广泛使用的抗焦虑药物,较常用的有阿普唑仑、艾司唑仑、劳拉西泮、氯硝西泮,目前一般不建议作为一线用药。用药期间,需避免操作快速器械(如高速驾车等),以避免意外。

(2)阿扎哌隆类药物。如丁螺环酮,这类药物具有不良反应轻、不会引起滥用、依赖和撤药反应等优点,但起效慢、疗效不确定。

(3)抗抑郁药。某些抗抑郁药如TCA类的丙米嗪、SSRI类抗抑郁药(如帕罗西汀、舍曲林、西酞普兰)、SNRI类的文拉法辛等药物,经临床研究证实对广泛性焦虑治疗有效。

(4)其他。β肾上腺素阻断剂(如普萘洛尔)。

\subsubsection{强迫症的治疗}
\paragraph{心理治疗}

心理治疗具有重要的意义,使患者对自己的个性特点和所患疾病有正确客观的认识,对周围环境、现实状况有正确客观的判断。丢掉精神包袱以减轻不安全感;学习合理的应激方法,增强自信,以减轻其不确定感;不好高骛远,不精益求精,以减轻其不完美感。同时动员其亲属同事,对患者既不姑息迁就,也不矫枉过正,帮助患者积极从事体育、文娱、社交活动,使其能逐渐从沉迷于穷思竭虑的境地中解脱出来。
\paragraph{药物治疗}

(1)SSRI类抗抑郁药:目前作为治疗强迫症的一线药物,尤其是氟伏沙明、氟西汀、舍曲林和帕罗西汀。其中氟伏沙明、舍曲林被推荐用于儿童。

(2)三环类抗抑郁药:三环类抗抑郁剂已应用于强迫症的治疗。国内报道氯米帕明、丙米嗪及多塞平均有一定的疗效,其中氯米帕明疗效最好,有效剂量为每日150~250mg,服药后第3~4周症状明显改善,显效和有效率达70%左右。但由于三环类抗抑郁药的严重不良反应,目前推荐作为二线用药。

(3)单胺氧化酶抑制剂:本类药物具有一定的抗强迫作用,但由于此类药物作用的特异性差,对食品的限制及药物配伍的禁忌,因而不作为一线用药。

(4)增效治疗:对于上述药物单药治疗无效或疗效欠佳时,可考虑增效治疗和(联合)应用。目前有研究报道的增效剂有氯硝西泮、奥氦平、利培酮、碳酸锂、加巴喷丁等。
\paragraph{电抽搐治疗}

对于那些伴有强烈抑郁情绪甚至有强烈自杀念头的强迫症患者,电抽搐治疗不失为一种有效的方法。
\paragraph{精神外科治疗}

精神外科治疗强迫症,或可减轻症状改善社会功能,但疗效并不肯定,不应作为常规治疗程序,因此必须严格掌握适应证。仅对极少数症状持续而严重的慢性强迫症患者,在充分药物治疗和心理治疗均失败后,而患者又处于极度痛苦中,才可考虑手术治疗。

\subsubsection{分离(转换)性障碍的治疗}
\paragraph{心理治疗}

个别心理治疗一般分若干段进行,首先详细了解患者的个人发展史、个性特点、社会环境状况、家庭关系、重大生活事件,以热情、认真、负责的态度赢得患者的信任。然后安排机会,让患者表达、疏泄内心的痛苦、积怨和愤憋。医师要耐心、严肃地听取,稍加诱导,既不随声附和,也不批评指责。医师要注意患者当前所遭遇的社会心理因素和困境,不能只限于挖掘童年的精神创伤。这种治疗方法几乎适用于全部分离(转换)性障碍的患者。
\paragraph{药物治疗}

有人认为药物治疗的作用有限,似乎都不比暗示治疗更有效。但临床实践中发现,分离(转换)性障碍的患者除了典型的发作以外,常常伴有焦虑、抑郁、脑衰弱、疼痛、失眠等症状。这些症状和身体不适感往往成为分离(转换)性障碍发作的自我暗示的基础。使用相应的药物有效控制这些症状,对治疗和预防分离(转换)性障碍的发作无疑是有益的。

\subsubsection{躯体形式障碍的治疗}
\paragraph{心理治疗}

心理治疗是主要的治疗形式,其目的在于让患者逐步了解所患疾病的性质,改变其错误的观念,解除或减轻精神因素的影响,使患者对自己的身体情况与健康状态有一个相对正确的评估。
\paragraph{药物治疗}

药物治疗主要在于解除患者伴发的焦虑与抑郁情绪,可用苯二氮䓬
类、三环类抗抑郁剂、SSRI以及对症处理的镇痛药、镇静药等。另外,对确实难以治疗的病例可以使用小剂量非典型抗精神病药物,如舒必利、利培酮等,以提高疗效。

\subsubsection{神经衰弱的治疗}
\paragraph{心理治疗}

(1)认知疗法。神经衰弱患者患病前多有一些心理因素,精神刺激虽不算严重,但可能由于患者的过度引申、极端思考或任意推断等形成错误认知,从而导致较明显的内心冲突。矫正患者的认知往往有釜底抽薪的效果。

(2)森田疗法。神经衰弱的患者,部分具有疑病素质,求生欲望强烈。森田疗法建设性地利用这一精神活力,把注意点从自身引向外界,以消除症状、适应环境。

(3)行为疗法。如松弛疗法。神经衰弱患者大多伴有失眠、紧张性疼痛,各种松弛疗法,包括生物反馈、静气功、瑜伽术均有一定效用。
\paragraph{药物治疗}

根据患者的症状,可酌情使用抗焦虑药、抗抑郁药、中枢兴奋药、镇静药、止痛药和促脑代谢药等。如果兴奋症状明显,以抗焦虑药或镇静药为主;如果衰弱症状明显,以中枢兴奋药和促脑代谢药为主;如果白天萎靡不振,夜里却浮想联翩,则可白天用中枢兴奋药,晚上用镇静药,促其恢复正常的生物节律。另可适当选用中成药。
\paragraph{其他治疗方法}

如开展体育锻炼、工娱疗法及各种方法的综合实施,也有一定疗效。