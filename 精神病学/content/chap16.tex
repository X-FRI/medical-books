\chapter{儿童少年期心理发育障碍}

儿童心理发育是指儿童的认知、情感、意志和社会行为的发展过程,当任何有害因素影响了儿童正常的心理发育过程,即出现心理发育障碍,主要表现为语言、运动、学习、社会交往、智力和社会适应能力的低下或异常。按照ICD-10精神障碍分类与诊断标准和中国精神障碍分类与诊断标准(CCMD-3),儿童少年期的心理发育障碍分为精神发育迟滞、特定性发育障碍和广泛性发育障碍三大类。这类疾病大多于学龄前起病,呈持续病程,多起源于认知功能缺陷,以神经发育过程中生物学因素为基础。

\section{精神发育迟滞}

精神发育迟滞(mental
retardation)是一组由生物、心理、社会多种病理性因素引起的智力发育低下和社会适应能力缺陷,起病于18岁前。本症可以单独出现,也可以同时伴有其他精神障碍和躯体疾病。

\subsection{流行病学}

世界卫生组织报道,在任何国家和地区,精神发育迟滞的患病率为1%~3%。我国在1987年进行了全国29个省市的残疾人同步调查,以DSM-Ⅲ作为诊断标准,精神发育迟滞按智商和社会适应行为能力划分等级,结果显示患病率为1.268%,其中男性为1.315%,女性为1.220%,男女发病率之比为109:87。1985---1990年进行全国8省市0~14岁儿童精神发育迟滞的流行病学调查,抽样方法和诊断标准与全国流行病学调查基本相同,结果显示患病率为1.41%,其中城市为0.70%,农村为1.41%;中度及重度占39.4%,轻度占60.6%,轻:重为1.5:1。男孩多于女孩,农村高于城市。

\subsection{病因和发病机制}

精神发育迟滞是由多种病理性因素引起的儿童发育障碍。最近由美国精神发育迟滞协会(AAMR)提供的数字显示,有350多种原因可以引起精神发育迟滞。随着医学和分子生物学的发展,每年均有新的致病因素被发现,目前认为有750多个基因与智力有关,与X染色体有关的精神发育迟滞大约有95种。各种精神发育迟滞的致病因素所占比例与社会经济文化和风俗习惯相关,在发达国家精神发育迟滞病因绝大多数是遗传和围产期因素,在发展中国家产前和产后因素、甲状腺功能低下、感染、外伤、中毒等病因更为突出。一般将其分为出生前、围生期和出生后。病因分类见表\ref{tab16-1}。

\subsubsection{遗传因素}

1.染色体异常 染色体数目异常,可见于常染色体和性染色体,分为单体型、三体型和四体型。染色体结构异常,有倒位、缺失、易位、环型、等臂染色体等。另外还有嵌合体,嵌合体多见于性染色体非整倍体,如45,X/46,XX/47,XXX/48等。

\begin{table}[ht]
    \caption{精神发育迟滞的常见病因}
    \label{tab16-1}
    \centering
    \begin{tabular}{lp{8cm}c}
    \toprule
    病因 & 常见病种 & 比例(%) \\
    \midrule
    遗传性疾病 & & 4~28 \\
\quad 染色体异常 & Down's综合征& \\
\quad 单基因突变 & 结节性硬化、半乳糖血症等代谢性疾病、脆性X染
色体综合征 &\\
\quad 多基因遗传 & 
家族性黑朦性痴呆 & \\
\quad 基因缺失 &
 Williams综合征、Angelman综合征 &\\
先天性畸形 & 
中枢神经系统畸形、神经管缺失多种畸形综合征、Cornelia de Lange综合征 & 
7~17 \\
有害因素 & 
孕期感染风疹、HIV、致畸形物质、胎儿乙醇综合
征、有毒物质,胎儿发育障碍未成熟儿、其他射线、损伤 & 5~13\\
围产期 & 
感染脑膜炎、分娩损伤窒息其他高胆红素血症 & 
2~10\\
出生后 & 
感染:脑炎;毒素:铅中毒;其他损伤:脑肿瘤;社会
心理因素:贫穷、精神疾病 & 3~12\\
未知因素 & &
30~50 \\
    \bottomrule
    \end{tabular}
\end{table}


性染色体畸变,如46,XXY,则为先天睾丸发育不全,46,Y为先天卵巢发育不全,染色体X畸变数越多,智力低下发生率越高,程度越重。常染色体畸变,包括染色体非平衡易位、三体型,即G组第21对染色体三体型所致,另外,还有嵌合体。Down综合征(21三体综合征)属于染色体畸变。脆性X染色体综合征也是常见的家族性X连锁的精神发育迟滞,脆性部位在Xq27或28带上,多见于男性,女性多为无临床症状的基因携带者。

2.基因突变 由于单基因或多基因发生突变,引起遗传性状的改变,造成与代谢有关的酶的活性不足或缺乏,导致遗传代谢性疾病。如苯丙酮尿症,是由于苯丙氨酸羟化酶的缺乏,不能将苯丙氨酸氧化成酪氨酸,以致大量苯丙氨酸蓄积在血液和脑脊液中,而酪氨酸缺乏,与酪氨酸代谢及合成有关的神经递质、黑色素缺乏,出现智力低下、苯丙酮尿等症状。半乳糖血症是由于1-磷酸半乳糖转变成1-磷酸葡萄糖的过程受阻,大量的半乳糖聚集在血液和组织中,对脑、肝、肾等器官的细胞产生损害,导致智力低下和多器官损伤。另外,同型胱氨酸尿症、家族性黑矇性痴呆、脂质沉积症、黏多糖病、脑白质营养不良等均为遗传代谢疾病导致的智力低下。

\subsubsection{母孕期损伤}

母孕期各种有害因素均可导致胎儿脑发育异常,出现智力低下,尤其是在怀孕的前三个月。

1.感染 病毒、细菌、螺旋体、寄生虫等的感染,以病毒感染最常见。目前认为风疹、单纯疱疹、巨细胞病毒、肝炎病毒、弓形虫5种病原体可直接感染胎儿,抑制细胞的增殖分化,影响DNA的复制,导致基因突变和染色体畸形,阻碍胚胎发育和器官形成,引起多器官畸形,大脑是最常累及的器官。常见有小头畸形、脑积水、脑发育迟缓、耳聋、白内障等。

2.药物和毒性物质 母孕期服用某些药物可影响胎儿发育,导致畸形,其中一部分出现精神发育迟滞。目前认为能引起精神发育迟滞的药物包括作用于中枢神经系统、内分泌和代谢系统药物,抗肿瘤药物和水杨酸类药物,如地西泮、苯妥英钠、甲氨蝶呤、碘化物等。有机汞、铅、X线、电磁波等有毒物质,亦可影响脑功能,造成智力低下。过度吸烟和酗酒可引起胎儿发育差、小头畸形和智力低下。

3.妊娠期疾病和营养不良 母孕期患有高血压、心脏病、重度贫血、肾脏病、糖尿病、癫痫
等均可引起胎儿缺氧、中毒、代谢障碍,从而影响胎儿大脑发育。

4.心理因素 急性精神创伤,长期情绪压抑、紧张、焦虑、忧郁,可引起孕妇代谢和免疫功能异常,影响体内激素水平,导致胎儿发育不良和神经系统发育缺陷。

\subsubsection{围生期损伤}

产前出血、前置胎盘、胎盘早期剥离、宫内窘迫和窒息引起的脑缺氧、产伤引起的颅内损伤和出血、因血液中胆红素浓度过高引起的胆红素脑病等,均可造成神经细胞损伤,导致智力低下。

\subsubsection{出生后有害因素}

1.感染和外伤 中枢神经系统感染,如各种脑炎、脑膜炎、中毒性脑病、继发于躯体疾病的脑部感染,均可造成脑细胞永久性损害,造成智力低下。较严重的颅脑外伤,并伴有意识障碍,会引起神经系统损伤,继发智力障碍。

2.脑缺氧 尤其是3岁以内的幼儿,神经系统处于快速生长发育时期,对缺氧的耐受性很低,各种原因引起的脑缺氧,如癫痫
、蒙被综合征等,可引起脑细胞坏死,造成智力低下。

3.内分泌和代谢功能异常 如地方性甲状腺功能低下(克汀病)、促性腺激素功能低下等。

\subsubsection{社会因素}

一些研究认为社会因素可以造成智商在20分范围内的变异,流行病学研究显示社会阶层低、贫穷、居住条件差和家庭环境不稳定与低智商有关。并且依据社会预测因素不同得到的智商也有明显不同。

\subsection{临床表现}

\subsubsection{起病形式}

精神发育迟滞不是一种特异性疾病,而是一种可以有各种各样的原因导致的状态,因此,没有具体的发病时间。然而,诊断标准定义为儿童期起病,明确为18岁之前。如果发病是在18岁之后,则归类为痴呆症。各种诊断分类标准均将起病年龄定于18岁之前,意味着认知能力可以持续发展,直到18岁。

\subsubsection{临床表现}

精神发育迟滞有认知功能缺陷,学习能力受损,语言处理和逻辑思维能力缺陷。由于推理和解决问题能力缺陷,影响他们语言处理、判断和分析能力。精神发育迟滞还影响情绪和情感活动,他们的情感表达幼稚,不成熟。

适应行为随年龄不断发展,与精神/智力和社会/情感技能有关的功能不断发展并完善。根据定义,精神发育迟滞适应功能有明显缺陷,主要表现在以下领域:沟通,自我照顾,家庭生活,社交/人际技巧,社区资源的使用,自我发展,学业技能,工作,休闲活动,健康和安全。这些缺陷大大阻碍个人处理日常生活和适应社会的能力。

精神发育迟滞也有一些共同的行为特点,如低容忍挫折。容易受挫的精神发育迟滞有时出现攻击和自伤行为。其他行为特点如冲动、固执、不成熟、注意力集中困难等。

有些精神发育迟滞具有独特的外貌特征,包括身材矮小,与精神发育迟滞相关的面部特征。也有完全正常的外貌。

精神发育迟滞的临床表现与智力损伤程度密切相关,主要表现为由智力低下和社会适应能力缺陷引起的一系列症状。

1.轻度(IQ50~69) 占精神发育迟滞的85%,在学龄前很难与正常儿童区别。早期心理发育较正常儿童迟缓,语言发育延迟,词汇较少,但无明显语言障碍,对语言的理解和使用能力较差,分析、综合、抽象思维能力差,社会适应能力低下,应变能力较差。在普通学校学习困难,通常在入学后才被诊断,经过努力,勉强达到小学毕业的水平,能够学会简单的谋生技能,结婚并养家,需要提供处理生活和工作问题的帮助。

2.中度(IQ
35~49) 占10%,多数有神经系统和躯体方面的异常发现,伴有明显的认知和适应功能受损。早期心理发育较正常儿童明显迟缓,讲话吐词不清,词汇贫乏,语言表达能力较差。不能建立抽象思维,缺乏分析综合能力。经过特殊教育,可以达到小学三年级水平。社会适应能力较差,终身需要给予支持性帮助,在监护下可进行简单劳动。大多数的Down's综合征和X脆性染色体综合征属于此类。

3.重度(IQ20~34) 占3%~4%,大多伴有1种或多种躯体畸形和神经系统功能障碍。出生后不久即可被发现。患儿发育明显落后,言语发育障碍,语言理解和表达能力极差,不能进行有效的语言交流,缺乏抽象思维能力,情感幼稚,动作协调性差,社会交往能力发育较差。部分患儿经过训练后,能够学会生活自理技能和简单劳动,需要终身给予监护和照顾。

4.极重度(IQ低于20) 占1%~2%,有明显的躯体畸形和神经系统功能障碍,认知、运动、社会交往功能均有明显损害,没有语言功能,感知觉减退,情感反应原始,生活完全不能自理,社会功能全部丧失,经过特殊训练后,部分患儿能够学会简单的生活技能。

\subsection{共病}

精神发育迟滞患者精神异常的患病率较正常人群高,虽然其精神异常的表现与正常人群有共通性,但其症状表现常因其智力水平的不同而有所不同。幻觉、妄想、强迫观念在重度精神发育迟滞以及言语发育受限的病人中不易被识别。因此对精神发育迟滞共病精神障碍患者的诊断着重于行为表现,这是与智力正常人群的不同之处。

1.精神分裂症 正常人群精神分裂症时点患病率为1%,精神发育迟滞为3%。患病率的增高与精神分裂症的神经发育假说相吻合。另外一些导致精神发育迟滞的病因也可以累及与精神分裂症的症状相关的脑区。精神发育迟滞患精神分裂症的主要特征是思维贫乏,幻觉内容单一重复,与以往不一致的古怪行为。治疗原则与精神分裂症相似。

2.注意缺陷-多动障碍 精神发育迟滞患儿中注意缺陷-多动障碍的发生率为9%~18%。患儿表现出与其心理年龄不适应的注意力集中时间短暂,过度的心理运动性活动,冲动。对重度精神发育迟滞,其诊断标准应该相应提高。

3.冲动控制障碍 精神发育迟滞患儿中冲动控制障碍的发生率为36%,增加其认知功能缺陷的严重性,是需要治疗的较为严重的问题。其主要表现为自伤和攻击性行为,典型症状表现为慢性、重复、频繁、刻板的,引起自己或他人伤害的行为,这些行为无明显原因和目的,持续存在,严重影响其社会交往功能。部分发生于遗传性疾病,如Lesch-Nyhan综合征、Smith-Magenis综合征。

4.焦虑障碍 精神发育迟滞患儿中焦虑障碍的患病率为1%~25%。各研究之间差异性较大的原因是由于患儿对焦虑症状的主观性体验的表达能力较差,使部分患儿漏诊。临床上主要表现为攻击性行为、易激惹、强迫或重复性行为、自伤、失眠。惊恐发作时可以表现为激惹,尖叫,哭泣,依附亲人,可以出现妄想或偏执性行为。恐惧性焦虑障碍和应激后创伤性障碍在这类患儿中也较多见。

\subsection{病程和预后}

精神发育迟滞起病于18岁前,其病程和预后与病因和脑组织损伤程度密切相关。由于精神发育迟滞病因复杂多样,如果造成脑结构和功能的损害,其结果是不可逆的,将导致终身残疾。少数病因明确者若得到及时治疗,脑组织未受到损害,则预后较好。

\subsection{诊断}

对精神发育迟滞的评估主要包括四个方面:①精神发育迟滞的病因学;②相关的躯体情况;③智力和社会技能发展;④共病的精神障碍及其原因和后果。完整的评估分为以下几个阶段:病史采集、体格检查、发育测试、功能行为评定。

1.病史和体格检查 详细了解病史、家族史、遗传史、孕产史、生长发育史,了解家族中是否有智力低下、精神神经疾病、退行性疾病和学习困难患者,母亲在孕产期是否有高危因素,儿童早期生长发育是否正常,家庭对儿童的照管和教育环境如何。

全面的体格检查,包括躯体发育情况,如头围、面部特征、身高、体重,有无先天性畸形,视、听觉有无障碍,神经系统是否有阳性体征,行为特征。

2.实验室检查 包括生化、代谢功能检查,如苯丙氨酸、甲状腺功能测定等;遗传学检查,包括染色体核型分析、DNA分子生物学检查;电生理检查,如脑电图、脑诱发电位等;影像学检查,如头颅CT、磁共振等。

3.发育评估

(1)智力测验:常用的智力测验是韦氏学前和儿童智力量表(WPPSI,WISCR),用于4~16岁儿童的智力测验。对4岁前的儿童常用发育评估,包括盖泽尔发展测验、丹佛儿童发展筛选测验等。

(2)适应行为评定量表:常用的有婴儿初中学生社会生活能力量表,是我国1988年从日本引进并修订,适用年龄为6个月至15岁,用于评定儿童社会生活能力;儿童适应行为评定量表,该量表是1990年在我国长沙编制,适用对象为3~12岁儿童,用于评定儿童适应行为发展水平,协助诊断精神发育迟滞儿童,并可帮助制订智力低下儿童的特殊训练计划。

4.诊断 根据发病年龄低于18岁,韦氏智力测验智商低于70分,社会适应能力低于正常,即可诊断为精神发育迟滞。根据智力测验结果,确定精神发育迟滞的严重程度。智商为70~85分,为正常与异常之间的边缘状态。根据检查结果,如能做出病因学诊断,则原发疾病的诊断应与精神发育迟滞并列。

5.诊断标准 临床将智商(IQ)作为分级标准,一般分为轻、中、重和极重度。

轻度:

(1)智商为50~69,心理年龄为9~12岁。

(2)学习成绩差。

(3)能自理生活。

(4)无明显言语障碍,但对语言的理解和使用能力有不同程度的延迟。

中度:

(1)智商为35~49,心理年龄为6~9岁。

(2)不能适应普通学习,可进行个位数的加、减法计算。

(3)可学会自理简单生活,但需督促、帮助。

(4)可掌握简单生活用语,但词汇贫乏。

重度:

(1)智商为20~34,心理年龄3~6岁。

(2)表现显著的运动损害或其他相关的缺陷,不能参加学习和劳动。

(3)生活不能自理。

(4)言语功能严重受损,不能进行有效的语言交流。

极重度:

(1)智商低于20,心理年龄约在3岁以下。

(2)社会功能完全丧失,不会逃避危险。

(3)生活完全不能自理,大小便失禁。

(4)言语功能丧失。

\subsection{鉴别诊断}

1.儿童孤独症 起病于婴幼儿期,约50%伴有智力低下,主要表现为社会交往障碍、交流障碍和刻板重复的行为,多无痴呆外貌,也无躯体畸形。而精神发育迟滞儿童的社会交往和交流能力与其心理年龄适应,多无刻板重复行为。

2.特定性发育障碍 该病起病较早,可出现学习困难,自卑,社会适应能力下降。但详细的体格检查和全面的心理评估,发现该类患儿智力和社会适应能力在正常范围。除了特定性发育障碍,在其他方面均表现正常,在不涉及有关的特定性技能的时候,可以完成学习任务。精神发育迟滞的学习能力与其智力水平相适应。

\subsection{预防与治疗}

精神发育迟滞是导致人类残疾最严重的疾病之一,工作的重点在预防,研究的重点是病因和治疗。我国1984年精神发育迟滞病因流行病学研究发现,出生时和出生后的因素比发达国家多得多,而遗传因素相对较少。对现已明确病因的精神发育迟滞,如碘缺乏病、苯丙酮尿症、唐氏综合征、胎儿乙醇中毒等,开展一级预防。出生时和出生后致病因素多为感染、中毒、外伤等。预防的重点在发展经济,改善医疗条件。目前广泛开展了遗传咨询、新生儿筛查、产前TORCH病毒学筛查和遗传检查,均有助于降低发病率。对精神发育迟滞的治疗,正在进行基因治疗和脑组织移植等方面的研究。全国许多地区建立了精神发育迟滞的特殊教育机构,开展对精神发育迟滞的康复治疗。

\subsubsection{预防措施}

1.遗传筛查与遗传咨询 根据家族史、精神发育迟滞的遗传学研究以及目前进行遗传筛查的可能性,对异常儿童出生风险进行评估。

2.产前监护 随着遗传学的研究深入,对很多遗传疾病可以进行产前检查和诊断。

3.产后预防 对新生儿进行苯丙酮尿症以及甲状腺功能常规检查,可以早期发现和治疗精神发育迟滞。

\subsubsection{治疗与康复}

治疗的原则是早期发现,早期诊断,早期干预,应用医学、社会、教育和职业训练等综合措施,使患儿的社会生活能力得到发展。

1.病因治疗 对能够发现病因者,应尽早进行病因治疗。如苯丙酮尿症,最好在出生后3周内开始低苯丙氨酸饮食治疗,有的可使用低苯丙氨酸水解蛋白治疗。随着年龄增长,可逐渐增加含苯丙氨酸较少的食品如米、大豆、白菜及糖等。半乳糖血症应及早停止使用乳类食品。先天性甲状腺功能低下者,早期使用甲状腺素均可以使症状完全消失,不发生智力低下。对脑积水、神经管闭合不全等颅脑畸形可做外科手术治疗。目前已有报道开展基因治疗。

2.药物治疗

(1)抗抑郁药:对伴有焦虑、强迫、冲动、自伤行为的精神发育迟滞可以使用抗抑郁药。三环类药物因其可诱发癫痫并有心脏方面的不良反应,已经很少使用。可根据病情需要选择5-HT再摄取抑制药,用量应该低于普通人群使用的剂量。

(2)抗焦虑药:苯二氮䓬
类药物在精神发育迟滞患者中可以引起意识蒙眬,认知功能损害,站立不稳和异常兴奋,因此较少使用。可以根据病情需要选择5-HT再摄取抑制药。

(3)抗精神病药:精神发育迟滞患者使用经典抗精神病药引起迟发性运动障碍和锥体外系反应的比例远远高于一般人群,为18%~30%。而非经典抗精神病药物不良反应较少,治疗效果也较好。

3.康复治疗 对精神发育迟滞儿童的训练需要教育、心理、医学、社会各方面共同开展,根据患儿的程度,制定不同训练目标,其目的是使患儿能够生活自理、自立和参与社会,提高其社会适应能力。社会适应能力主要包括独立生活能力、运动功能、职业能力、交流能力、社会交往能力、自我管理能力、社区设施使用能力、闲暇时间安排能力等。训练的原则是:①早期发现,早期干预;②矫治缺陷;③因材施教,充分发展潜力;④以目标训练为主,灵活使用教学方法。

主要分为三个阶段:

(1)早期干预:早期干预从出生后至3岁的服务,要对患儿进行体格、运动功能、认知功能、语言、社会行为、情绪的发育评估。对家长和家庭提供情感上的支持和相关的知识,制订以家庭为基础的早期干预计划。

(2)学龄前和学龄期:由教育部门提供特殊教育,根据评估结果,制订个别训练计划。

(3)成人期:提供居家和社区服务,有条件者可以进入庇护工厂。


\section{广泛性发育障碍}

广泛性发育障碍(pervasive developmental disorders,
PDDs)起病于儿童早期,是一种严重神经发育障碍,以社会交往和沟通模式质的损害,局限、刻板、重复的兴趣和行为为临床特征。这组疾病包括儿童孤独症(childhood
autism)、Asperger's综合征(Asperger's disorder)、Rett's综合征(Rett's
disorder)、儿童期瓦解障碍(childhood disintegrative
disorder)又称Heller综合征、非特定的广泛性发育障碍(pervasive
developmental disorder not otherwise
specified)(包括非典型孤独症)。本书主要介绍儿童孤独症。

儿童孤独症是广泛性发育障碍的一个亚型,也就是1943年Leo
Kanner最早报告时所称的“婴儿孤独症”,后来也有“孤独症”、“孤独障碍”以及“自闭症”等名称。它起病于3岁前,多见于男孩,主要表现为明显和持久的社会交往能力的损害,交流功能异常和狭窄,刻板的兴趣行为。

\subsection{流行病学}

根据Fombonne
2003年综述的流行病学资料,儿童孤独症的患病率有显著增加的趋势,如1966---1991年孤独症的患病率为4.3/万,而1992---2001年则达到12.7/万。全部PDDs的患病率可能高达40~60/万。患病率上升的原因可能有很多,如诊断标准的变迁、公众对疾病的认识增加、早期诊断、环境因素等。我国尚缺乏全国性的流调数据,仅有部分地区性流行病学调查结果,如2001年天津市5000名0~6岁儿童抽样调查结果显示:儿童孤独症的患病率为0.1%。该病男女发病率差异显著,男女发病率之比为(2.6:1)~(5.7:1)。

\subsection{病因和发病机制}

儿童孤独症是一种与遗传密切相关的异常行为综合征。

\subsubsection{遗传因素}

1991年Folstein和Piven报道儿童孤独症的单卵双生子同病率为82%,双卵双生子同病率为10%。儿童孤独症的同胞患病率为4.5%,为普通人群50~100倍。Bolton等(1994年)发现儿童孤独症家系中儿童孤独症患病的负荷度增加,儿童孤独症同胞及双亲存在类似的认知功能缺陷和特定的人格特征,这些都表明儿童孤独症的发病存在遗传学基础。进一步研究发现儿童孤独症不符合单基因遗传的特征,多基因遗传的可能性较大。

\subsubsection{神经生物学因素}

1.神经生化研究 研究发现儿童孤独症病儿中存在多种神经递质的异常,但只有5-羟色胺水平增高是较为一致的结果,在儿童孤独症患者中约有38%存在全血或血小板中的5-羟色胺水平增高。

2.神经电生理研究 与正常人群相比,儿童孤独症病儿脑电图异常的比例很高,但无特异性。

3.神经病理学研究 1998年Bailey等研究6例儿童孤独症的脑组织发现小脑部位有神经细胞迁移异常,浦肯野氏细胞数量减少。

4.神经影像学的研究 采用正电子发射体层摄影(PET)、功能性核磁共振成像(fMRI)以及单光子发射电子计算机体层扫描(SPECT),发现儿童孤独症的边缘系统、脑干和小脑以及相关皮层存在结构和代谢方面的异常。1995年日本研究了102例孤独症病儿,发现小脑蚓部第6、7叶比正常儿童小,脑干、海马、胼胝体等也有异常。MRI研究认为儿童孤独症患者出生时脑体积正常,但在其出生后前1~2年内的脑生长速度过快,提示脑发育轨迹的异常。

利用弥散张量成像(DTI)及fMRI研究发现,儿童孤独症患者脑的短程连接增加而长程连接减少,存在中枢连通性的异常。

\subsubsection{环境因素}

已有研究证实儿童孤独症的发病与养育方式无关,但母孕期的感染、婴幼儿期的感染、疫苗及抗菌素的使用等环境因素有可能影响了某些基因的表达,从而在儿童孤独症的发病过程中起到一定的作用。从另一个角度来说,早期的干预也在一定程度上可以影响患儿大脑的发育,改善预后。

\subsection{临床表现}

1.起病年龄 儿童孤独症起病于3岁前,绝大多数一岁以内已有孤独样症状,容易引起家长注意的问题是不会讲话或不理人,但从引起家长怀疑到带孩子去就诊再到确诊之间的时间间隔往往比较长。有10%~25%的孩子在起病之前有发展正常的阶段。

2.社会交往障碍 绝大部分儿童孤独症病儿在婴幼儿期不会区分亲人和陌生人,很少出现“陌生人焦虑”,表现为谁抱都无所谓,部分病儿较大后会有分离焦虑,表现出对熟悉人的依恋。婴幼儿时期,还会表现出不理人,听而不闻、视而不见,回避与人的视线接触。18个月时还不能指点东西、用视线来表达信息以及缺乏假扮游戏能力是儿童孤独症的早期指征。三、四岁时表现不合群,对小朋友没兴趣。在儿童时期受到挫折或受伤时,大都不会主动要求父母安慰,当然也很难主动去安慰关心别人。缺乏理解他人的能力,不会交朋友,难以建立友谊。

3.言语交流障碍 表现有:

(1)非语言交流功能障碍:病儿很少使用体态语言和面部表情进行交流,如不使用点头、摇头、摆手等表达自己的意愿,面部表情也较同龄儿童呆板,显示出非言语沟通技能的障碍。

(2)语言发育延迟或不发育:大部分病儿语言发育迟缓,甚至不发育。少数病儿在2~3岁前有语言表达,起病后语言逐渐减少、消失。

(3)语言内容及形式的异常:即使有语言功能的病儿,在语言的应用上也存在很多明显的问题,不会恰当地运用语言进行交流。常有刻板重复性语言或模仿性语言,又称为鹦鹉学舌,也有自言自语、哼哼唧唧,自得其乐。除有什么需要外,一般不会主动与人交谈,难以提出或维持话题,往往是自顾自地讲话。还有的表现为语音、语调和语速的异常,缺乏抑扬顿挫的感情色彩。

4.兴趣狭窄、刻板动作及强迫重复性行为

(1)兴趣狭窄或异常的兴趣:对一般儿童喜欢的玩具和游戏不感兴趣,而专注于一些重复性较多的事物,如圆的可以旋转的物体等。依恋某些无生命的物体,如小棍子、木块等,整天拿在手上,如果强行拿开则烦躁不安。有些高功能的病儿会对数字、认字、天文、地理或绘画等表现出特殊的兴趣和才能。

(2)刻板重复动作:常反复扑翼样振动手臂,旋转,蹦跳;将手置于眼前,长时间凝视。扑打、撞击自己的头部和身体,兴奋和烦躁时更频繁。

(3)固定的仪式行为:拒绝改变自己的生活习惯和环境,如走固定的路线,东西摆放在固定的位置,不愿意吃新的食物换新的衣服等,不得以改变时往往焦虑不安。

5.感知异常 孤独症儿童表现为各种感知觉过弱、过强或不寻常。即:对痛觉的感受迟钝;触觉的敏感和异常,不愿意用手或脚接触到沙子、泥土或水,反复触摸光滑的物体;听觉上对很强烈的声音感觉迟钝,但对某些特定的声音却很敏感;视觉上喜欢看光亮的或旋转的物体;味觉上经常用舌去舔某些物品,偏食明显;有的喜欢用鼻子来探索周围的世界,不论给他们什么东西都要先闻一闻;有的病儿平衡能力特强,怎么转也不晕。

6.智能障碍 约有75%的病儿有智力低下,其中45%为重度至极重度,30%为轻度至中度。此外,病儿还表现出智能发育的不平衡性,操作智商优于语言智商,也有少数病儿在个别能力上表现超常,称为“岛状智力”。

7.其他伴发的行为和情绪障碍 多数病儿伴有行为和情绪问题,如多动、注意力涣散,冲动、攻击性、破坏性行为及自伤行为等,青春期的病儿易出现焦虑、抑郁、强迫、偏执等症状;还可以伴有进食和睡眠障碍,如偏食、挑食,入睡困难;少部分还伴有抽动症状。

\subsection{病程与预后}

儿童孤独症的症状和能力会随着年龄的变化而变化,一般在4~6岁时孤独性症状最为典型,之后会有不同程度的改善,如对父母产生依恋,刻板动作减少,语言、认知等能力也有一定的发展,极少数出现行为衰退。至成年期,约有25%的预后良好,虽然在社会交往和人际关系方面存在困难,但能接近正常生活。约50%的患者预后较差,生活不能自理,需要终身监护。

总体认为,儿童孤独症是导致终身残疾的疾病,长期预后较差。预后与疾病的严重程度、病前语言功能、智商高低以及是否得到及时有效的治疗有关。

\subsection{诊断}

儿童孤独症的诊断主要通过详细地病史询问、精神检查、体格检查和必要的辅助检查,然后依据诊断标准作出。

1.病史收集 首先要详细了解病儿的生长发育过程,包括运动发育、语言发育、认知发育以及社会情绪能力的发育等。然后,针对发育落后的领域和让家长感到异常的行为进行询问,注意异常行为出现的年龄、持续时间、频率及对日常生活的影响程度。同时,也要收集孕产史、家族史、既往疾病史以及家庭对儿童的养育过程。

2.精神检查 主要采用观察法,有语言能力的孩子可以结合交谈。注意观察儿童对陌生环境的反应、对父母离开的反应、对玩具的兴趣、玩玩具的方式、有无同理心、言语及非言语沟通的能力以及有无多动、刻板怪异动作等。

3.体格检查 包括躯体发育情况,如头围、面部特征、身高、体重,有无先天性畸形,视、听觉有无障碍,神经系统是否有阳性体征等。

4.心理评估 常用的筛查量表有孤独症行为量表(autism behavior check list,
ABC)、儿童孤独症评定量表(childhood autism rating scale,
CARS)、克氏行为量表(Clancy behavior scale,
CBS)和婴幼儿孤独症筛查量表(checklist for autism in toddlers,
CHAT)。常用的诊断量表有孤独症诊断观察量表(autism diagnostic
observation schedule generic, ADOS-G)和孤独症诊断访谈量表修订版(autism
diagnostic interview-revised, ADI-R)。

5.实验室检查 可根据情况进行电生理检查如脑电图、脑干诱发电位,影像学检查如头颅核磁共振等,目前遗传学检查如脆性X染色体检查等也认为很有必要进行。

\subsection{诊断标准}

根据《中国精神障碍分类与诊断标准第三版》(CCMD-3)中有关儿童孤独症的诊断标准如下:

\subsubsection{症状标准}

在下列(1)、(2)和(3)项中,至少有7条,且(1)项至少有2条,(2)、(3)项至少各有1条:

(1)人际交往存在质的损害,至少2条:

①对集体游戏缺乏兴趣,孤独,不能对集体的欢乐产生共鸣;

②缺乏与他人进行交往的技巧,不能以适合其智龄的方式与同龄人建立伙伴关系,如仅以拉人、推人、搂抱作为与同伴的交往方式;

③自娱自乐,与周围环境缺少交往,缺乏相应的观察和应有的情感反应(包括对父母的存在与否亦无相应反应);

④不会恰当地运用眼对眼的注视以及用面部表情、手势、姿势与他人交流;

⑤不会做扮演性游戏和模仿社会的游戏(如:不会玩过家家等);

⑥当身体不适或不愉快时,不会寻求同情和安慰;对别人的身体不适或不愉快也不会表示关心和安慰。

(2)言语交流存在质的损害,主要为语言运用功能的损害:

①口语发育延迟或不会使用语言表达,也不会用手势、模仿等与他人沟通;

②语言理解能力明显受损,常听不懂指令,不会表达自己的需要和痛苦,很少提问,对别人的话也缺乏反应;

③学习语言有困难,但常有无意义的模仿言语或反响式言语,应用代词混乱;

④经常重复使用与环境无关的言词,或不时发出怪声;

⑤有言语能力的患儿,不能主动与人交谈、维持交谈,应对简单;

⑥言语的声调、重音、速度、节奏等方面异常,如说话缺乏抑扬顿挫,言语刻板。

(3)兴趣狭窄和活动刻板、重复,坚持环境和生活方式不变:

①兴趣局限,常专注于某种或多种模式,如旋转的电扇、固定的乐曲、广告词、天气预报等;

②活动过度,来回踱步、奔跑、转圈等;

③拒绝改变刻板重复的动作或姿势,否则会出现明显的烦躁和不安;

④过分依恋某些气味、物品或玩具的一部分,如特殊的气味、一张纸片、光滑的衣料、汽车玩具的轮子等,并从中得到极大的满足;

⑤强迫性地固着于特殊而无用的常规或仪式性动作或活动。

\subsubsection{严重标准}

社会交往功能受损。

\subsubsection{病程标准}

通常起病于3岁以内。

\subsubsection{排除标准}

排除Asperger's综合征、Heller's综合征、Rett's综合征、特定性感受性语言障碍、儿童分裂症。

\subsection{鉴别诊断}

儿童孤独症需要与其他的广泛性发育障碍以及其他儿童常见精神神经疾病进行鉴别。

1.Asperger's综合征 Asperger's综合征病儿语言的发育和智能基本正常。Asperger's综合征儿童也像儿童孤独症儿童一样存在人际关系方面的困难,但他们是有交往的兴趣但缺乏相应的社交技能;Asperger's综合征的病儿也有兴趣狭窄和刻板的动作,常表现于对数字或日子的记忆,以及对某些学科知识的强烈兴趣;在语言及沟通方面,Asperger's综合征的病儿在说话时往往表现出过度书面化、较差的节奏和音调,在讲话的内容方面则显得有自我中心倾向,常常围绕自己感兴趣的话题是“对人讲话”而不是“与人交谈”;在运动技能方面,Asperger's综合征的病儿往往显得比较笨拙。

2.非典型孤独症(atypical
autism) 发病年龄超过3岁,或无法同时符合上述诊断标准中的三组症状,只符合其中的两条时诊断为非典型孤独症。非典型孤独症可见于极重度智能低下的病儿,也可见于有些高功能的儿童孤独症患者,到学龄期部分症状改善或消失,无法完全符合儿童孤独症诊断条件的。

3.Rett's综合征 Rett's综合征仅见于女孩,患儿在早期发育正常,在6~24个月起病,表现出语言功能和手部运动功能的丧失。有特征性的手部刻板“洗手”动作、智力显著倒退、过度通气、注视或凝视他人呈现“社交性微笑”、共济失调等临床表现。多数病例伴有癫痫
发作。

4.儿童期瓦解障碍 又称Heller's综合征、婴儿痴呆,病儿至少在2岁以前发育完全正常,常在3~4岁发病,之后出现原有全部功能的迅速丧失,同时合并和儿童孤独症相同的社会功能和沟通功能质的异常。

5.精神发育迟滞 部分精神发育迟滞儿童可以表现孤独样症状,多数儿童孤独症儿童亦表现精神发育迟滞。鉴别的关键在于精神发育迟滞的病儿有社会交往的兴趣,语言和非言语上主要是发育的迟滞,而无运用上的质的损害。

6.其他 需要与儿童孤独症鉴别的疾病还有注意缺陷-多动障碍、严重学习障碍、选择性缄默症、感受性语言发育障碍、儿童精神分裂症等。

\subsection{治疗}

目前针对儿童孤独症尚无有效的治疗药物,提倡综合的干预方法,即根据患儿的症状,有选择地使用各种心理、教育干预技术,辅以药物治疗。治疗的目的主要是提升缺陷行为,矫正问题行为,减少冲动行为,提高学习能力,尤其是语言能力、交流能力和生活自理能力。

1.药物治疗 目前尚无药物可以治疗儿童孤独症的核心症状。用药的目的在于改善特定的行为和情绪障碍,为教育和心理干预提供条件。

(1)利培酮:可改善多动、攻击、刻板行为、兴奋及情绪不稳定。常用剂量为每日0.5~2mg。该药的不良反应较传统抗精神病药物为轻,常见不良反应有镇静、体重增加及便秘等。

(2)氟西汀:对强迫行为和焦虑情绪有改善作用,常用剂量为每日2.5~10mg。用药1~3周后起作用。常见不良反应为恶心、头痛、失眠和皮疹等。

2.教育干预和行为矫正 方法主要有在学习原理和行为主义的原则上发展出的应用行为分析(applied
behavior analysis, ABA)和结构化教学(treatment and education of
autistic and communication-related handicapped children,
TEACCH)。ABA中比较成熟的技术有:回合式教学法、关键反应教学法、随机教学法等。同时针对家庭的心理支持也十分重要。


\section{特定性发育障碍}

特定性发育障碍起病于婴幼儿或儿童期,病程持续,临床表现以语言、运动和学习能力发育延迟或异常为主要特征。这类疾病主要分为言语和语言发育障碍、学校技能发育障碍和运动技能发育障碍。

\subsection{学校技能发育障碍}

特定学校技能发育障碍是指起病于学龄前,但多在入学后才被发现,主要表现为学校技能的获得与发展障碍,如阅读、拼写和计算等与学习有关的技能发育迟缓或异常,其心理行为发展有明显不平衡,学习成绩与智力水平或其他能力之间存在明显差距,标准化的学习技能测验评分明显低于相应年龄同年级儿童的正常水平或低于相应智力的期望水平,至少达两个标准差以上,严重影响患儿的学习成绩或日常生活中需要这种技能的活动。这类障碍不是由于缺乏教育机会、智力发育迟缓、中枢神经系统疾病、视觉和听觉障碍,也不是行为和情绪障碍引起。其病因与生物因素有关,以神经发育过程的生物学因素为基础,导致与学习有关的认知功能缺陷。目前临床上主要分为阅读障碍、拼写障碍和计算技能障碍,可以单独存在,也可能混合出现。

\subsubsection{流行病学}

国外的一些流行病学资料显示,阅读障碍在学龄儿童中的患病率为4%,男女发病率之比为(3~4):1。书写障碍的患病率及性别比例与阅读障碍相似。计算障碍在学龄儿童中的患病率约为8%,目前关于性别比例还不是很清楚,但是有资料显示,女孩比男孩具有较高的易患素质。

\subsubsection{病因和发病机制}

(1)遗传因素:家系研究认为本症与遗传因素有关,学习能力障碍儿童的家族成员中30%~88%有阅读问题。单卵双生儿阅读困难的一致性达100%,18%~40%的阅读障碍儿童的父母亲或同胞曾有相似问题,患儿中有家族史者为正常儿童的4~13倍。对遗传方式的研究发现,本病可能为多基因遗传。

(2)器质性因素:围生期损害如窒息、产伤、宫内感染、妊娠期服药、难产、早产以及低体重儿等因素与特定性发育障碍有关。脑瘫和癫痫
患儿的特殊阅读障碍的发病率增高。也有人认为本病与颅脑外伤、感染、铅中毒有关。目前认为学习障碍是与学习有关的一种或多种知觉和语言技能的脑成熟障碍。

(3)环境因素:心理社会因素和文化教育因素对儿童早期智力开发有一定影响,家庭环境不良和早期母子关系问题可能与学习障碍有关。

\subsubsection{临床表现}

1.阅读障碍 起病于学龄前,最主要表现为阅读能力发育明显损害,而智力水平正常,没有视力和听力缺陷,没有神经系统疾病,接受教育的机会与其他儿童相等。虽然临床表现因损害的严重程度不同而有较大的差异,但所有的阅读障碍的儿童都表现出三种主要症状:①阅读不准确,朗读时常添字或漏字,错读或漏读;②阅读速度较慢,患儿经常要借助手指进行阅读;③阅读理解能力较差,朗读时不能按词组或意群停顿,缺乏抑扬顿挫,对阅读的内容很难理解。患儿正确阅读单词的能力也较差。最近的一些研究显示,阅读时主要依靠语音信息来理解单词,阅读障碍者捕捉语音信息和上下文中提供的信息非常困难,并因此出现阅读和理解困难。患儿阅读时出现字或词被替换、歪曲和遗漏。被替换的词可以是任何种类的词,如比较重要的名词和动词,或不太重要的介词等。字词被替换是经常出现的症状,特别是形似或音似的单词,意思相近的词,甚至毫不相干的词,均会出现被替换。

阅读障碍常同时伴发其他特定性发育障碍,最常见的为计算障碍、书写障碍、特定性语言表达障碍、特定性语言感受和表达性障碍,约95%以上的阅读障碍儿童同时并发一种和多种特定性发育障碍。另外,还可以有多种与认知功能有关的功能缺陷,如感知、语言、注意和记忆,也可以有运动协调能力较差。部分患儿还可以伴有社会交往技能受损。

2.计算障碍 本病主要表现为不能理解简单的算术概念,不能认识数学符号和数学标识,计算困难。本病准确的起病年龄尚不确切,部分患儿一年级即可以做出诊断。大部分患儿通过死记硬背的方法,可以应付低年级的计算,进入高年级后,死记硬背的方法不能解决问题,患儿的计算障碍将会明显显示出来。计算障碍的症状个体之间差异较大,主要分为三类:①数学符号语言障碍,即在理解和命名数学术语和概念,进行数学运算时非常困难;有的患儿对用数学符号来表达语言文字的能力很差,如将“5的平方”写成“5\textsuperscript{2}
”时非常困难。②感知障碍,表现为认识和朗读数字和数学符号非常困难,排列数字能力很差,不会分门别类地进行归纳。③数学能力障碍,表现为进行最基本的数学运算很困难,包括数数、乘法、四则运算等。这些患儿学习数学乘法口诀时会遇到非常大的困难,他们常常用连加或连减的方法,而避免使用乘法,导致计算时错误较多。

计算障碍常伴发阅读障碍、书写障碍、言语和语言发育障碍、运动技能发育障碍等。认知功能缺陷也较多见。社交技能缺陷,焦虑,抑郁,注意缺陷多动障碍也常常伴随出现。

3.书写障碍 书写障碍最突出的症状表现为作文能力受损害,常出现拼写、造句和语法错误;标点符号使用错误;语句的组织能力很差;字的书写很差,错误较多;作文缺乏主题内容。拼写错误主要表现在读音相似或形态相似的字之间,还出现不能正确读出发音近似的字。造句和语法错误包括添字、漏字,动词和代名词的使用错误,句子组织不完整,句子成分和语法结构错误。书写障碍者使用标点符号时会遇到极大的困难,甚至超过使用词语和组织文章;字的书写时经常出现笔画错误,错别字较多。作文能力很差,作文的主题、人物、情节、事情的经过不能清楚描述,结构松散,非常单调。

书写障碍常伴有语言发育障碍、运动协调技能发育障碍、注意缺陷多动障碍和社会交往问题。

\subsubsection{评估和诊断}

阅读障碍的诊断是通过对患儿的阅读准确性和理解力的标准化测验来完成的。计算障碍诊断可以通过韦氏智力测验的算术分量表以及相关学业测验。

CCMD-3诊断标准:

(1)存在某种特定学校技能障碍的证据,标准化的学习技能测验评分明显低于相应年龄和年级儿童的正常水平,或相应智力的期望水平,至少达2个标准差以上;

(2)特定学校技能障碍在学龄早期发生并持续存在,严重影响学习成绩或日常生活中需要这种技能的活动;

(3)不是由于缺乏教育机会、神经系统疾病、视觉障碍、听觉障碍、广泛发育障碍或精神发育迟滞等所致。

特定阅读障碍诊断标准:

①符合特定学校技能发育障碍的诊断标准;

②阅读准确性或理解力明显障碍,标准化阅读技能测验评分低于其相应年龄和年级儿童的正常水平,或相应智力的期望水平,达2个标准差以上;

③有持续存在的阅读困难史,严重影响与阅读技能有关的学习成绩或日常活动。

特定书写障碍诊断标准:

①符合特定学校技能发育障碍的诊断标准;

②有文字符号书写表达的学校技能障碍,其准确性和完整性均差,标准化书写表达能力测验评分低于其相应年龄和年级儿童的正常水平,或相应智力的期望水平,达2个标准差以上,但阅读与计算能力在正常范围;

③有持续存在的书写表达困难史,严重影响与书写表达技能有关的学习成绩或日常活动。

特定计算能力障碍诊断标准:

①符合特定学校技能发育障碍的诊断标准;

②有基本运算、推理能力障碍,标准化计算测验评分低于其相应年龄和年级儿童的正常水平,或相应智力的期望水平,达2个标准差以上,但阅读准确性、理解力和书写表达能力在正常范围;

③有持续存在的计算困难史,严重影响与计算能力有关的学习成绩或日常活动。

\subsubsection{治疗}

学校技能发育障碍的治疗包括特殊教育、药物和社会心理干预。最有效的方法是教育。国外的一些研究显示,“意义强化”(meaning
emphasis)和“代码强化”(code
emphasis)项目对阅读障碍的治疗有明显作用。对计算障碍最有效的治疗是系统性的教育和训练,包括对患儿计算能力进行全面的评估,制订系统的方案,包括概念、技能和解决问题的能力等。

对有行为和情绪障碍者,可以针对性地使用抗精神病药和抗焦虑药。

社会心理干预包括:支持性心理治疗、对父母亲的指导、社会技能训练、放松训练和行为矫正治疗等。

全面的治疗计划还应该包括对儿童的学习动机、学习方法、注意能力、听课方式等给予帮助和指导。

\subsubsection{病程和预后}

学校技能发育障碍大多起病于学龄前,通常入学后二年级才能被诊断,一些智商较高者可能到四或五年级后才被发现。随着年龄的增长,学习障碍可以得到部分缓解,并发症包括品行障碍、自我评价过低等。至成年期后,大多数人还存在部分残留症状,预后与障碍的程度、智商、并发症、社会经济水平以及是否早期诊断和治疗有关。

\subsection{言语和语言特殊发育障碍}

言语和语言发育障碍是指在发育早期就有正常语言获得方式紊乱,表现为发音、语言理解或语言表达能力发育的延迟和异常,这种异常影响儿童的学习、生活和社会交往功能。这一障碍不是由于神经或言语机制异常,听力或发音器官缺陷,也不是由于精神发育迟滞和环境因素所导致。

\subsubsection{流行病学}

言语和语言发育障碍的发病率与分类标准和确定异常的界限有关。一般认为患病率在2%~7%。男女发病率之比约为4:1。

\subsubsection{病因}

言语与语言发育障碍的病因尚无确切定论,目前认为与遗传、围产期损伤以及出生后的社会心理因素有关。在有语言和学习障碍的家族中发生率比一般人群高,也有较高的精神障碍共病率。

\subsubsection{分类和临床表现}

1.语音障碍 患儿主要表现为言语发音能力低于同龄人水平,但语言技能发育正常。正常儿童4岁时可以出现发音错误,7岁以后发音正确,可能偶尔有少量错误。12岁以后发音应该完全正确。而患儿发音障碍非常严重,甚至很难让人理解。

2.表达性语言障碍 对语言的理解正常,但表达能力低于相应水平,无生理性缺陷所致的发音困难。

3.感受性语言障碍 能听到声音,但是不能理解;能理解手势。

\subsubsection{评估和诊断}

在评估和诊断儿童言语和语言发育障碍时须注意下列几点:①儿童言语和语言功能必须通过标准、个体化的测验进行评估。仅通过临床观察得出的诊断是不可靠的;②对言语和语言功能的评估必须建立在特殊的非语言性智力测验基础之上。这是因为许多智力测验对语言的依赖很重,因此,当有特殊语言功能损害时其测验结果是不可信的。

CCMD-3诊断标准:

1.特定言语构音障碍诊断标准

(1)发音困难,讲话时发音错误,以致别人很难理解。患儿说话时的语音省略,歪曲或代替的严重程度,已超过同龄儿童的变异范围;

(2)语言理解和表达能力正常(韦氏儿童智力测验语言智商、操作智商及总智商均不低于70);

(3)不是由于听力缺陷、口腔疾病、神经系统疾病、精神发育迟滞,或广泛性发育障碍所致。

2.表达性语言障碍诊断标准

(1)言语表达能力明显低于实际年龄应有水平。2岁时不会说单词,3岁不能讲两个单词的短语,稍大后仍有词汇量少、讲话过短、句法错误等,其严重程度超过同龄儿童的变异范围;

(2)语言的理解能力正常;

(3)标准化测验总智商正常(韦氏儿童智力测验操作智商及总智商均不低于70);

(4)不是由于听力缺陷、口腔疾病、神经系统疾病、精神发育迟滞,或广泛发育障碍所致。

3.感受性语言障碍诊断标准

(1)言语理解能力低于实际年龄应有的水平。1岁时对熟悉的名称无反应,2岁时仍不能听从日常简单的口令,以后又出现不能理解语法结构、不了解别人的语调、手势等意义,其严重程度超过同龄儿童的变异范围;

(2)伴有语言表达能力和发音异常;

(3)非言语性智力测验智商在正常水平(韦氏儿童智力测验操作智商不低于70);

(4)不是由于听力缺陷、口腔疾病、神经系统疾病、精神发育迟滞,或广泛发育障碍所致。

\subsubsection{治疗}

治疗方法主要包括语音质量、口形运动控制、语音学和语言流畅性等方面。还包括语言的感受和处理过程,训练的目的是提高对单词的理解、命名能力、语法造句能力和高级听力、简洁表达技巧等。

根据患儿的能力,应该以训练实用语言能力为主,如理解间接含义、利用非语言线索和有效交流必需的相关技巧(如肢体语言、面部表情等)。治疗实用性缺陷是语言干预的重要方法,重点是社会技巧发育的行为塑造。如果有潜在的能力,应该考虑该方法。

\subsection{运动技能障碍}

特定运动技能发育障碍是指运动技能发育的明显延迟,常有视觉空间运动功能障碍。临床主要表现为精细或粗大运动的共济协调能力明显低于其年龄应有的水平,或标准化运动技能测验低于其年龄期望值,达两个标准差以上;患儿早期发生的运动技能障碍持续存在,并严重影响其学习和日常生活。排除视、听觉缺陷,神经系统疾病或运动系统缺陷。

\subsubsection{流行病学调查}

有报道大约6%的学龄儿童有发育性运动技能障碍。有知觉运动障碍的儿童常伴有学习障碍和情绪行为问题。

\subsubsection{临床表现}

1.笨拙 表现为简单动作无异常,但复杂动作组织能力障碍或不成熟,完成技能性动作笨拙,尤其做精细动作慢,动作幅度大,效率低;难于长时间维持静态姿势。投掷时易出现身体失衡,手-眼协调能力差。动作笨拙可能会累及一组肌群(如面肌、手和手指、肩带肌)或几组肌群,甚至全身肌肉系统。爬、跑、跳跃或跳绳、拍球或折纸动作不协调。

2.视觉空间障碍 涉及立体视知觉、认知作业的操作困难。如走迷宫困难,搭积木、搭模型、玩球、描画和认识地图困难等。他们的社会适应能力和学校学习可能会受影响,表现书写困难。并常伴有知觉、思维异常,语言障碍或发育迟缓(如口齿不请),咀嚼困难等。

3.偶然动作 过多的连带动作、舞蹈动作、震颤、肌肉抽搐等。其中连带运动最常见,可能是同源性(对称性)或异源性(不对称性)。抽搐通常以面部、口部、头部、颈部和膈肌为多见。

4.运用障碍 也称协同障碍,运动技能障碍的儿童尽管肌力、感知觉均正常,实施运动的各神经肌肉结构是完整的,但不能组织实施一系列有效的随意动作和完成技巧性动作,或学习技巧性动作有困难。

5.特殊技能运用障碍 表现为书写不能或书写困难、绘画和建构障碍、运动性语言障碍等。

6.神经软体征 神经系统软体征是发生于年龄幼小的儿童,随着年龄增长而消失。如果超过一定年龄(8~9岁)仍有该体征,则属异常。运动技能障碍患儿常有神经软体征阳性。

\subsubsection{诊断}

1.进行详细的病史收集和体格检查 详细了解病史、家族史、遗传史、孕产史、生长发育史,儿童早期生长发育是否正常,各阶段标志性发育是否正常,家庭对儿童的照管和教育环境如何。

全面的体格检查,包括躯体发育情况,有无先天性畸形,视、听觉有无障碍,神经系统软体征等。

2.神经心理测验 如Peg-Moving笨拙动作测验,Cubbay标准化测试可以用来评定笨拙程度,Bruinink-Oseretesky动作熟练测验,感觉统合测验等。

3.诊断标准

(1)精细或粗大运动的共济协调能力低于其年龄应有的水平,或标准化运动技能测验低于其年龄期望值,达2个标准差以上;

(2)早期发生的运动技能障碍持续存在,并严重影响学习成绩或日常生活;

(3)不是由于视、听觉缺陷、神经系统疾病,或运动系统障碍所致。

\subsubsection{治疗}

可以通过物理疗法和职业治疗进行训练。如:①感觉统合功能训练是:4~12岁运动技能障碍儿童重要的和常用的康复治疗方法,其意义在于,充分提供内耳前庭、皮肤碰触等感觉刺激,并科学、恰当地控制刺激输入的量和环境,促使儿童逐渐自觉地形成顺应和适应,进而激发其自信心和潜能,最终改善协调与控制能力。②动作治疗(motor
therapy):是一种有效的矫正运动障碍、改善个体动作行为的处方式治疗方法。