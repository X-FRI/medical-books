\chapter{精神疾病病因学分类与诊断}

\section{病  因}

半个世纪以来的大量探索性研究表明,精神障碍是社会、心理、生物因素相互作用的结果,但是其确定的致病因素和发病机理目前均不十分清楚,病因学是目前精神医学研究亟须解决的主要内容,因为病因的最终了解可以改变现象学的描述,而加深对疾病症状的理解,有助于精神障碍的诊断和提出针对病因的防治措施改善疾病的预后。

\subsection{与疾病有关的各种致病因素}

为了探索发病的因素,可从个体内的生物学因素、个体外的心理因素和社会环境因素三方面来寻求。三者分别从不同的层面反映各种有害因素的影响,而又相互作用影响。对于某种疾病,生物学易感性是必要因素,但不足以说明疾病发生发展的全部过程,对于另一种疾病,心理、社会因素可能是必要因素,但也不足以解释全部的病因,精神疾病的病因不是单一的致病因素,而是多种因素共同作用形成的。

1.生物因素

(1)遗传因素:遗传因素是决定个体生物学的特征,指遗传物质基础发生病理性改变,从而影响正常与异常行为,精神分裂症、情感性精神障碍、癫痫
与某些类型的精神发育迟滞都有遗传倾向。

国内外有关家谱调查首先证实了遗传因素的作用,患者亲属之中发生同类精神疾病的,比正常人口普查所得发病率明显提高,血缘愈近,发病率愈高。以精神分裂症为例,精神分裂症患者的家庭成员中,精神病的患病率比一般人群高数倍(6.2);双生子研究发现单卵双胎的同病率比异卵双胎高4~6倍,更进一步证明了遗传因素的作用。

细胞遗传学研究发现染色体畸变、数目或结构异常可导致严重的躯体和精神发育障碍,有的还引起人格异常、反社会倾向、儿童行为障碍和儿童孤独症等表现,统称为染色体病。

生化遗传学研究表明基因突变形成某种氨基酸、类脂质的代谢障碍,造成体内某种正常酶的质或量的改变,可致先天性代谢缺陷或遗传性代谢病,在已知的200多种酶缺陷疾病中,可引起精神发育障碍或行为异常者有70余种。但大多数精神障碍致病基因未明,一般认为属于复杂性多基因遗传方式。

(2)素质因素:素质因素包括体质和性格,指一个人躯体素质与内在的心理素质。

心理素质即气质和在其背景上的形成的性格,患病前性格特征与精神疾病的发生有一定关系。巴甫洛夫经实验提出四种类型:①弱型;②强不均衡型;③活泼型;④镇静型。强不均衡型者遇到困难退缩不前、忧愁无策,易患神经症。他又将人们分为:①思想型易患强迫性神经症;②艺术型易患癔症;③中间型易患神经衰弱。

躯体素质指个体以遗传为基础,在发育过程中受内外环境影响所形成的整体机能状态,包括体型大小、体力强弱、营养状况、健康水平及疾病的抵抗、损伤或恢复能力。急、慢性躯体感染和颅内感染,均可引起精神障碍。最常引起精神障碍的感染有:败血症、流行性感冒、伤寒、斑疹伤寒、肺炎、脑膜炎、神经梅毒以及获得性免疫缺陷综合征(AIDS)等;内脏各器官、内分泌、代谢、营养和胶原病等疾病发生发展过程中,各种因素引起的脑缺氧、脑血流量减少、电解质平衡失调、神经递质改变等亦均有可能引起不同精神障碍;颅脑损伤、脑血管疾病、颅内肿瘤、脑变性疾病是引起脑器质性精神障碍的主要原因,特别是脑的弥漫性损害和位于额叶、颞叶、胼胝体、基底核和边缘系统的病变更易引起精神障碍;在环境因素下,如某些体外毒物中毒等从不同途径经体内侵入脑部引起精神障碍。

(3)性别和发病年龄:性别和年龄不是致病因素,但与精神病的发生和性质有一定关系。女性受内分泌和生理周期的影响,常可出现情绪冲动、抑郁和焦虑等。同时,女性情感丰富、个性较脆弱敏感,出现心理应激时较易出现脑功能障碍,表现为各种神经症和某些精神症状。男性常因饮酒、吸毒、外伤、性病、感染等机会较多,因而易患酒依赖、脑动脉硬化性精神障碍、颅脑损伤性精神障碍和神经衰弱等。

不同的年龄可发生不同的精神疾病。儿童期由于心身发育尚未成熟,缺乏控制情感和行为的能力,对各种心理因素过于敏感,容易出现情感和行为障碍。青春期由于分泌系统尤其是性腺的逐渐发育成熟,植物神经系统不稳定,往往易患神经症或精神分裂症、躁狂抑郁症。中年期处于脑力和体力最充沛最活跃时期,生活工作易处于兴奋紧张状态,如遇生活应激易引起心身疾病和抑郁性障碍。老年期脑和躯体的生理功能处于衰退或老化时期,易患脑动脉硬化性精神障碍、帕金森氏病和其他脑退行性疾病所致精神障碍等。

2.心理因素 心理因素包括心理素质和心理应激两方面。心理因素对某些精神疾病的发生有一定作用。如心因性精神障碍、神经症和与文化密切相关的精神障碍等,心理因素起着主导作用,但不是起病的单一致病因素,是否发病主要由患者的心理素质和心理应激的性质与强度而定。心理应激一般称为精神创伤,通常来源于生活中与当事人有重要利害关系的各种重大事件(离婚、丧偶、失败、失恋、失学、家庭纠纷、经济问题等),即生活事件,这些生活事件常常是导致个体发生应激反应的应激源。适当的心理应激具有动员机体潜力、应付困难、鼓舞斗志的作用,但是对于心理素质不健全的人,过度强烈的应激常导致急性应激反应或创伤后应激障碍,对于某些精神疾病具有易感素质的人,则在一些并不特别强烈的应激影响下也会发病。

除外来的生活事件外,内部需要得不到满足,动机行为在实施过程中受挫,也会产生应激反应;长时间的应激则会导致神经症、心身疾病。

3.社会环境因素 社会环境因素指对个体心理健康产生良好的或不良的社会影响的因素,包括环境因素与文化传统。环境因素是指社会环境对心理因素的影响。如环境污染、住房拥挤、交通堵塞、社会变迁、人际关系恶劣等可增加心理和躯体应激,对精神健康产生不良影响。文化传统是指不同的民族、不同的文化和不同的社会风气以及宗教信仰、生活习惯等与精神障碍的发生密切相关。如马来西亚、印度尼西亚等东南亚国家有拉塔病(Latah)、杀人狂(Amok)和缩阳症(Koro),加拿大森林地区的冰神附体(wililge)等。

总之,生物学因素、心理因素和社会因素共同参与每个疾病的发生、发展与转归,但在某一具体疾病中作用有主次大小之分。在某些精神疾病中某种因素起主导作用,如神经症、心因性精神障碍等,心理因素是起主导作用,同时还有其他生物因素的存在;而在另一些精神疾病中某些因素起决定性影响,如精神分裂症、躁狂抑郁症等疾病,主要受生物因素的影响,如遗传、性格、年龄和性别等因素起主导作用,而心理社会因素则是发病的诱发因素,因此社会因素与心理因素的作用是经常紧密结合在一起的。

\subsection{与精神障碍病因学相关的学科研究}

1.临床描述性调查 即使不断有基因芯片、功能影像等先进技术应用于精神障碍的病因学研究,单纯的描述性调查对于精神障碍的病因学研究具有重要的价值,是精神病学研究的基础。如“精神分裂症和情感性精神障碍是病因不同的两类疾病”这一论断,正是前辈研究者基于严谨的描述性研究得出的。虽然描述性调查可做的工作已经不多,但设计严谨、操作良好的描述性调查仍将在未来的精神病学研究中占有一席之地。

2.流行病学 流行病学方法用于研究某一人群中精神障碍在时间和空间的分布情况以及影响因素,主要回答以下问题:①特定高危人群的精神障碍患病率;②具有重要病因学意义(研究重点)的因素;③精神症状的临床和社会关系。三方面的工作特别重要:界定危险人群、确定病例以及发现病例。界定危险人群是研究的基本所在;确定病例是精神疾病流行病学研究的中心问题;可用两种方法来发现病例,首先是清点在医疗和其他机构登记的已知的所有病例(公开病例),其次是在社区中调查公开和潜在(未公开)的病例。

3.心理学 心理学方法在精神障碍病因学研究中的应用有如下特征:①认为常态与病态间并非界限分明;②关注个人与环境间的交互作用;③注重异常行为的维持因素。心理学中的“学习理论”、“条件反射理论”被用于神经症病因的解释和行为治疗,并已建立了近乎合理的病因学解释。认知心理学中的“信息处理理论”则被广泛用于记忆和与情绪反应的研究中。实验心理学较之神经生化和神经心理学更广泛地用了理论体系研究,如巴甫洛夫条件反射研究。根据这种框架结构就可以对精神障碍进行研究。目前神经心理学研究中也广泛采用了脑成像技术,使与特定心理过程有关的神经环路和脑区定位成为可能,例如幻听是精神分裂症的一个标志性症状,运用功能磁共振技术在给予工作记忆任务的同时可以检查幻听的神经基础,发现出现听幻觉的受试组,加工外界听觉语言的颞叶结构的活性降低,表明听幻觉与外界语言刺激竞争使用同一部分的神经结构(Wible,
2008)。

4.遗传学 遗传学的研究主要涉及三个问题:遗传因素和环境因素在精神障碍发病中的地位和作用;有遗传基础的精神障碍的遗传方式;遗传因素如何致病。第一个问题的研究目前已取得重要进展,其余两个问题相对较少。遗传学研究方法大致可分为三类:群体和家系的遗传研究(遗传流行病学)、细胞遗传学研究和分子遗传学研究。前者包括谱系分析、双生子研究、寄养子研究、高风险子女长期随访研究及遗传流行病学研究,主要评估遗传因素在发病中的作用和疾病的遗传模式。后两者则用于识别染色体和基因的结构功能异常,以此来研究精神病遗传的机制。

5.生化研究 生化研究可用于了解疾病的原因及了解临床效应的发生机制。生物化学研究的方法很多,研究者采用了各种间接的方法,包括外周组织样本如血液、尿液、脑脊液、唾液等进行过检测,研究较易进行,但结果有时难以解释。此外,血液和尿液的检测指标很容易受到被检测者的饮食及行为变化的影响。患者去世后对其脑组织的研究可提供生化改变的直接证据,但对观察结果的解释却存在困难。首先,必须排除神经递质与酶的浓度改变出现于死后的可能性;其次,由于精神障碍本身通常不直接导致死亡,最终致死原因可能是所见脑部生化改变的原因(如支气管肺炎和服用药物过量等)。另外,生化改变亦有可能是治疗的结果而非单纯疾病本身引起的,但分子遗传学技术可补充对死后生物化学研究的不足。另一方面,在过去20年中,神经化学脑影像的发展为在体研究精神分裂症的神经生化基础带来了新的机遇。神经化学脑影像研究包括直接测量神经生化指标如受体密度、递质浓度等,或利用相对特异的神经生化探针测定局部脑神经活动。常用的两种研究方法是:①放射示踪物影像:包括正电子发射扫描技术(PET)和单光子发射计算机扫描技术(SPECT),有研究运用正电子标记的5-HT\textsubscript{1A}
受体拮抗剂,结合PET成像技术,发现重性抑郁症病人脑内受体结合降低,但抑郁自杀者死后尸体解剖结果却相反,亦提示体外与体内研究可能有不同的结果(Sargent等,2000);②磁共振影像(MRI):包括功能性磁共振(fMRI)和磁共振频谱分析(MRS)。相对于前者,MRI受试者不必暴露于放射线,在体无损伤,受试合作度高,最近还被用于检测脑血流。

6.药理学研究 研究对疾病的有效的治疗措施有助于对病因学的理解。由于直接对脑的研究存在困难,研究者希望通过研究精神药物的有效作用机制来帮助理解推断精神障碍的生化改变。但这种推断必须十分小心,因为治疗药物的药理作用并非都能直接反映疾病的生化异常。例如,抗胆碱能药物治疗帕金森病有效,但这种疾病的症状是由多巴胺能缺陷引起而并非胆碱能递质过多所致。精神药理研究中的主要问题是:第一,大部分抗精神病药物具有一种以上的作用,常常难以区别哪一种与治疗作用有关。例如,虽然碳酸锂有大量已知的药理效应,但迄今仍未发现其明显稳定躁狂抑郁患者心境作用的作用机制。第二个难点是,与实验室中药物很快出现许多药理作用不同的是,精神药物的治疗作用出现缓慢。例如:SSRI类抗抑郁药物的起效需要数周的时间。

7.生理学 生理学的方法被用于研究与疾病状态有关的中枢与外周生理功能的异常,例如,对脑血流的研究特别适用于慢性脑器质性综合征中。常应用的是电生理和生理心理学的方法。常规脑电图描记有助于研究癫痫
和精神障碍之间的关系,但在其他方面帮助不大。目前已有对视觉、听觉和躯体感觉诱发电位的研究,但这些形式电活动的确切意义尚不明。

生理心理学的测量至少在两个方面有所提示。第一种是直接反映外周器官活动情况,例如紧张性头痛患者的头皮肌电活动是否增加(EMG)。第二种是通过外周测量来推测中枢神经系统唤醒状态的变化。如皮肤电传导增加、脉搏加快和血压升高,表示患者处于较高的“唤醒”状态。脑成像技术的进展也提高了精神障碍脑血流检查的技术水平,包括前面提到的PET与SPET等断层扫描技术,可以三维显示局部脑血流的情况。另外基于血氧水平依赖成像原理的功能核磁共振(fMRI)对于神经活动所伴随的局部脑血流量增加相当敏感。另外,fMRI的优点在于其具有较高的时间与空间分辨率,不需要使用放射性物质,其特殊成像方式,如弥散张量成像技术(DTI)、质子波谱成像技术(MRS)亦有其特定的优势。

8.神经病理学和脑影像学 神经病理学研究试图回答精神障碍是否伴随大脑的结构改变(局限的或弥漫的),这在阿尔茨海默病的病因研究中获得了重要的发现,在精神分裂症和情感性障碍患者死后的大脑研究中尚未发现特征性改变,因此目前假设精神障碍是功能性的而不是结构性的。

脑影像作为一种非侵袭性研究手段,在精神障碍病因、病理、生理方面的研究越来越多。传统的结构影像可帮助确定脑部的结构与容量,包括断层扫描(CT)与MRI。功能性脑影像是在不同的条件下检测大脑功能状态、生理功能变化。包括生物化学影像核磁共振波谱MRS、药理学和生理学影像正电子发射断层显像术(PET)、脑血流影像、单光子发射计算机断层显像术(SPECT)以及功能性磁共振(fMRI)。SPECT、PET可测出脑的局部血流(rCBF)、局部葡萄糖代谢及神经受体的分布。MRS可测出脑化学成分,如去甲肾上腺素、乙酰胆碱等。fMRI将脑神经的活动视觉化,把刺激相应的脑反应状态影像化,对正常人的言语、记忆及精神分裂症患者的症状、认知功能障碍都具有结构和功能定位作用。脑影像为精神疾病病因的研究开辟了新的方向。

\section{精神障碍的分类与诊断标准}

精神障碍分类与诊断标准的制定是精神病学不断发展的重要标志之一,对精神病学分类的目的主要有:首先便于各国、各种学派之间相互交流具体病人的症状、病情的预后和讨论治疗措施;其次保证研究工作在有可比性的一组病人中进行;最终保证流行病学资料能够成为研究或制定服务计划的基础。

\subsection{精神障碍分类的发展史}

人类认识事物都是先认识现象,然后深入到本质。精神障碍的分类也是先认识症状,然后深入研究其本质。对精神障碍的本质认识是一个渐进过程,精神障碍的分类也同样处于不断发展之中。

1.国际精神障碍分类历史 (18世纪末)19世纪早期,法国精神病学家Pinel(Treastuse
on
insanity)将精神障碍分为4类,即伴谵妄的狂症(mania)、不伴有谵妄的狂症、忧郁症(melancholia)、痴呆症(dementia)、白痴(idiotism)等。19世纪初与此同时,德国精神病学家Kahlbaum对如何分类提出两个要求,即分类系统中的精神障碍定义必须以其完整的病程特点为基础;同时应包含其总体的临床相。Kraepelin(1856---1926年)采纳了此观点,在非器质性精神障碍中,首先根据临床表现与病程分出了早发痴呆(即精神分裂症)、躁狂抑郁症(即情感性精神病)和偏执性精神病,为精神障碍的分季作出了重大贡献,影响至今。1889年于巴黎召开的国际精神病学会议通过的国际分类法,将精神障碍划分为11种,1984年《国际疾病、外伤、死因分类手册(第6版)》(ICD-6)首次列入精神疾病,其第五章标题为“精神病、神经症和人格障碍”,列了精神病10种,神经症9种,人格、行为与智能障碍7种,共计26种精神障碍的病名。1957年ICD-7版本本章内容无变化。1966年ICD-8版本中增加了对全部精神性疾病的描述性定义。1975年公布的ICD-9版本其他精神病性精神障碍增加至17类,总共精神障碍扩展至30类。1992年公布的ICD-10接纳了DSM-Ⅳ制定诊断标准的方法,将前版的30类扩展至100类(F00~F99)。

1952年美国精神医学会(APA)出版了《精神疾病诊断统计手册(DSM-I)》。以简要的术语汇编了当时在美国被普遍接受的精神分析理论观点。1968年作为ICD-8的美国国家术语集出版了DSM-Ⅱ。1974年为配合ICD-9的出版,专门成立了特别工作组来制定修订版,并于1980年出版了DSM-Ⅲ,DSM-Ⅳ草案经过反复讨论和评估,在与ICD保持编码与专业技术方面一致性的基础上,于1994年出版。

2.中国精神障碍的分类历史 祖国医学很早就对精神障碍进行归纳和描述。在殷甲骨文中,已有“心疾”、“首疾”等记载。古代医家把精神障碍初步分为“癫、狂、痴、呆”四类,但未有现代概念的中国精神疾病分类系统。1958年前一直采用前苏联分类编码,直到1958年6月卫生部在南京召开的第一次全国精神病防治工作会议上,提出了一个包含14类的首个《中国精神障碍分类草案》。1978年中华医学会神经精神科第二届学术会议召开,并成立专题小组,对1958年分类草案进行修订,次年正式公布,名为“精神疾病分类(试行草案)”,将精神疾病分为10类,即:①脑器质性精神障碍;②躯体疾病伴发的精神障碍;③精神分裂症;④情感性精神病;⑤反应性精神病;⑥其他精神病;⑦神经官能症;⑧人格异常;⑨精神发育不全;⑩儿童期精神疾病。精神疾病的分类十分法为我国较先采用,沿用至今,后来ICD-10亦使用此分类法。

受DSM-Ⅲ[《美国精神疾病诊断统计手册(第3版)》](1980)制定精神疾病诊断标准的启示,1981年在苏州召开的精神分裂症学术会议上讨论制定了我国的精神分裂症诊断标准;1984年在黄山召开的情感性精神病学术会议上制定了我国躁狂抑郁症临床诊断标准;1985年在贵阳召开的神经症学术会议上讨论并制定了我国神经症临床诊断标准。1979年对中国精神疾病分类方案经过两次修订,加上对最常见的三类精神疾病逐个制定了临床诊断标准,因此定为“中国精神疾病分类方案与诊断标准第一版(CCMD-1)”。1986年在重庆召开的第三届全国神经精神科学术会议上决定成立以杨备森教授为组长的精神疾病诊断标准工作委员会,制定了我国全部精神疾病的诊断标准与分类方案。历时3年的现场测试和分类适用性研究,于1989年在西安召开的中华神经精神科学会精神科常委扩大会议上,通过了CMD-2,并迅速得到广泛应用。以后又由姚传芳教授牵头,对CCMD-2进行了修订,并于1994年在泉州召开的中华精神科学会第一届委员会上通过,形成CCMD-2修订版(CCMD-2R)。1995---2000年,陈彦方教授牵头,通过41家精神卫生机构,负责对24种精神障碍的分类与诊断标准进行了前瞻性随访测试,编写了《中国精神障碍分类与诊断标准第三版(CCMD-3)》和《CCMD-3相关精神障碍的治疗和护理》。

\subsection{分类原则}

1.病因学分类 疾病按病因分类是临床各科共同追求的理想目标,它有利于临床准确评估病变性质、疾病特点、病程预后,同时合理用药。如多发性梗塞性痴呆(指明病因与病变部位),肝豆状核变性所致精神障碍,皮质下血管病所致精神障碍,酒精所致精神障碍,系统性红斑狼疮所致精神障碍,急性反应性精神病等,就是按此原则分类。这样病因明确或比较明确的精神疾病在我国仅占全部精神障碍的10%左右。

2.症状学分类 由于受目前认识水平和科学研究能力所限,约占90%的大多数精神障碍至今仍然病因不明,因此只能按临床主要症状或症状群的不同进行分类。如:精神分裂症、躁狂抑郁症、恐惧症、焦虑症、儿童孤独症、多动症、抽动症、精神发育迟滞等就是按此原则分类。

\subsection{精神障碍的诊断标准}

诊断标准是将不同疾病症状按照不同的组合,以条理化的形式列出标准化的条目。诊断标准包括内涵标准和排除标准两个主要部分。内涵标准又包括症状学指标、病情严重程度指标、功能损害指标、病期指标、特定亚型指标、病因学指标等。其中症状学指标作为最基本的指标,又有必备症状和伴随症状之分。诊断标准的设置症状标准是指符合某种精神障碍的主要症状特征若干条;严重程度标准是指自知力、日常生活和社会功能状况明显受损程度;病程标准是指病期持续时间达到若干规定时间;排除标准是指排除其他原因所引起的类似疾病。

\subsection{单一诊断与多个诊断,单轴诊断与多轴诊断}

临床实践中常发现同一患者可以仅患一种疾病,也可同时患几种疾病,以往多以本次就医的主要疾病作为单一诊断,而现今则主张除以本次就医的主要疾病作为第一诊断外,对于所发现的其他疾病也应分别做出诊断,这样有助于全面评价患者的身体健康状况。

诊断一个精神障碍,如仅从症状单一角度诊断,显然不够全面,需要从多方面去考虑,这就是建议临床使用多轴诊断的依据。如诊断精神分裂症,必然需要考虑其人格特点、躯体状况、社会功能等。1980年美国精神病学会正式将多轴诊断原则引入DSM-Ⅲ中,明确列入5个轴。即,轴Ⅰ:临床障碍;轴Ⅱ:人格障碍;轴Ⅲ:一般医学情况;轴Ⅳ:心理、社会问题及环境问题;轴Ⅴ:功能的全面评价。

\subsection{目前常用的精神障碍分类系统}

1.国际精神障碍分类系统(ICD系统) 世界卫生组织编写的《疾病及有关健康问题的国际分类》(international
statistical classification of diseases and related health problems,
ICD)简称国际疾病分类,其发展历史已在分类的发展史中详细介绍。

ICD-10主要分类类别如下:

F00~F09 器质性(包括症状性)精神障碍

F10~F19 使用精神活性物质所致的精神及行为障碍

F20~F29 精神分裂症、分裂型及妄想型障碍

F30~F39 心境(情感性)障碍

F40~F49 神经症性、应激相关性及躯体形式障碍

F50~F59 伴有生理障碍、生理紊乱及躯体因素的行为综合征

F60~F69 成人的人格与行为障碍

F70~F79 精神发育迟滞

F80~F89 心理发育障碍

F90~F98 通常发生于儿童及少年期的行为及精神障碍

F99    待分类的精神障碍

2.美国精神障碍分类系统(DSM系统) 美国出版的《精神障碍诊断与统计手册(Diagnostic
and Statistical Manual of Mental Disorders,
DSM)》首创多轴诊断,为临床全面评估患者精神状况作出了有益探讨。其与ICD-10主要差别在于以下几个方面(表\ref{tab2-1}):

\begin{table}[ht]
    \caption{ICD-10与DSM-Ⅳ的比较}
    \label{tab2-1}
    \centering
    \begin{tabular}{lp{5cm}p{5cm}}
    \toprule
    & ICD-10 & DSM-Ⅳ \\
    \midrule
    来源 & 国际 & 美国精神病学会\\
    呈现形式 & 有不同版本供临床、研究、基层使用 &
    只有单一的版本\\
    语言 & 较为通用的语言都有相应的版本 & 只有英文版\\
    结构 & 属于ICD-10总体框架上的一部分,第五章只有一轴,但有单独的多轴系统供使用 &
    多轴\\
    内容 & 诊断指南与标准未包含障碍的社会后果 & 诊断标准包括对社会、职业及其他领域功能的社会后果\\
    \bottomrule
    \end{tabular}
  \end{table}

DSM-Ⅳ将精神障碍分为17大类:

①通常在儿童和少年期首次诊断的障碍。

②谵妄、痴呆、遗忘及其他认知障碍。

③由躯体情况引起,未在他处提及的精神障碍。

④与成瘾物质使用有关的障碍。

⑤精神分裂症及其他精神病性障碍。

⑥心境障碍。

⑦焦虑障碍。

⑧躯体形式障碍。

⑨做作性障碍。

⑩分离性障碍。

⑪性及性身份障碍。

⑫进食障碍。

⑬睡眠障碍。

⑭未在他处分类的冲动控制障碍。

⑮适应障碍。

⑯人格障碍。

⑰可能成为临床注意焦点的其他情况。

3.中国精神障碍分类系统(CCMD系统) 中华医学会精神科分会编写的《中国精神障碍分类及诊断标准》(chinese
classification and diagnostic criteria of mental disorders,
CCMD)。目前已发展到第三版即CCMD-3,其主要特点在于以前瞻性现场测试结果为依据,在参考ICD、DSM的同时,充分考虑我国的社会文化特点和传统,兼用症状分类与病因、病理分类,保留某些精神障碍或亚型,如神经症、反复发作性躁狂等的同时,暂不纳入某些精神障碍如ICD-10的性欲亢进、童年性身份障碍等,注意文字表达,使分类系统条目更加分明,可操作性更强,而且更进一步向ICD靠拢。

附:中国精神障碍分类系统第三版(CCMD-3)主要分类类别简介

\textbf{0 器质性精神障碍[F00~F09,表示ICD-10编码,以下均与此相同]}

00 阿尔茨海默(Alzheimer)病[F00]

01 脑血管病所致精神障碍[F01]

02 其他脑部疾病所致精神障碍[F02]

02.1 脑变性病所致精神障碍[F02]

02.2 颅内感染所致精神障碍[F02.8]

02.3 脱髓鞘脑病所致精神障碍[F02.8]

02.4 脑外伤所致精神障碍[F07]

02.5 脑瘤所致精神障碍[F02.8]

02.6 癫痫 所致精神障碍[F02.8]

02.9 以上未分类的其他脑部疾病所致精神障碍[F02.8]

03 躯体疾病所致精神障碍[F02.8]

03.1 躯体感染所致精神障碍[F02.8]

03.2 内脏器官疾病所致精神障碍[F02.8]

03.3 内分泌疾病所致精神障碍[F02.8]

03.4 营养代谢疾病所致精神障碍[F02.8]

03.5 结缔组织疾病所致精神障碍[F02.8]

03.6 染色体异常所致精神障碍[F02.8]

03.7 物理因素所致精神障碍[F02.8]

03.9 以上未分类的其他躯体疾病所致精神障碍[F02.8]

\textbf{1 精神活性物质所致精神障碍或非成瘾物质所致精神障碍[F10~F19;F55]}

10 精神活性物质所致精神障碍[F10~F19]

10.1 酒精所致精神障碍[F10]

10.2 阿片类物质所致精神障碍[F11]

10.3 大麻类物质所致精神障碍[F12]

10.4 镇静催眠药或抗焦虑药所致精神障碍[F13]

10.5 兴奋剂所致精神障碍[F14;F15]

10.6 致幻剂所致精神障碍[F16]

10.7 烟草所致精神障碍[F17]

10.8 挥发性溶剂所致精神障碍[F18]

10.9 其他或待分类的精神活性物质所致精神障碍[F19]

11 非成瘾物质所致精神障碍

11.1 非成瘾药物所致精神障碍[F55]

11.2 一氧化碳所致精神障碍[F59]

11.3 有机化合物所致精神障碍[F59]

11.4 重金属所致精神障碍[F59]

11.5 食物所致精神障碍[F59]

11.9 其他或待分类的非成瘾物质所致精神障碍[F55.8;F55.9]

\textbf{2 精神分裂症(分裂症)和其他精神病性障碍[F20~F29]}

20 精神分裂症(分裂症)[F20]

20.1 偏执型分裂症[F20.0]

20.2 青春型分裂症[F20.1]

20.3 紧张型分裂症[F20.2]

20.4 单纯型分裂症[F20.6]

20.5 未定型分裂症[F20.3]

20.9 其他型或待分类的分裂症[F20.8;F20.9]

21 偏执性精神障碍[F22]

22 急性短暂性精神病[F23]

23 感应性精神病[F24]

24 分裂情感性精神病[F25]

\textbf{3 情感性精神障碍(心境障碍)[F30~F39]}

30 躁狂发作[F30]

30.1 轻躁狂[F30.0]

30.2 无精神病性症状的躁狂[F30.1]

30.3 有精神病性症状的躁狂[F30.2]

30.4 复发性躁狂[F30.8]

31 双相障碍[F31]

31.1 双相障碍,目前为轻躁狂[F31.0]

31.2 双相障碍,目前为无精神病性症状的躁狂[F31.1]

31.3 双相障碍,目前为有精神病性症状的躁狂[F31.2]

31.4 双相障碍,目前为轻抑郁[F31.3]

31.5 双相障碍,目前为无精神病性症状的抑郁[F31.4]

31.6 双相障碍,目前为有精神病性症状的抑郁[F31.5]

31.7 双相障碍,目前为混合性发作[F31.6]

31.9 其他或待分类的双相障碍[F31.8;F31.9]

32 抑郁发作[F32]

32.1 轻抑郁[F32.0]

32.2 无精神病性症状的抑郁[F32.1]

32.3 有精神病性症状的抑郁[F32.2]

32.4 复发性抑郁[F33]

33 持续性情感障碍[F34]

33.1 环性情感障碍[F34.0]

33.2 恶劣心境[F34.1]

33.9 其他或待分类的持续性情感性精神障碍[F33.8;F33.9]

39 其他或待分类的情感性精神障碍[F39]

\textbf{4 癔症、严重应激障碍和适应障碍、神经症[F44,F43,F40~F49]}

40 癔症[F44]

40.1 癔症性精神障碍[F44.9]

40.2 癔症性躯体障碍[F44.4]

41 严重应激障碍和适应障碍[F43]

41.1 急性应激障碍[F43.0]

41.2 急性应激性精神病(急性反应性精神病)[F23.91]

41.3 创伤后应激障碍[F43.1]

41.4 适应障碍[F43.2]

41.5 与文化相关的精神障碍[F43.9]

42 神经症[F40~F49]

42.1 恐惧症(恐惧症)[F40]

42.2 焦虑症[F41]

42.3 强迫症[F42]

42.4 躯体形式障碍[F45]

42.5 神经衰弱[F48.0]

42.9 其他或待分类的神经症或躯体形式障碍[F45.8;F48]

\textbf{5 心理因素相关生理障碍[F50~F59]}

50 进食障碍[F50]

50.1 神经性厌食[F50.0]

50.2 神经性贪食[F50.2]

50.3 神经性呕吐[F50.5]

50.9 其他或待分类的非器质性进食障碍[F50.8;F50.9]

51 非器质性睡眠障碍[F51]

51.1 失眠症[F51.0]

51.2 嗜睡症[F51.1]

51.3 睡眠-觉醒节律障碍[F51.2]

51.4 睡行症[F51.3]

51.5 夜惊[F51.4]

51.6 梦魇[F51.5]

51.9 其他或待分类的非器质性睡眠障碍[F51.8;F51.9]

52 非器质性性功能障碍[F52]

52.1 性欲减退[F52.0]

52.2 阳痿[F52.2]

52.3 冷阴[F52.2]

52.4 性乐高潮障碍[F52.3]

52.5 早泄[F52.4]

52.6 阴道痉挛[F52.5]

52.7 性交疼痛[F52.6]

52.9 其他或待分类性功能障碍[F52.8;F52.9]

\textbf{6 人格障碍、习惯与冲动控制障碍和性心理障碍[F60~F69]}

60 人格障碍[F60]

60.1 偏执性人格障碍[F60.0]

60.2 分裂性人格障碍[F60.1]

60.3 反社会性人格障碍[F60.2]

60.4 冲动性人格障碍(攻击性人格障碍)[F60.30]

60.5 表演性(癔症性)人格障碍[F60.4]

60.6 强迫性人格障碍[F60.5]

60.7 焦虑性人格障碍[F60.6]

60.8 依赖性人格障碍[F60.7]

60.9 其他或待分类的人格障碍[F60.8,F60.9]

61 习惯与冲动控制障碍[F63]

61.1 病理性赌博[F63.0]

61.2 病理性纵火[F63.1]

61.3 病理性偷窃[F63.2]

61.4 拔毛症(病理性拔毛发)[F63.3]

61.9 其他或待定的习惯和冲动控制障碍[F63.8,F63.9]

62 性心理障碍(性变态)

62.1 性身份障碍[F64]

62.2 性偏好障碍[F65]

62.3 性指向障碍[F66]

\textbf{7 精神发育迟滞与童年和少年期心理发育障碍[F70~F79;F80~F89]}

70 精神发育迟滞[F70~F79]

70.1 轻度精神发育迟滞[F70]

70.2 中度精神发育迟滞[F71]

70.3 重度精神发育迟滞[F72]

70.4 极重度精神发育迟滞[F73]

70.9 其他或待分类的精神发育迟滞[F79]

71 言语和语言发育障碍[F80]

71.1 特定言语构音障碍[F80.0]

71.2 表达性语言障碍[F80.1]

71.3 感受性语言障碍[F80.2]

71.4 伴发癫痫
的获得性失语(Landau-Kleffner综合征)[F80.3]

71. 9其他或待分类的言语和语言发育障碍[F80.8,F80.9]

72 特定学校技能发育障碍[F81]

72.1 特定阅读障碍[F81.0]

72.2 特定拼写障碍[F81.1]

72.3 特定计算技能障碍[F81.2]

72.4 混合性学习技能障碍[F81.3]

72.9 其他或待分类的学习技能发育障碍[F81.8;F81.9;F83]

73 特定运动技能发育障碍[F82]

74 混合性特定发育障碍[F83]

75 广泛性发育障碍[F84]

75.1 儿童孤独症[F84.0]

75.2 不典型孤独症[F84.1]

75.3 Rett综合征[F84.2]

75.4 童年瓦解性精神障碍(Heller综合征)[F84.3]

75.5 Asperger综合征[F84.5]

75.9 其他或待分类的广泛性发育障碍[F84.9]

\textbf{8 童年和少年期的多动障碍、品行障碍和情绪障碍[F90~F98]}

80 多动障碍[F90]

80.1 多动与注意缺陷障碍(儿童多动症)[F90.0]

80.2 多动症合并品行障碍[F90.1]

80.9 其他或待分类的多动障碍[F90.8,F90.9]

81 品行障碍[F91]

81.1 反社会性品行障碍[F91.0;F91.1;F91.2]

81.2 对立违抗性障碍[F91.3]

81.9 其他或待分类的品行障碍[F91.8,F91.9]

82 品行与情绪混合障碍[F92]

83 特发于童年的情绪障碍[F93]

83.1 儿童分离性焦虑症[F93.0]

83.2 儿童恐惧症[F93.1]

83.3 儿童社交恐惧症[F93.2]

83.9 其他或待分类的童年情绪障碍[F93.8,F93.9]

84 儿童社会功能障碍[F94]

84.1 选择性缄默症[F94.0]

84.2 儿童反应性依恋障碍[F94.1]

84.9 其他或待分类的儿童社会功能障碍[F94.8,F94.9]

85 抽动障碍[F95]

85.1 短暂性抽动障碍(抽动症)[F95.0]

85.2 慢性运动或发声抽动障碍[F95.1]

85.3 Tourette综合征(发声与多种运动联合抽动障碍)[F95.2]

85.9 其他或待分类的抽动障碍[F95.9]

86 其他或待分类的童年和少年期的行为障碍[F98]

86.1 非器质性遗尿症[F98.0]

86.2 非器质性遗粪症[F98.1]

86.3 婴幼儿和童年喂食障碍[F98.2]

86.4 婴幼儿和童年异食癖[F98.3]

86.5 刻板性运动障碍[F98.4]

86.6 口吃[F98.5]

89 其他或待分类的童年和少年期精神障碍[F98.8,F98.9]

\textbf{9 其他精神障碍和心理卫生情况[F09;F29;F99]}

90 待分类的精神障碍[F99]

90.1 待分类的精神病性障碍[F29;F99]

90.2 待分类的非精神病性精神障碍[F99]

92 其他心理卫生情况[F99]

92.1 无精神病[---]

92.2 诈病[Z76.5]

92.3 自杀[---]

92.4 自伤[X60-X84]

92.5 病理性激情[F23.9]

92.6 病理性半醒状态[F51.8]

92.9 其他或待分类的心理卫生情况[F99]

99 待分类的其他精神障碍[F99]



