\chapter{精神障碍症状学}

\section{概 述}

人的精神活动是人脑反映客观事物时所进行的复杂的机能活动。精神疾病的临床表现为各种精神症状,也就是异常的精神活动。正常与异常是一个相对概念。正常一词在临床各科中可有不同的含义。当我们说一个人身高正常,系指该人既不太高,也不太矮,接近该区域内统计学上的平均身高值。内科医师说某人某脏器正常,表明此人该脏器无病理变化。由于人的精神活动个体差异很大,且内容复杂多样,因此精神活动的正常概念没有临床其他学科中理解那样比较确切。判断某一精神活动是否正常,必须仔细考察引起这种精神活动的各种因素,如性格特征、文化背景、当时的处境和背景进行具体的分析与判断。在观察精神症状时,不但观察症状是否存在,还要观察其出现频度、持续时间和严重程度。

正确地辨认和评价精神症状是正确诊断精神障碍的基本条件,是精神科医务人员的临床基本功之一,是学习临床精神病学的入门课。在精神病学中,精神活动正常通常使用统计学概念,即横向对比,是指处在同一文化环境的背景下,大多数人的表现为常态。一个人的举止行为越是处于平均状态,就越被认为是正常,普通即为正常。再者进行纵向对比,即与本人的过去比较是否与其一贯的人格相统一,故又有不平常之义。最后,当然还要结合其社会适应能力是否良好来综合评判。然而在实践中却远非如此简单,原因是:①人们总是不能离开社会价值观念的判断,离不开用一定历史阶段的文化价值标准和尺度来判定,在某一文化范畴背景中被视为异常的状况在另一文化范畴背景下未必会被视为异常,甚至在同一文化范畴中的不同历史阶段也是如此;②世间万事万物无不节律地变化,人的精神活动也无时不处于波动之中,面对不同的对象、时节或环境,可以有不同的表现,精神病患者亦然,并非方方面面都不正常,正常与异常之间往往有交错和交织,要与相应的背景联系并做全面观察分析才能够确定结果;③对同一患者的精神检查资料,不同的医师常常会有不同的评价结果,这既取决于医师所掌握的理论水平和实践经验,也与各自的社会阅历及日益多元化的文化价值标准不无关系;④精神检查不如体格检查或器械检查那样直观可靠和具有可重复性。由此可见精神科医师专业工作的难度及不确定性、模糊性是可想而知的。

鉴于以上种种局限,界定正常或异常,做起来并非易事。在临床实践中,客观真实的病史资料必不可少,精神检查的基本功不可马虎,器械检查和心理测试的结果不能忽视,在此基础之上作出的全面综合的分析判断才能经得起实践的检验。

\section{感 知 障 碍}

感觉(sensation)是人们认识事物的第一步,是客观刺激作用于感觉所产生的最简单的感受,它反映事物的个别属性。知觉(perception)是一事物的各种不同属性反映到脑中进行综合,并结合以往经验,在脑中得到整体的映象。当我们听演奏时,感觉得到的是高低不同的声音,知觉得到的是优美的乐曲。感觉障碍多见于神经系统器质性疾病,知觉障碍多见于精神疾病。

\subsection{感觉障碍}

1.感觉过敏(hyperesthesia) 对体内外刺激的感受性增高。对一般强度的刺激,如感到阳光特别耀眼,轻声的关门声感到震耳,普通的气味感到异常浓烈刺鼻,躯体上轻微不适感到异常难忍。多见于神经症、更年期综合征、感染后的虚弱状态等。

2.感觉减退(hypoesthesia) 对外界的刺激感受性降低。如强烈的疼痛,或者难以忍受的气味,只有轻微的感觉。严重时对外界刺激不产生任何感觉(感觉消失anesthesia)。感觉减退和消失多见于神经系统疾病。多见于疲劳瞌睡状态、抑郁状态、木僵状态以及癔症和催眠状态。

3.感觉倒错(paraesthesia) 对外界的刺激产生与常人相反或不同性质的感受。如对凉的刺激反而产生热感。多见于癔症。

4.内感性不适(senestopathia) 躯体内部产生各种不舒适、难以忍受的、异样的感觉,难以言表、难以定位的不适感。特点为部位不固定,描述单纯,引起病人的不安,常为疑病观念的基础。多见于神经症、抑郁状态。

\subsection{知觉障碍}

\subsubsection{错觉}

错觉(illusion)是对外界真实刺激的错误感知。把实际存在的事物错误地感知为另一事物。正常人在光线暗淡、恐惧紧张及期待等心理状态下,可产生错觉,但经验证后可以纠正。谵妄时出现的病理性错觉,多具恐怖性质。错觉只有在大量涌现,且无法纠正时才有临床意义。

\subsubsection{幻觉}

幻觉(hallucination)无相应的客观刺激存在而出现了虚假的感知体验。正常人也可出现幻觉,多发生在觉醒和睡眠的过渡状态,如入睡前幻觉或醒前幻觉。

1.按涉及的感官分类

(1)听幻觉(auditory
hallucination):临床最常见的一种幻觉。幻听的内容多种多样,可听到单调的或复杂的声响,如机器声、敲门声、音乐声、人语声等。幻听的内容为言语交谈,称为言语性幻听,言语性幻听可以是几个字、几句话、一段话等,言语内容若是评论患者的言行,称为评论性幻听;内容上则以嘲讽、辱骂、威胁等,多令人不快。幻听内容若是命令患者做某事,称为命令性幻听。如命令患者拒绝服药、殴打他人、自伤或自杀,这些命令患者往往无法违抗,无条件服从,因此可产生危害社会行为。

(2)视幻觉(visual
hallucination):较幻听少见,常与其他幻觉一起出现。幻视可以是简单的闪光,也可以是复杂的图像、场景等。若幻视中的形象比实物大为视物显大;比实物小为小人国幻视;在意识清晰状态下出现幻视多见于精神分裂症。在意识障碍的状态下出现幻视多见于器质性精神障碍。

(3)嗅幻觉(olfactory
hallucination):多为令其不愉快的气味。如腐烂恶臭、化学品刺激气味等,常与被害妄想交织在一起。病人可有掩鼻动作或拒食行为。

(4)幻味(gustatory
hallucination):尝到特殊或奇怪的味道。幻嗅、幻味很少单独出现,常与其他幻觉、妄想合并出现。

(5)幻触(tactile
hallucination):患者感到皮肤或黏膜表面或底下有接触、针刺、麻木、虫爬、通电感等。有的患者有性器官接触感,称为性幻觉;也可有温度幻觉、潮湿性幻觉。

(6)本体幻觉(body-sensory
hallucination):是肌肉、肌腱、关节等本体感受器的幻觉。病人身体未动,却感到被人推动或自己在运动;未讲话,却感到口、舌在活动或在讲话。

(7)内脏幻觉(visceralhallucination):内脏性幻觉产生于某一固定的器官或躯体内部。病人能清楚地描述自己的某一内脏在扭转、断裂、穿孔,或有昆虫在游走。这类幻觉常与疑病妄想、虚无妄想结合在一起。

2.按结构分类

(1)不成形幻觉(要素、原始性):指简单、无意义的声、色、光等形成的结构缺乏完整性,因只反映个别属性,系感觉性幻觉。

(2)成形幻觉:最多见,有完整结构形态的幻觉形象,常具有某种意义,系知觉性幻觉。

3.按性质分类

(1)真性幻觉(genuine
hallucination):幻觉体验来自于客观空间,被认为是通过感官获得的,形象鲜明、清晰生动,不能随意志转移消长,会向外界“投射”。其主观体验常不易与知觉区别,故而坚信不疑,多支配行动。

(2)假性幻觉(pseudo
hallucination):幻觉体验来自于主观空间,而不是通过感官获得的,形象不够鲜明,也不随意志消长,但不向外界“投射”;坚信程度则与真性幻觉一样,很少支配行动。若幻觉来源于感觉领域之外(域外幻觉),亦属假性幻觉。

4.幻觉的特殊形式

(1)思维鸣响或思维化声(audible
thought):又称思维回响(thought-echoing),当病人想到什么,就听到(幻听)说话声讲出他所想的东西,幻听的内容就是病人当时所想的事。如病人想喝水,即出现“喝水!喝水!”的声音。

(2)机能性幻觉(function
hallucination):幻觉(通常是幻听)和现实刺激同时出现,共同存在而又共同消失,在同一感官,共存共消,但两者并不融合(与错觉不同)。引起幻觉的现实刺激多为中性、无关、单调的声音,引发的幻觉有特定的意义。如打开水龙头,在听到流水声中夹着声音“辩证唯物主义!辩证唯物主义!”主要见于精神分裂症。

(3)反射性幻觉(reflex
hallucination):即现实刺激作用于某一感官产生现实体验的同时,引起另一感官的幻觉。如当病人听到关门的响声,便看到一个人的形象(幻视)。

(4)自窥症(镜像幻觉、自体形象幻视):实际不存在镜子,而在客观空间见到自己另一全部或局部的形象,且可随自身而移动。对此双重自身体像的出现,常导致惊讶或悲哀情绪。内脏自窥少见。若在实际存在的镜子前见不到自身镜像,则为阴性自窥症。

\subsection{感知综合障碍}

感知综合障碍(psychosensory
disturbance)是对客观存在的某一事物的整体属性能够正确感知,但却歪曲了其个别属性。常见的有:

1.时间感知综合障碍 患者对时间体验的判断出现障碍。如自觉时间飞逝、停滞或凝固的感觉等。

2.空间感知综合障碍 患者对事物空间距离的判断出现障碍。如病人想把杯子放置在桌子上,但由于桌子实际上距离尚远,因而杯子掉落在地上。

3.运动感知综合障碍 患者感觉实际运动的物体静止不动,或静止不动的物体在运动。如患者感到面前的房屋在往后退。

4.非真实感(derealization) 自觉周围环境事物变得模糊不清,似雾里看花,不够鲜明,缺乏真实感。如病人诉说:“我感到周围的东西似乎都变了,好像隔了一层东西似的!”“好像都是假的”。

5.视物变形症(metamorphopsia) 患者感到周围的人或物体在大小、形状等方面发生了变化。看到物体的形象比实际增大,称之为视物显大症(macropsia),反之称视物显小症(micropsia)。如一患者称看到他父亲与以前的父亲不一样了,四肢一侧长、一侧短,而头也变得非常巨大等。

6.体形感知综合障碍 又称体像障碍,指病人感到自己整个躯体或个别部分,如四肢长短、轻重、粗细、形态、颜色等发生了变化。有些早期精神分裂症病人反复照镜子(所谓窥镜症状),感到自己的脸变得非常难看,两只眼睛不一样大,鼻子和嘴都斜到一边,耳朵大的像猪耳。虽然病人还知道是自己的面孔,但模样发生了改变。如提醒病人对着镜子用眼睛衡量时,体像障碍可暂时消失,但不目测时,体像障碍则重复产生。

\section{思 维 障 碍}

思维(thinking)是人脑对客观事物概括的、间接的和抽象的反映,是人类最主要的心理功能,是认识的高级阶段。通过大脑对感知觉得到的材料进行分析和综合,抽象与概括,而形成概念,在概念的基础上进行判断和推理。思维在感觉和知觉的基础上产生,借助语言和文字来表达。

从发展心理学看,人类的思维是从直觉的形象思维逐步发展到抽象的逻辑思维。这个发展过程通过大脑结构和功能的日益完善,通过不断学习和社会实践来完成。正常的人类思维活动的特征有:

(1)目的性:思维是围绕一定的具体目的,有意识地进行的。

(2)连贯性:思维过程中的概念前后衔接,相互联系的。

(3)逻辑性:思维过程是有一定道理,合乎逻辑的。

思维障碍的临床表现多种多样,主要分为思维形式和思维内容障碍。思维形式障碍以联想过程的障碍为主要表现;思维内容障碍则主要表现为妄想、超价观念和强迫观念。

\subsection{思维形式障碍}

\subsubsection{联想障碍(disturbance of association)}

联想是指人脑中由一个概念引起其他概念的心理活动。联想障碍可表现在联想的速度、数量、结构和自主性等方面。

1.联想速度和量方面的障碍

(1)思维奔逸(flight of
thought):是联想速度加快和量的增加。其特点为:①快。联想异常迅速,一个概念接着另一个概念大量涌现,以致有时来不及表达。②多。话多,思维内容及新概念不断地涌现,且与周围现实相关,有目的的行为也增多,形式活泼生动。③高。情感高涨、声音高亢。④变。思维虽有一定目的性,但常为外界环境变化吸引而转变话题(随境转移),或因音连、意连而转变主题。

(2)思维迟缓(inhibition of
thought):联想受抑制,联想速度减慢与困难。特点为:①慢。反应和联想慢而困难,语流缓慢。②少。语量和动作减少。③低。语音低沉,情绪低落。④短。言语简短。

(3)思维贫乏(poverty of
thought):联想的数量减少,概念缺乏。其特点为:①空。思维内容空虚。②乏。概念和词汇贫乏。③无所谓。对此状况漠然处之。

(4)病理性赘述(circumstantiality):思维过程的主题转换具有黏滞性,停留在细枝末节问题上进行累赘的详述,以至于无意义的琐碎情节掩盖了主题;又具固执性,即使被打断,仍要按原思路继续赘述;尚有目的性,离题不远,终能到达预定的终点。

2.联想连贯性方面的障碍

(1)思维松弛(散漫)(looseness of
thinking):每一句话尚且通顺,但整段叙述、上下文之间的结构联系松散,缺乏一定逻辑关系,主题不突出,让人摸不着要领,不知所云,交谈困难。

(2)思维破裂(splitting of
thought):病人在意识清楚的情况下,思维联想过程破裂,缺乏内在意义上的连贯性和应有的逻辑性。单独语句的文法结构尚正确,但上下句之间缺乏内在联系,令人费解。程度严重时,甚至在一句话内部的词与词之间也毫不关联,只是词汇的堆砌,又称“语词杂拌”。

(3)思维不连贯(incoherence of
thought):在有意识障碍的背景下产生,表现类似思维破裂,但言语更零乱,语句成片段。

(4)思维插入(thought
insertion):无意识障碍时,病人感到在外力的作用下有异己思想进入自己的大脑。

(5)思维中断(bloking of
thought):无意识障碍和无外界干扰因素时,思维过程不由自主地突然短暂停顿,表现言语突停片刻,虽经提醒,再开口时却已变换了内容。因不受其意愿支配,故伴有明显不自主感。

(6)思维云集(pressure of thought):又称强制性思维(forced
thought),异己的思潮不受其意愿的支配,出乎意料的强制性涌现,内容杂乱多变,与现实无关,出现突然消失也迅速。

3.思维逻辑性方面的障碍

(1)病理性象征性思维(pathological symbolic
thinking):病人以具体的形象概念来表示某一特殊的抽象概念,带有愚蠢和荒谬的性质,其中的特殊意义只有本人明白,他人无法理解。如某一病人坚持将所有的衣服反着穿,无论如何劝说,仍我行我素。病情好转后,病人解释是表示自己表里如一。

(2)语词新作(neologism):以自创的文字、图形或符号,或将几个无关的概念、几个不完全的词拼凑成“新词”并赋予特殊的含义。

(3)逻辑倒错性思维(paralogic
thinking):病人推理过程十分荒谬,既无前提,又缺乏逻辑根据,推理离奇古怪,不可理解,甚至因果倒置。病人以无法理解的、离奇古怪的、荒谬的推理过程,得出自以为是的所谓“结论”。如某一病人开始拒食荤菜,几天后除喝水外,拒绝一切食物,当问其原因时,病人解释人是从动物进化来的,人不能吃自己的祖宗,所以不能吃荤菜;又想到动物是从植物进化来的,所以也不能吃蔬菜米面,只能喝水。

(4)诡辩症(sophistic
thinking):以无现实意义的琐事为据,用牵强附会、似是而非的、无目的、无意义的空泛议论,以形式逻辑的推理反复证明最浅显、最简单的内容。

4.思维活动形式障碍

(1)持续言语(perseveration):表现言语黏滞不前、原地踏步,在某一概念上停滞不前。除首句回答切题外,其余均与现实无关,答非所问,单调重复不变。如医生问:“你今天来做什么?”病人答:“看病”。以后医生接着提出其他许多不同问题,但病人仍持续回答看病\ldots{}\ldots{}

(2)重复言语(palilalia):病人重复自己一句话的话尾,与现实无关,有时还能意识到并无必要,但不能克服,也不因当时环境影响而产生变化。

(3)刻板言语(stereotype of
speech):病人机械而刻板地重复某一无意义的词或句子。自发地老是说一句与现实无关的话,常伴有刻板行为。

(4)模仿言语(echolalia):如同鹦鹉学舌般地模仿他人的问话,周围人说什么,病人就重复说什么,常伴模仿动作。

\subsection{思维内容障碍}

\subsubsection{强迫观念}

强迫观念(obsessive
idea)是反复、持续出现的想法、冲动或想象等,明知不对、不必要、不合理,但难以摆脱与克服。伴有主观的被迫感和痛苦感,抵抗(反强迫)是强迫观念的特征,也是与妄想鉴别的要点。

(1)强迫思维(obsessive
thoughts):患者反复、持续地出现一些想法,如怕接触细菌、病毒,怕染上某种疾病,或反复出现某些淫秽或亵渎神灵的想法。

(2)强迫性穷思竭虑(obsessive
rumination):患者反复思考明知无意义、无必要的一些问题,却控制不住地一遍又一遍地想。如:人为什么长两只眼睛,不长三只眼睛?

(3)强迫疑虑(obsessive
doubt):患者对已做过的事反复怀疑或担忧,如门窗是否关好、电源是否切断等,常导致强迫行为,反复检查门窗、电源开关。

(4)强迫冲动或强迫意向(obsessive
impulses):患者反复出现某种冲动的欲望,如攻击别人,采取危险行动等,虽然不会转化为具体行动,但使患者感到非常紧张害怕。如某一抱着孩子的母亲,反复产生将孩子从楼上扔下去的冲动与想法。

(5)强迫回忆(obsessive
reminiscence):患者对往事、经历反复回忆,明知没有实际意义,但无法摆脱,不断回忆。

\subsubsection{超价观念}

超价观念(overvalued
idea)是指由某种强烈情绪加强了的、并在意识中占主导地位的观念。多以某种事实为基础,无明显的歪曲,推理过程大体上合乎逻辑,因受强烈的情绪影响,对此事实做出片面、偏激、超乎寻常的判断,过分地评价自我,他们的想法虽然与事实不相符合,往往因为过于迷恋他们的理想而不易纠正,因而影响其行为。

\subsubsection{妄想}

1.概念 妄想(delusion)是一种在病理基础上产生的歪曲的信念、病态的推理和判断。其特点是:①无事实根据;②与患者的文化水平和社会背景不相符合;③坚信不疑,难以用摆事实、讲道理的方法加以纠正;④个体所独有的和自我卷入的。正常人可产生错误的想法,如前提不足,得出错误的结论,但通过实践验证较易得到纠正。正常人也可坚持一些错误的看法,如偏见、迷信观念等,偏见是由于人们思想方法不正确或认识水平的限制造成的,迷信观念与其生活的社会文化背景相联系。通过教育和生活经验的积累,随着知识的掌握,偏见、迷信观念可以被纠正。妄想是个别的心理现象,群体的信念有时尽管不合理,也不能称为妄想,如宗教、迷信观念。

2.妄想的分类

1)按妄想的来源分

(1)原发性妄想(primary
delusion):具有突然发生、内容不可理解,与既往经历和当前处境无关,又非来源于其他异常精神活动的病理信念。主要包括:①妄想心境:突然产生怀疑、不安和惶恐等情绪体验,感到周围发生了与己有关的情况,导致原发性妄想形成。②妄想表象:突然产生的记忆表象,接着对之赋予一种妄想意义。③妄想知觉:对正常知觉体验,赋予妄想性意义。④突发性妄想:妄想的形成突然完全,既无前因,又无后果,也无推理判断过程,无法理解。原发性妄想对精神分裂症的诊断具有重要价值。

(2)继发性妄想(secondary
delusion):是发生在其他病理心理基础(如感知、情感、记忆、思维、智能等障碍)的背景上发生、发展,而妄想的产生是对原发障碍的解释,原发障碍消失以后妄想也将随之消退。

2)按妄想的结构分

(1)系统性妄想:发展缓慢,结构严密,逐渐形成系统,且有不断泛化的趋势。

(2)非系统性妄想:妄想的内容杂乱无章或前后矛盾,支离破碎,变化不定。

两者的区别重点在于观察妄想的泛化性、逻辑推理错误的程度、固定性、单一或多个妄想的组合及其现实性等情况。若较少泛化,逻辑推理错误的程度轻、固定单一又较接近现实,则为系统性妄想;反之,则为非系统性妄想。

3)按妄想的内容分

(1)被害妄想(delusion of
persecution):是最常见的一种妄想,患者常常坚信有一种不安全感,感觉周围在监视、跟踪、窃听、诽谤、诬陷、迫害他。如有些精神分裂症患者常认为饭菜、水里有毒而拒绝吃别人烧的饭菜或提供的水,或称公安局等要来抓他等,因此表现不出门、拒食、自伤或采取攻击行为。

(2)关系妄想(delusion of
reference):患者在生活或工作学习的环境中,对与己无关的事物均认为与自己有关。如别人的言行总认为是针对自己的,报纸、电视、广播的内容也是含沙射影地讲他,周围吐痰、谈笑和一些举动都是针对自己的。常与被害妄想一起出现,多见于精神分裂症。

(3)夸大妄想(grandiose
delusion):患者常坚信自己有超人的才华,具有大量的财富,并且具有领袖人物的才干,或是认为自己出身名门贵族或是某大人物的后代等,多见于躁狂症、精神分裂症等。

(4)影响妄想(delusion of influence):也称物理影响妄想(delusion of
physical
influence),患者坚信自己的一切活动(包括精神的和躯体的)均受外界的控制和支配,这种外界的控制和支配可以来自超自然的现象,如天体、外星球等,也可以来自周围生活中的,如电波、磁场或一些仪器等,也可以来自神灵等。常见于精神分裂症或与文化相关的精神疾病。

(5)罪恶妄想(delusion of
guilt):患者常常毫无根据地认为自己犯了严重的错误和罪行,应该被严惩,认为自己罪大恶极,应该被处死,故患者常常会采取自伤、自杀或拒食的方式,或反复向有关部门交代自己的罪行,要求以劳动改造的行为来解脱自己。常见于抑郁症和精神分裂症。

(6)疑病妄想(hypochondriacal
delusion):患者毫无根据地坚信自己患了某种严重的躯体疾病,是不治之症,一系列医学检查和反复的医学验证不能纠正其病态信念,常导致反复的就医行为和焦虑抑郁情绪。

(7)嫉妒妄想(delusion of
jealousy):患者常坚信配偶对自己不忠,有外遇,为此常采取跟踪、检查配偶的行为、逼问配偶,以求证实。多见于精神分裂症、偏执性精神病、酒精所致精神障碍。

(8)钟情妄想(delusion of
love):患者坚信自己被某一异性看中、所爱,因而眷恋、追逐、纠缠对方,即使遭到对方严词拒绝,仍毫不置疑,而认为对方是在考验自己对爱情的忠诚,仍纠缠不已。多见于精神分裂症。

(9)被窃妄想(delusion of
steal):患者坚信自己的东西被别人偷走了,为此常常把东西收藏起来,并且会反复检查自己的东西。多见于老年性精神病,也可见于精神分裂症。

(10)内心被揭露感(experience of being
revealed):又称被洞悉感,患者认为自己内心所想的事周围的人都知道了,虽然病人不能确切讲出别人是通过什么方式知道的,但确信已经人人皆知,甚至搞得满城风雨,所有的人都在议论他。多见于精神分裂症。

(11)其他:如非血统妄想、虚无妄想、变兽妄想、动物寄生妄想、附体妄想、妊娠妄想等。

一般来说,妄想可使患者采取相应行为,如攻击、自伤、反复就医等。妄想的确诊:在详细地掌握了知情者提供的客观病史的情况下,一般并不难,但在其内容较为接近现实时则又相当困难。检查者要善于询问和多方面、多角度地调查,一定要以调查为前提,以事实为依据;更要善于综合分析,既要注意其内容是否违背客观事实,还应重在注意其结果的推理判断是否符合通常的情理和基本的逻辑规律。

\section{注 意 障 碍}

注意(attention)是指在某一时间内,人的精神活动选择性地集中指向一定对象的心理过程。注意分为主动注意(随意注意)和被动注意(不随意注意),主动注意是自觉的、有预定目的的,使注意指向一定的对象,为了实现这一目的,必要时需做出努力才能完成。被动注意是由外界刺激引起的探究反射、定向反应,无自觉目的,不由自主,无须加以努力,是自然地注意。

\subsection{注意程度方面的障碍}

注意程度取决于外界刺激的强度、新异性和多变性。常见的注意障碍有:

1.注意增强(hyperprosexia) 指主动注意的显著增强。病态的注意增强多与妄想有关。如有被害妄想的患者,对周围环境的细微变化和妄想涉及对象的一举一动都保持高度注意和警惕。有疑病妄想的患者,对自己身体内的某些变化和健康状况过分关注,以至于可以强化或促进某些症状的发展。

2.注意减弱(hypoprosexia) 是指主动和被动注意的兴奋性减弱,难以在较长时间内集中于某一事物。在同一时间内所掌握的客体范围显著缩小,稳定性降低而影响记忆。注意力减弱亦称注意松懈、迟钝。

\subsection{注意稳定性方面的障碍}

1.注意转移(transference of
attention) 主动注意不能持久,被动注意的兴奋性增强但不能持久,注意对象常常因周围环境的变化而转移。

2.注意涣散(aprosexia) 主动注意不易集中,稳定性差,不能集中注意某一事物并保持相当长的时间,极易分散。如花了很长时间看书,仍不知所云,就像没读过一样。

3.注意固定(fixation of
attention) 注意稳定性增强。如有顽固性妄想者,总是固定注意于妄想的内容上。

\subsection{注意力集中方面的障碍}

1.注意狭窄(narrowing of
attention) 注意范围显著缩小,主动注意减弱,当注意集中于某一事物时,其他事物引不起他的注意。

2.注意缓慢(blunting of
attention) 注意的兴奋性集中困难及缓慢。在面对接连提出的需要回答的问题时,其回答问题的速度很快变慢。

\section{记 忆 障 碍}

记忆(memory)是对既往事物经验的重现。从过程上包括识记、保持、再识和回忆(记住、不忘、认得和回想起来),四者既互相关联又密切配合。从内容上可有形象记忆,如旧景历历在目或老友的音容笑貌犹在;情绪记忆,指既往情绪体验的记忆;逻辑记忆,如语词或以往思想活动的回忆;运动记忆,包括全部习得的技能动作。从时间上又可分即刻记忆、短时记忆、近事记忆或远事记忆。

\subsection{记忆量的障碍}

1.记忆增强(hypermnesia) 对患病前不能够且不重要的事都能回忆起来,连细节都无遗漏。

2.记忆减退(hypomnesia) 是指记忆的全过程普遍减退,早期常为对涉及抽象思维的内容(如数据、术语、概念等)难以回忆。表现为记忆力减弱,或由近及远的记忆困难。

3.遗忘症(amnesia) 指局限于某一事件或时期内经历的遗忘,系“回忆的空白”或丧失,不是记忆普遍性减弱,而后天习得的运动记忆,如骑车、游泳等则一般不易消失。

(1)顺行性遗忘(anterograde
amnesia):在时间界限上是指回忆不起来疾病发生后一段时期内所经历的事件。

(2)逆行性遗忘(retrograde
amnesia):即回忆不起来疾病发生以前某一段时间内所经历的事件。

(3)进行性遗忘(progressive
amnesia):主要是再识和回忆困难,是大脑弥漫性损害引起的全面性痴呆和全面性遗忘,同时呈现日趋加重的倾向。

(4)界限性遗忘(circumscribed amnesia):又称心因性遗忘(psychogenic
amnesia),由沉重的创伤性情感体验所引发,内容仅限于和某些痛苦体验相关的事件,而在此之前或之后的记忆却保持良好,有如“不堪回首话当年”。此症状是由于大脑皮质的功能性抑制产生,并没有脑器质性损害。

(5)后发性遗忘:疾病发生以后在意识恢复的初期(如脑外伤或缺氧复苏后)记忆尚好,但经过一段时间之后,再表现出明显的遗忘症状。

\subsection{记忆质的障碍}

1.错构症(paramnesia) 为“记忆的错觉”。对过去曾经历过的事件在具体的时间、人物或地点上出现错误的回忆,造成唐汉不分,张冠李戴,且不易纠正,同时伴有相应的情感反应,常常会将生活经历中的远事近移。错构症多见于脑器质性疾病,其错构表现固定;若信口开河,多变则为功能性改变。

2.虚构症(confabulation) 为“记忆的幻觉”,以想象的、不曾经历过的事件来填补自身经历记忆的空白缺损。虚构症多发生在严重记忆损害的基础上,即刻产生,内容可生动、荒诞,但连虚构的情节也会片刻即忘,也多见于脑器质性疾病。需要注意的是,此处并不包括人格障碍的病理性谎言(以少量的事实和大量的谎言相结合)及精神分裂症患者可能对往事的妄想性解释,因为此时并不存在记忆本身的障碍。至于那些毫无事实根据与经历的“妄想性虚构”就更与记忆障碍的虚构症无关。若虚构症伴有突出显著的近事遗忘、定向障碍则为Korsakoff综合征。

3.似曾相识症(熟悉感、阴性错认) 在经历体验完全陌生的新事物之时,却有似乎早已体验过的熟悉感。似曾相识症有时也会发生于正常人,但是,很快就会认识到是自己弄错了;而在患者则常坚持认为确实是经历过的。重演性记忆错误是指对于一段时间的生活经历的似曾相识症。

4.旧事如新症(生疏感、阳性错认) 对已多次经历过的事物感到从未体验过的生疏感,是因当前感知的事物映象与以往不同而又类似的事物表象相混淆所致。潜隐记忆是旧事如新症的特殊表现,亦称歪曲记忆,系对不同来源的记忆混淆不清,相互颠倒(忘记了见到或听到的事实的来源,却不自觉地当作本人的经历来表达)。可以是把别人经历过的事回忆成是自己经历过的,也可能是把本人经历过的事回忆成是自己听过、看过或读过的。

\section{智 能 障 碍}

智能(intelligence)是智慧和能力的合称,包括既往获得的知识、经验以及运用它们来解决新问题、新概念的能力。智能活动与思维、记忆、注意和情感等活动密切相关。

能力是指善于完成某种特殊活动的个人心理活动特征。各种能力的综合便是智慧,智慧具有一定的遗传倾向。正常智能的基础是健全的大脑和适当的学习。美国著名心理学家,芝加哥大学教授布鲁姆的追踪研究得出了国际公认的研究表明:如果以17岁智力成熟作为100%的话,50%的智力是在4岁以前获得的,表明这一时期是孩子性格和体格发展的重要时期。所以,我国第一部关于学前教育的地方法规(北京市)明确规定:“提倡和支持开展3周岁以下婴幼儿的早期教育”。

智能障碍系综合性的认知功能障碍,其主要特征为:意识清晰;思维常为病理性赘述;记忆力下降;计算能力削弱;分析、综合、理解、判断和推理能力较差;学习、工作困难,社会适应能力受影响;生活自理能力正常或减退;严重时可有定向错误,可伴有人格改变、情感幼稚或行为异常等。

智能障碍可表现为全面性或部分性智能减退,程度严重时称为痴呆。根据发生的情况,主要分为两种类型:先天性(或18岁以前发生)智力低下和后天性获得性痴呆。

\subsection{智力低下}

智力低下也叫精神发育迟滞(mental
retardation)。在围生期或婴幼儿时期,大脑的发育由于遗传、感染、中毒、头部创伤、内分泌异常或缺氧等诸因素的影响而受到阻碍,以致大脑发育不良,智能发育停留在一定的阶段,患儿社会适应能力困难,智商通常在70分以下。

\subsection{痴呆}

痴呆(dementia)是指在大脑发育已基本完善、成熟和智能发育正常之后,由于有害因素影响导致大脑器质性损害,造成智能严重障碍。器质性痴呆可根据起病的缓急而划分为:

1.急性痴呆 源于急性脑病(外伤、感染、中毒、缺血、缺氧等)之后,遗留不同程度的智能损害。

2.慢性痴呆 则由于慢性进行性脑病(变性病、动脉硬化等)引起,智能损害逐步发生发展而来。

再根据大脑皮质与皮质下结构的弥漫性或局限性损害,又可以分为全面性和局限性痴呆。全面性痴呆,如阿尔茨海默病,早期表现为个性改变;局限性痴呆,如血管性痴呆,早期即有智能改变。

此外,在临床实践中,可见到与痴呆表现类似而本质却迥然不同的功能性(假性)痴呆,是由强烈的精神因素引起大脑功能性障碍所致的暂时性智能改变。主要有两种类型:

1.心因性假性痴呆 即Ganser综合征(Ganser
syndrome),表现为虽然能够解决一些复杂的问题,但是对于一些非常简单的问题都给予错误却又近似于正确的回答,表明其对于问题性质的理解还是正确的。

2.童样痴呆(puerilism) 表现出类似儿童般的稚气和使用童言童语,但仍保留着成人的知识、经验和技能。

真性痴呆与假性痴呆的鉴别要点如下:

真性痴呆多数是逐渐起病,难以确定起病的确切日期;智能损害符合心理学的规律,并与其平日的行为表现相协调;对疾病主动的诉说比较少,而且内容模糊,显得并不很关心,有时还可能力图把认识功能障碍加以掩饰或缩小,在行动上会竭力去完成作业的要求,并依靠各种方式(如记笔记等)以保持其认识功能状态;时间上具有持久性。

假性痴呆的智能障碍发生多较突然,能确定出起病日期,家人等也能觉察到其障碍的存在及严重程度;智能损害结构与其平日的行为表现之间充满着矛盾,如能够完成较为复杂的计算,却不能完成简单计算,或病始时近事与远事记忆同样严重受损,或智能严重损害,但能完成较复杂的行为;对疾病都抱有强烈的痛苦感,常主动、详尽、强调地诉说功能丧失的情况,可是在行动上并不力图去恢复其丧失的认识功能;时间不持久。

\section{自知力障碍}

自知力(insight)是临床精神病学上一个很重要的概念。从广义上说是一个人对自己的认知和态度;此处所指是“某人对其自身状况的认识”和“患者对其所患疾病的认识和了解程度”,从而成为精神科疾病的组成部分之一。自知力的丧失也是诊断精神疾病的重要依据之一,从而与非精神科疾病有所区别。在临床实践中,评定自知力时应从三个维度考虑:患者对疾病的认识,对精神病理性体验的正确分辨和描述以及对治疗的依从性。

自知力又是判断精神障碍患者好转程度极其重要的标志。症状自知力并不是简单的有或无的问题,患者对于多个症状的存在往往有各不相同的自知力。有些典型症状的本身就蕴含着自知力的缺乏,如幻觉妄想等;有些症状则恰恰相反,其本身就意味着有症状自知力,如各种神经症性症状。再有,在疾病初期,因精神症状违背了患者以往经历中通常的情理,故往往知其有误或半信半疑,因而当时可能存在部分自知力;随着病情的波动发展,中间也可能有短暂的良好的自知力,但不久即会丧失;直至疾病缓解时,自知力才会逐步得以恢复。

自知力的完全恢复必须是患者经过治疗精神病症状已完全消失,不仅能认识并承认患有精神病,还能从正常人的立场上对自己病中的体验和表现给予正确的客观分析、评价和判断,并同时具有良好的治疗依从性。

患者若还不愿意和盘托出全部的异常表现,或在回忆症状时尚有相应的情感波动而不是抱以嘲笑他的态度,或根本拒绝回忆,讳言以往,或虽然在口头上对症状进行了泛泛的批判,在行为上却仍有所流露,治疗护理的依从性不够好等,这些现象的存在都是自知力恢复不完全或根本没有恢复的表现。

\section{情 感 障 碍}

“人非草木,孰能无情”。正常人在一定处境下也会出现情感障碍的某些反应,但只有此反应不能根据当时的处境背景来说明时,方可作为精神症状来考虑。正常的情感变化特点是应有相应的刺激;所产生的情感反应与外界刺激和内心体验协调一致;持续的时间适当并保持适当的强度和稳定性。

情绪(emotion)即为个体在为满足需求的活动中,对满足程度的主观体验和行为流露的反映。具有“满意”体验的情感活动为正性情感;反之则为负性情感。实践中情绪、情感和心境等词常常互相通用。情绪好比天气现象多变化,情感好比气候背景多稳定。从心理学上对此作了划分。广义的情绪包括情感。狭义的情绪是指较低级的、生物性的、与生理需要(机体活动)相结合的相关体验;而情感是指高级的、社会性的、与社会需要(社会活动)相结合的相关体验。情绪的生物性成分主要是“食、色、性”,与此需求满足程度相应的情绪状态很简单(愉快不愉快)。情绪的社会学成分很复杂,德国冯特的三极坐标说(愉快不愉快、紧张松弛和兴奋抑制)和中医的七情说都未能将情绪完全包括,如还可以有爱国热忱、幸福豪迈或荣辱感、义务感、正义感、审美感、五味杂陈的“百感交集涌上心头”和“此情绵绵无绝期”等社会情感。随着文明社会的不断发展,情绪的社会学成分(人的社会实践)将日益居于主导地位,并对生物学成分加以改造和引导。

情感障碍的常见表现形式:

1.情感高涨(elation) 是自发的正性情感明显的高涨,对过去、现在和将来都充满着不同程度的异常的快乐、喜悦和幸福感。对外界的一切均自觉感兴趣,可出现典型的随境转移。同时,联想加快且内容丰富,言语增多且语音高亢,付诸于忙忙碌碌的行动却常常虎头蛇尾、一事无成;其知识技能在言行中可以得到发挥和利用。虽然既不深刻、也看不到自身的毛病,却不缺乏机智,从而常常能一针见血地指出别人的错误和不当,常好与5争辩;时间长了是会让人感到废话太多,而短暂的接触却能使人感到颇具风趣和幽默。联想之快速以至于自觉“用舌头都来不及表达”。其乐观情绪与环境之间协调一致,具有一定的渲染力。

2.欣快(euphoria) 无精力充沛和活动增多,是空虚的高兴,是一种伴有异常的身体舒适感的高度心满意足的状态。欣快与情感高涨有着本质的区别:此是被动和无所作为地沉浸在体验之中,心理上是封闭自足的,对外界的注意及兴趣反而是明显减弱的。联想和言行的量及内容并不丰富,始动性和进取性都削弱。缺乏机智,对知识技能的利用下降,无创造性,无自知力,容易出现脱离现实的荒诞简单的想法。肤浅的欣喜,呆傻的表情常显得刻板单调,与环境不协调,缺乏渲染力,多伴有智能障碍,常作为器质性症状存在。

3.销魂状态 在极乐状态下自觉良好,如逢大喜,处于飘然欲仙的状态,但无思维奔逸,也不一定伴有精神运动性兴奋,可有轻度的意识障碍,见于毒品成瘾者在过瘾之后与癫痫
先兆或精神性发作状态之中,往往与上帝、天使或神灵等相关,并伴异常幸福感。

4.情绪低落(depression) 情绪低沉,常常少言少动,语音低微,反应迟钝;终日忧心忡忡,兴趣索然;积极性和动机丧失,感到无法振作;丧失自尊与自信;自我评价过低,可有自责自罪,消极悲观与失望,甚至有自杀的意念或行为。这种情绪低落不以环境中的喜乐所动,自觉“高兴不起来”。如有亲朋好友来访,可能漠然置之;或面对喜讯的到来也表现充耳不闻而毫无表情。应当注意的是,此时莫要轻率地认为存在“情感淡漠或不协调”。若能结合当时的具体情况稍做启诱,便可见其有流泪等伤感的情感反应外露,表明当时存在着抑郁的内心体验,并非是无动于衷。

5.情感平淡(apathy) 又称迟钝,对于比较强烈的精神刺激也不会引起鲜明生动的情感反应,显得平淡,缺乏与之相应的内心体验。具体评定时要注意排除相对封闭的住院环境和使用抗精神病药后可能的影响。情感平淡的症状应是逐渐发展和长期存在的,是情感普遍而深刻的变化,治疗趋向好转后恢复的过程也是缓慢的。这一情感变化不仅限于外在的表情和言行,更重要的是恰似一潭死水的主观体验,外界刺激难以在情感上激起波澜。若进一步发展到内心体验极度贫乏,对外界任何刺激均缺乏相应的情感反应,与环境失去了感情上的联系,即称之为情感淡漠。

6.焦虑(anxiety) 焦虑的主观体验是没有明确对象和具体内容的恐惧,有如大祸临头般的惶惶不可终日,总自觉有迫在眉睫的危险,同时也知道威胁或危险并不实际存在,却不明白为何如此。客观表现主要是运动性不安:手震颤,肌肉紧张所致的躯体多处的痛胀不适;重者坐立不安,不时有小动作,来回踱步或捶胸顿足。再者是自主神经功能紊乱,出现自主神经功能亢进症状:面赤、口干、出汗、气急、心悸、食欲不振、便秘或腹胀、腹泻、尿频尿急、易晕倒等等。若在观念上不与任何确定的生活事件或处境相联系,则为无名(浮游)焦虑。惊恐发作是一种间歇性发作的极端焦虑不安和惶恐状态,伴有明显的自主神经功能紊乱。

7.恐惧(phobia) 这是一种不以患者的意志、愿望为转移的恐惧情绪,患者常对平时无关紧要的物品、环境或活动产生一种紧张恐怖的心情,甚至感到这种恐怖感是不正常的,但无法摆脱。恐惧的内容很多,可以是动物、某些物体、所处的环境、场地等。

8.易激惹(irritability) 表现出动辄为小事引发短暂、剧烈的情感反应,极易烦躁激动、发怒,甚至大发雷霆或有冲动行为。出现在慢性器质性疾病时,为小事易激惹和大事上则漠然处之;精神分裂症则可能继发于幻觉妄想,或是毫无缘故、来去均突然,事后如同无事一般,无自知力;躁狂状态也可有易激惹,一般多事出有因,并不否认事实;神经症的易激惹则是极力自控,发泄以后又后悔,对象多局限于家庭内或家人。

情感暴发(emotional outburst,
raptus):是指在一定精神因素作用下,发生和中止均较突然地暴发情绪(如吵闹、哭笑无常、捶胸顿足、满地打滚等),常常出现戏剧性的变化。

在激惹性增高的同时还伴有强度过分剧烈的反应,突发的、强烈而短暂的情感暴发,存在一定程度的意识障碍,发作后入睡,事后又不能回忆,此为病理性激情。

9.情绪不稳定 指情绪反应极易变化,易于诱发也易于消退,从一个极端波动到另一个极端。当情绪的自控能力减退时,首先出现情绪不稳定,再进一步发展即为情感脆弱,表现为情绪反应过敏,易悲易喜、易感动,一旦流泪或发笑,便失控而痛哭或大笑,其严重形式又称情绪失禁。

10.情感不协调 情感反应与其当时的内心体验及处境不相协调。如在述说自己的不幸遭遇或妄想内容时,好像在讲述与己无关的故事一般,缺乏应有的相应情感体验,或表露出与之不相称的情感反应。情感不协调为思维与情感活动之间的不协调。若情感表现与其内心体验或客观刺激及处境相反(遇到悲痛的事反而表现喜悦),称为情感(表情)倒错(parathymia)。若内心体验本身发生矛盾,即同一人对同一事件同时产生两种相反的情感体验(既悲又喜),此为矛盾情感(ambivalence)。情感反应与其年龄不符,如同幼儿一般,此谓情感幼稚。

\section{意 志 障 碍}

意志(will)是人类认识活动过程进一步发展的结果,与情绪密切相关,并体现社会性需求的高级心理活动,有社会优劣之分,常体现在对意向的克制与调节。意志是人由于某种动机和需要,自觉地选定目标,付诸行动,克服阻力,以实现预定目标的心理反应过程。意志的强弱常取决于情感。

意志特征的表现在于克服阻力。有对某些与现实不相适应的需求(与欲望对立)采取内部的自我克制;也有为满足某种正当需求与实现崇高理想而坚忍不拔(与欲望不对立),以克服外部阻力;当存在几种互相矛盾的动机或需要时,则又表现出意志的选择能力。直接推动意志行为的力量称为动机。行为即指有目的、有动机的行动。

意向是与人的本能活动有关,体现生物性需求的低级心理活动,有强弱、多少之分。若食欲、性欲或防御等不恰当的意向过于增强亢盛时,正体现出正当的意志控制力的不足。

单一的意志障碍不能作为诊断依据,必须结合其他表现综合分析。当存在思维或情感障碍时,必定也存在着意志障碍。实际上任何一种精神病状态均已包含着意志减低,即使是在偏执性精神障碍或情感性精神障碍躁狂发作时也是如此。其意志增强是有条件限制的,仅限于病理状态部分的意志增强,而正当的意志活动是减低的。

1.意志增强(hyperbulia) 在病态的情感或妄想的影响下,意志活动增多,可坚持某种具有极大顽固性的行为。如情感高涨时活动增多,过于忙碌而不觉得疲劳;受被害妄想支配时,不断地追查不休,到处控告妄想对象等;嫉妒妄想者对配偶的监视、跟踪不止等。有的可出现本能意向活动的亢进,行为多而紊乱,无明显目的性与环境不协调。

2.意志减退(hypobulia) 表现为意志活动减弱。常由于情绪低落以致意志消沉,活动减少,懒于料理一切,需他人督促才行。本人尚能意识到这些变化,自知力部分存在,尚不脱离环境。实际上并不缺乏一定的意志要求,只是感到做不了或自觉没有意义而不想做,常见于抑郁症。更严重者则称为意志缺乏,常与情感平淡(abulia)、淡漠和思维贫乏并存。其表现对任何活动都缺乏动机,处处丧失主动性,被动,懒散。患者本人对这种病理变化毫不觉察,无自知力,孤僻、退缩与环境不相协调。见于精神分裂症衰退期或器质性痴呆综合征。

3.矛盾意志(ambitendency) 也系意志减退所致。对非常简单的事(而不是复杂的事)同时产生对立的、互相矛盾的意志活动或两种行动意向的交替。如遇到友人握不握手,在伸手似乎要握时又随即缩回手,当别人放了手时却又伸出了手。患者本人对此毫无自觉,无自知力、无痛苦、无纠正的要求,也见于精神分裂症。

4.意向倒错(parabulia) 指意向要求与一般常理相违背或为常人所不允许、难以理解。如自伤、性欲倒错、拔毛发或异食症等表现。在幻觉妄想支配下也可发生,并有荒谬的解释。

5.病理性意向过强 如精神活性物质的依赖、成瘾者,为了获取该物质会千方百计、不择手段地去追寻,而置工作、生活、学习、家庭和前途于不顾,甚至可能发生危害社会的行为。

\section{行 为 障 碍}

精神运动是指有意识的行为,以有别于单纯的躯体性运动。精神运动性障碍有两大类:

\subsection{精神运动性兴奋}

精神运动性兴奋(psychomotor excitement)是指动作和行为的增加。

\subsubsection{协调性兴奋}

协调性兴奋亦称躁狂性兴奋(manic
excitement),包括情感高涨、思维奔逸、意志行为增强伴有自我感觉良好的满足感,核心是情感的高涨影响并支配其行为。认知、情感、意志各过程之间以及与周围环境基本保持协调,目的明确,具有可理解性。焦虑激动时的兴奋和轻度的销魂状态也属协调性兴奋。

\subsubsection{不协调性兴奋}

表现认知、情感、意志各心理活动过程之间互不协调,与周围环境也不协调,缺乏目的意义,难以理解。

1.青春性兴奋(hebephrenic
excitement) 情绪变化莫测,动作和行为无明显动机,缺乏一定的指向性,与其他精神活动之间的统一性及完整性被破坏殆尽,行为杂乱,表现出幼稚、愚蠢、装相、作态和戏谑等,让人无法理解,常有本能意向增强的色情行为。见于青春型精神分裂症。

2.紧张性兴奋(catatonic
excitement) 常常突然发作,持续较短的兴奋,具有冲动性,也可表现刻板单调或无端的攻击性行为。既无明显原因,也无确切的指向和目的性,使人防不胜防,可与紧张性木僵交替发作。见于紧张型精神分裂症。

3.器质性兴奋(organic
excitement) 存在不同程度的智能、定向和意识障碍。行为多杂乱,常有冲动性,无目的性,情绪欣快、不稳定或有强哭、强笑,下午或晚上有加重趋向。多见于脑器质性精神障碍。

\subsection{精神运动性抑制}

精神运动性抑制(psychomotor
inhibition)主要为言语、动作和行为的减少。举止缓慢,活动明显减少,但并非完全不动。表现为问之不答,唤之不动,表情呆滞,而姿势较自然,无人时尚能自行进食、排便,此为较多见的亚木僵。严重时运动完全抑制,不食、不语、不动,表情固定,持续24小时即可评定木僵症状的肯定存在。有时判断木僵者是功能性还是器质性或有无意识障碍确实不易,临床上难以就木僵症本身进行病因学诊断,故应重视病史调查和神经系统检查及电生理、影像学等技术检查,以充分排除器质性病变。为了避免误诊误治,此时最好假定其存在意识障碍,以免延误必要的其他专科的检查、治疗;而在服务态度上,则又应假定其意识是清晰的,以免造成不良的心理影响。

1.木僵状态(stupor) 常见的木僵类型有:

(1)抑郁性木僵(depressive
stupor):常由急性抑郁引起,多见为亚木僵。在反复劝导或耐心询问下,尚能做出少许反应,如点头或摇头,或微动嘴唇低声回答,表情与内心体验一致。若在一旁谈其身世和不良遭遇或与其密切相关的事件时,也可见到暗自流泪或欲哭等表情的流露。

(2)心因性木僵(psychogenic
stupor):由突发严重而强烈的精神创伤引起,伴有自主神经系统症状,如心动过速、面色苍白等,有时可有轻度意识障碍,事后常无完全的回忆。当精神因素消除或环境改变后,木僵也可随之消除,一般持续时间很短。

(3)紧张性木僵(catatonic
stupor):意识清晰,木僵解除后能够回忆当时的状况。木僵表现为不同程度的少动或完全不动,严重木僵者甚至出现含涎、大小便潴留。有时患者表现同机器人一样不加选择地服从他人的任何指令,就连令其做很难堪的动作或姿势的指令也照办不误。其极端形式称之为蜡样屈曲(waxy
flexibility),表现为肢体如同泥蜡铸般地任人任意摆布。若抽去枕头,其头部仍以悬空于床面的姿势长时间躺着,称为空气枕。此时患者是完全清楚地知道别人在对他摆布,却不能抗拒。作态与特殊姿势,表现为几乎成天以一特殊的姿势站、坐或躺着,有时多动或兴奋,具有发生和结束均快且突然的特点;很少言语或不语,以身体运动为主,常有破坏或攻击性行为,但不可理解;动作常具刻板、作态等特点。往往与紧张性兴奋交替出现,这即成了精神分裂症的亚型,故也常伴有此症的其他症状,如思维形式障碍、幻觉妄想或情感平淡、倒错等。

(4)器质性木僵(organic
stupor);存在意识或智能障碍,大小便失禁而不是潴留,罕见含涎,更无典型的蜡样屈曲或空气枕。可见于脑炎、脑肿瘤、癫痫
、老年痴呆、神经梅毒等。可有神经系统及器械检查相应的阳性征像。基于此,应主张对所有木僵者做全面详尽的检查,否则不能轻率地认为是功能性的。

2.违拗症(negativism) 患者对于向他提出的要求没有相应的反应、加以拒绝或采取相反的动作。如对于要求其做的动作不但不执行,反而表现出相反的动作,如令其张口,却偏偏闭口或嘴巴闭得更紧,此为主动性(阳性)违拗(active
negativism);如只是拒绝执行吩咐而不表现相反的行为,称为被动性(阴性)违拗(passive
negativism)。

3.缄默症(mutism) 患者缄默不语,也不回答任何问题,有时可以做些手势示意,见于癔病及精神分裂症紧张型。

本能行为障碍:人类的本能可归为自我保存和种族保存的本能,具体为躲避危险、饮食、睡眠和性的本能。

\subsection{自杀}

自杀(suicide)一般是属于自我保存本能的障碍,指在观念上有想死的意念和用行动结束自己生命的决心,行动上采取了导致死亡的行为,结果可能是死亡、伤残或无恙,后者为自杀未遂。自杀者中患有精神障碍的人究竟占多大比例尚无定论,一般认为有精神障碍者的自杀率要远远高于一般人群。

1.抑郁症者的自杀 在临床实践中,自杀的最大危险人群来自于抑郁症患者。据研究发现,曾有过一次抑郁严重到需住院程度的人,最后有1/6死于自杀。曾自杀未遂者则尤其危险,再次行动时往往采取几乎必死的方法。一般来说,抑郁症者并不隐瞒自杀的意念,只是在人们的严密防范下欲死不能,遂可能谎称心情好了,不想再自杀了,甚至装出笑脸,主动了解周围情况,或只是泛泛地讲些公式化的大道理,以假象麻痹人们,借以趁机采取自杀行动;若确有好转,应能够具体生动地描述其心情从抑郁中摆脱出来的实际经过及体验。

抑郁症者可有两种变异形式的自杀:①扩大(利他)性自杀,是基于怜悯家人在其身后可能遇到的困境,故先杀死家人然后再自杀;②曲线自杀,是由于长期抑郁的折磨或自杀未遂而决心自己去杀人闯祸,被害者往往是与其毫无关系或偶遇的陌生人,杀人后非但不逃逸,且常自首认罪伏法,以求速死,实际上这才是凶杀的真正目的。

2.精神分裂症者的自杀 多数是在病态体验支配下发生的自杀,又称伪自杀。即并无想死的观念,却有导致死亡的行为或后果。引起自杀的常见精神症状如下:被害妄想、罪恶妄想、焦虑、抑郁、嫉妒妄想、疑病妄想、关系妄想、被控制感、命令性幻听或思维逻辑倒错等,一般为多个症状的组合。近年来因治疗及时,已使精神分裂症早期的自杀率有所下降。要注意的是,在疾病的缓解恢复期也可能发生自杀,因为自感在病中对他人的骚扰而无地自容,或在社会适应中(如受到歧视或在就业、就学和择偶时)遇到难以排解的困境而出现悲观失望的情绪,以致自杀。这种疾病缓解恢复期的自杀也称为真自杀。另外,有人也认为应用了抗精神病药,可能因出现无法耐受的药物不良反应而使自杀率上升。

3.类自杀或准自杀或自杀姿势 并无坚决非死不可的观念或根本就不想死,采取的行动主要是一种呼救行为,想得到别人的理解、同情、支持和帮助或以此相威胁、要挟和表示抗议。患者往往采取一些致死可能性不大的方法:如当众投水、服毒或在闹市区登高,或服用一些不及致死量的药物,或在估计能够被很快发现而获得抢救的时间、地点采取行动,事前常反复公开扬言要自杀。对此应当给予及时的心理治疗,切忌以讥讽、冷漠的态度相对,更忌用激将法,以免弄假成真。

\section{意 识 障 碍}

意识(consciousness)作为专门术语,在不同领域或学科中有着不同的含义,甚至在同一学科内的各个学派对该词的理解和使用也会有分歧。精神病学家多认为,意识是大脑普遍的功能状态,使人正确而清晰地认识自我和周围环境并作出适当的反应,涉及许多心理活动并成为其基础。凡意识障碍必定有一般性感知觉削弱及注意、记忆障碍,此乃最重要的指征。无意识障碍即意识清晰,但后者并不等于精神正常。

正常意识状态下,意识清晰,大脑皮质处于最适度的兴奋状态,能够进行正常的精神活动。维持意识清晰是全脑的功能。20世纪50年代初就已证实上行网状激活系统(ARAS)对维持意识清晰起着关键作用。ARAS受损时出现意识障碍。边缘系统和意识状态也有关,内侧边缘通路与ARAS有密切联系。大脑皮质对维持意识清晰虽然必要,但是并不存在专司意识的区域中枢。另外,清晰的知觉也有赖于运动系统的参与,是指任何感官的知觉模式都有运动模式参与。可见意识清晰和意识障碍的生物学机制涉及复杂的反馈途径以及多系统的协同活动。

意识障碍分为两大类型:

\subsection{周围意识障碍}

\subsubsection{以意识清晰度下降为主的意识障碍}

1.昏迷(coma) 表现为对任何刺激都无反应,无言语、无自主动作,肌肉松弛,惟保存生理性的无条件反射(如浅反射、光反射、角膜反射、保护反射、肌腱反射等反射),可以出现病理性反射。根据生理性反射存在的程度可分为:深昏迷:反射消失或明显减退;浅昏迷:反射基本存在,但无法唤醒;中昏迷:介于两者之间。功能性障碍不会发生昏迷。

2.昏睡状态(sopor) 为深度睡眠状态。在强刺激下才能被唤醒,但持续很短暂就又入睡。言语反应接近消失或反应不全,无痛苦表情,无随意运动,大小便失禁,痛觉反应迟钝,失去与环境的接触能力,亦称昏迷前期或半昏迷。进入昏睡或从中醒转是个渐进的过程。癔病可有昏睡,特点是精神因素诱发,发生和消失均很突然。

3.混浊(反应迟钝)状态(confusion) 以各心理过程的反应迟钝为特征。对外界刺激的反应阈值上升,表现为茫然呆板,联想理解困难,可以回答简单的问题,但常重复别人的问话或回答时词不达意。患者多处于半睡状态,或虽醒在床上却显得很迷惑,有定向障碍和发作后遗忘。广义上讲,从意识清晰到昏迷之间各种不同程度的意识障碍都可称为意识混浊。临床上常用其狭义,指的是以知觉清晰度减低为主,而无附加症状(如精神运动性兴奋、错觉、幻觉、一过性妄想等)的意识障碍。一般多见于器质性疾病。急性精神障碍发作时也可有意识混浊,尤其在兴奋或瓦解严重的发作时。在紧张、焦虑和恐惧等心理因素与生活时程紊乱、过于劳累、饥饿、脱水等生理因素相结合而引起的急性精神障碍也是多以意识混浊为主要临床相。

4.嗜睡状态(drowsiness) 一唤就能被唤醒,但无自发言语,可以有简短的言语交谈及运动反应,刺激一消失即又入睡。可见于常人或过量服用药物者。

5.酩酊状态 皮质抑制过程减弱,表现在意识清晰度下降的基础上伴有丰富的情感体验,情感反应不稳,易激惹,言语和动作增多。在催眠药中毒昏迷前期的兴奋状态、醉酒或缺氧时均可出现。

\subsubsection{以意识范围改变为主的意识障碍}

1.朦胧状态(twilight) 意识活动集中于狭窄而孤立的范围,伴有清晰度的下降,对周围事物的感知困难,发生和终止均突然,历时数分钟或至数日,发作后多陷入深睡,可反复发作。不能正确评价周围事物并有定向障碍,可有相应的情感或攻击性行为、片断的幻觉妄想等,自主神经系统功能紊乱明显,事后完全或片断遗忘。癫痫
朦胧状态的发作形式呈刻板性,有发作先兆症状,中止后不会立即清醒。癔病朦胧状态持续时间较长,有选择性的内容,发作后能够迅速清醒,重要的是在其缩小了的注意范围内,知觉并无一般性削弱。这一点与意识障碍是不同的,实为一种意识改变状态。另外,还有与睡眠相联系的朦胧状态。发生在睡眠初期的称为半睡状态;发生在睡眠末期的为半醒状态;发生在睡眠中期的则为睡行症。

2.漫游性自动症 是意识蒙眬状态的一种特殊形式。表现为行为目的性不明确,与当时的处境不相适应,或为无意义的动作,以无幻觉妄想及情绪改变为特点。如漫游发生于睡眠中则为睡行症,又称梦游症(somnambulism)。癫痫
性睡行症在发作时是无法被唤醒的,主要表现为:不辞而别地离开平日常居之地;发生在醒觉时;事先无目的构思;发生突然、结束也快;事后遗忘,发作时的身份障碍则可有可无;当时有意识障碍。在多种性质不同的精神障碍中也可有发生,如癔症、脑外伤、脑肿瘤、精神分裂症、各种抑郁或焦虑状态等。

\subsubsection{以意识内容改变为主的意识障碍}

1.谵妄(delirium) 是一种中等到严重程度的意识混浊,至少有以下附加症状之一:错觉、幻觉等知觉障碍(以真性为主,生动清晰,具恐怖色彩);言语不连贯;精神运动性不安,行为瓦解,习惯性或无目标导向的动作(但攻击性行为很少);短暂而片段的妄想,有昼轻夜重的特点,当时的自我意识存在。发热性谵妄,高热时出现,热退后即缓解;传染病性谵妄,高热时不一定发生,在疾病后期、衰竭时出现;癫痫
性谵妄,存在癫痫 病史,幻觉较多。

2.亚谵妄 由Mayer-Gross提出的比谵妄程度轻的一种意识障碍,其程度轻且波动不定,清醒时自己似乎了解病态,感到无能与困惑,会问“我在哪儿?在干什么?”数分钟后又回到模糊状态,无自知力,思维凌乱,幻觉零碎且不完整,不生动恐惧,有重复或无目的不协调的动作。多见于感染、中毒性疾病。

3.梦呓性谵妄 言语改变,像说梦话似地喃喃自语,让人听不清。因脏器功能衰竭,故兴奋不严重,多限于在病床周围的单调刻板的抓握动作,摸索或拉扯被单等不可理解的行为,对外界刺激缺乏反应。多见于老年人,是疾病严重的表现。

4.精神错乱状态(amentia) 比谵妄更严重,以精神活动不协调、不连贯和无法理解等为特征。极度兴奋,情绪惶惑、恐惧、焦虑,片断错、幻觉,但内容不恐怖、不丰富,思维极不连贯而无法接触,也可有片断的妄想观念,环境意识和自我意识均丧失。历时数周到数月,预后有遗忘或以死亡结局。多见于感染中毒性疾病。

\subsubsection{例外状态}

1.病理性醉酒 在意识的清晰度、范围和内容上均有改变,一般从不饮酒或对酒耐量很差的人,在饮用较少量的酒以后突然出现的意识障碍,迅速进入谵妄,极度兴奋,可有幻觉妄想及盲目的攻击和危害行为。一般发作短暂(数小时或一天),通常以深睡结束,醒后有遗忘。与酩酊不同,不存在情感欣快。不是由于饮酒过量而是个体对酒的过敏反应,且无明显的中毒性神经体征,再饮试验可以重现发作。在原有癫痫
、脑外伤、动脉硬化的基础上易于发生。

2.病理性半醒状态 Gudden于1905年首先报道,一般不见于自清醒入睡的过渡阶段,而见于自深睡觉醒的过渡阶段之中。在长期睡眠不足、过劳后深睡或噩梦之后发生,也可在大量饮酒、激情状态、心境恶劣等情况下于入睡以后发生。此时意识尚未清醒,运动功能却已经恢复,因无皮质的调控使得动作带有自动症的性质,表现为强烈的惊恐反应,出现错、幻觉和妄想性感知体验、冲动攻击性行为,发作短暂,发作后有遗忘。多为功能性改变,可能由精神创伤所致。

3.病理性激情 在意识清晰度下降的基础上出现情感暴发及一系列攻击性行为。自发,或虽有一定的外界刺激也与现实不匹配。发生很突然,短暂而强烈,无理智、无指向性,多见于癫痫
。有时也可有指向性,但是缺乏预谋与计划,可见于癔病、脑外伤或冲动性人格障碍等。在精神分裂症或急性应激反应时也有发生。发作时无控制力,发作后即入睡,事后也不能回忆为其特点。

\subsection{自我(人格)意识障碍}

W. James将自我分为:物质的我、社会的我、精神的我和纯粹的我。K.
Jaspers又将纯粹的我发展为自我意识学说,将自我意识划分为4个形式:

(1)自我的能动性:自己的心理活动都被体验为属于我所支配之。(我受我的支配)

(2)自我的现在统一性:感到自我在任一瞬间是统一的单一整体。(我是单一独立的我)

(3)自我的历史同一性:始终体验到在不同时间里的自我是同一个个性。(现在的我和过去的我是同一个我)

(4)自我的界限性:体验到我与非我是截然不同的,与其他事物之间是有界限分隔的。(我是我,你是你)

在病理情况下,这四个方面都可能出现问题,即称自我意识障碍。

1.自我能动性障碍 主要为各种被动体验和人格解体。人格解体(depersonalization)是指丧失对自身完整性、同一性和自身行为的现实体验。自我人格解体:觉得自我产生了特殊的变化(但未变成他人或其他),与原来的自我有了不一样、不真实、不存在的感觉(即狭义的人格解体);现实人格解体(非真实感):觉得周围现实产生了特殊的变化,变得不真实、不存在了;精神性人格解体:自觉得精神和肉体分离了,成了没有灵魂的肉体;躯体人格解体:自感到躯体分离或裂开了。若发生在精神分裂症时,多复杂、多变及不固定;出现在神经症或抑郁症时,则多单一、固定。

2.自我现在统一性障碍 主要表现是双重自我体验。双重人格(dual
personality),在同一时间内分裂成两种人格,往往相互对立,而以其中的一个为主,争着实现各自的意志和行为。多重人格(multiple
personality),同时出现了两种以上的人格且顽固荒谬,多见于精神分裂症。

3.自我历史同一性障碍 是出现了化变妄想和人格、身份的转换(交替人格alternating
personality)。在不同时间内的同一人表现出完全不同的个性特征和内心体验,并可交替出现,其中的一种人格对另一种人格可以完全不了解,常常以第三人称来称呼自己。在精神分裂症或癔症都可出现。

4.自我界限性障碍 在精神分裂症也多见,如思维被播散、被强加、假性幻觉或读心症等。

意识障碍的判断:严格地说并没有直接检查意识障碍的方法,只能借助于有关心理活动的表现,才能以此推断是否存在意识障碍。意识清晰到意识障碍是一个连续系统,只有量的差别而无显的分界线。轻微的意识障碍与精神症状的区别往往是临床实践上的难题,而且意识障碍的深浅又常常会有波动,被检查的对象往往也不合作,难以接触,可是要确定其当时有无意识障碍又显得十分重要。对于短暂的意识障碍,经常在医师尚未觉察时就已经成为了过去,在其意识清晰以后,对当时体验的追述常能提供有价值的信息,但是也会像对梦境的回忆一样,随着时间的推移而模糊减少。所以,医师的检查方法应灵活机动,不可拘泥。

首先,必须尽可能详尽地了解病史,重点了解容易产生意识障碍的可能原因(是诊断器质性疾病的重要依据),资料的收集面一定要宽,避免遗漏。同时,精神科医师要加强对本专科知识以外的继续医学知识的学习,提高对专科以外疾病的识别和诊疗水平。另外,交谈检查和观察中要注意区别是精神症状还是意识障碍,主要是观察意识的清晰度,确定有无感知觉的削弱和注意、记忆方面的障碍及判定有无定向障碍,意识障碍时的时间定向是最先容易出问题的,但要除外由于妄想、记忆障碍引起的定向障碍;体检也应该要细之又细,尤其是包括软体征在内的神经系统体征要重视;及时进行必要的技术检查,如脑脊液、脑电图、CT、MRI等;还要排除神经科疾病所致的失语和痴呆;在采取医疗处置措施时要慎之又慎;若届时仍然无法确定,就只能是待症状缓解以后,再根据有无事后遗忘来回顾性分析当时有无意识障碍,以此积累经验或接受教训,以利于进一步提高临床实践水平。

\section{精神疾病常见综合征}

精神疾病往往不是以个别零散的精神症状方式表现出来,其中有不少是由某些症状组合成为综合征(症状综合或症群)的形式表现出现。精神病临床上存在许多不同的综合征,有的精神疾病有其特有的综合征,但是同一综合征也可能出现于不同的疾病。例如,情感性精神病具有躁狂状态和抑郁状态,但是这两种综合征在其他精神疾病中也并不少见。组成综合征的各症状之间并不是偶然地、杂乱无章地拼凑在一起,往往具有一定的内部联系或某种意义上的关联性,它们还可以同时或先后地出现和消失。单独一个症状说明的问题很有限,对疾病诊断的价值相对较小,而综合征往往反映了疾病的本质,反映了机体的某些病理、生理变化或病因,对临床确诊具有重要的意义。精神疾病中的综合征,有的是以其组成综合征的症状命名的,有的是以提出某一综合征的人名命名的,有的是这两种形式同时应用,有的还以其他情况来命名。下面介绍一些精神科临床上常见的综合征。

\subsection{幻觉症}

幻觉症(hallucinosis)指在意识清晰时出现大量的幻觉,主要是言语性幻听、幻视以及其他感官的幻觉则较为少见。言语性幻觉常可伴发与其关联的妄想以及恐惧或焦虑的情绪反应,在慢性乙醇中毒性幻觉症的多数病例中,其特点只有幻听而无妄想。这类患者一般并无个性特征改变,并且常可保持其原有的劳动能力。幻觉症可分为急性(一般持续数日)以及慢性(往往持续时间较长,可数月或更长时间)两种。幻觉症最多见于乙醇中毒(慢性)性精神病,也可见于感染和中毒性精神病、反应性精神病及精神分裂症等。

\subsection{幻觉妄想综合征}

幻觉妄想综合征的特点是以幻觉为主,多为幻听、幻嗅等。在幻觉背景上又产生迫害、影响等妄想。妄想一般无系统化倾向。这类综合征的主要特征在于幻觉和妄想之间既密切结合又相互依存、互相影响。这一综合征较多见于精神分裂症,但也见于器质性精神病等其他精神障碍。

\subsection{紧张综合征}

紧张综合征常见于精神分裂症紧张型的一组症状,最突出的症状是患者全身肌张力增高,因而得名。包括紧张性木僵和紧张性兴奋两种状态。

紧张性木僵常有违拗症、刻板言语和刻板动作、模仿言语和模仿动作、蜡样屈曲等症状。紧张性木僵可持续数月或数年,可无任何原因地转入紧张性兴奋状态。此征持续时间短暂,往往突然暴发兴奋激动和暴力行为,然后进入木僵状态或缓解。多数发生于意识清晰状态,少数在梦样意识障碍背景上产生,可伴有周围环境的感知障碍以及幻想性形象。

此外,由Stauder(1934年)首先报道一种紧张综合征。患者表现为严重的紧张性兴奋状态,常伴发热、意识不清、大量幻觉、错觉、奇特动作和危险行为,虽经治疗,往往于数日或1~2周内衰竭死亡,称之为急性致死性紧张症(acute
lethal
catatonia)。多数学者认为,这是精神病患者过度兴奋以致水、电解质紊乱所致,亦有人认为是所用的镇静药物影响下丘脑体温调节中枢之故。以往认为,电抽搐治疗是唯一的救命措施,此种状态近年罕见。

木僵状态根据病因可区分为:①功能性木僵状态:除精神分裂症外,还包括心因性木僵、抑郁性木僵,起病急,由强烈沉重精神刺激或创伤引起。患者虽不活动、不言语,但仍与周围环境保持一定联系,患者的眼神或视线与周围人可能保持某种联系,视线随周围人的体位或活动改变而移动。②器质性木僵:指中枢神经系统器质性病变所致的木僵状态,例如大脑基底动脉、大脑前动脉血栓,脑干损伤,煤气中毒等。在高热患者可引起急性致死性紧张症。

\subsection{遗忘综合征}

Kopcako{B} (1887年)提出了遗忘综合征(amnestic
syndrome),后被定名为柯萨可夫综合征。它的临床特点是识记能力障碍,时间定向力障碍,虚构症和顺行性或逆行性遗忘症。

患者开始时是对其发病后的事件,或刚做过的事情都不能回忆。如饭后不久,他就记不得吃了些什么。遗忘综合征常可与记忆错误结合在一起,患者常以错构或虚构的方式去填补既往经历中记忆脱失的空白部分。

\subsection{精神自动综合征}

俄国精神科医生康金斯基(В. X.
Кандинский)在1886年论述过假性幻觉,巴西学者Clérambault在1920年提出精神自动征这一术语,是在意识清晰状态下产生的一组症状,表现为假性幻觉、强制性思维、思维鸣响、附体感、被揭露感和被控制感,被称为Clérambault综合征或Кандинский-Clérambault综合征。此综合征有如下特点:①脱离了自己意识与意识控制的精神现象和行动;②患者认为这些精神现象与行动的出现是外界强加的,是与他自己的人格不相符的,丧失属于自己的特性,感到是外力作用影响所致;③开始仅有思维云集、强制性回忆、思维中断等,后来才有所谓异己的言语幻听的出现。

此综合征多见于精神分裂症偏执型,也可见于感染性、中毒性精神障碍及乙醇、药物依赖和精神分裂症,如精神自动综合征占主要地位且比较固定时,则预后常较差。

\subsection{Capgras综合征}

Capgras综合征又名易人综合征,由法国精神病学家Jean-Marie Joseph
Capgras(1873-1950年)在1923年提出的。患者通常在意识清楚情况下,认为其周围的某个非常熟悉的人是其他人的化身。多数患者是中年妇女,认为她的丈夫是他人冒充的。源出于希腊神话宙斯神(God
Zeas)欲诱奸Amphitryon之妻,Alcmene乃使自己化身为Amphitryon,使其仆人化身为Amphitryon之仆人Solias。当患者认为其配偶为他人化身时,称为“Amphitryon”错觉,认为周围其他人也是亲人改扮时,称为“Solias”错觉。其实,这类情况的发生并非感知障碍,多数患者认为其配偶或周围人的外形并无改变或仅稍有改变,亦可认为患者本人也是一个替身或其身体被伪造物体所代替。实际上是一类偏执性妄想。替身错觉一词并不恰当。在Capgras综合征中的人物外表并没有改变,感知也没有出现失误。患者承认替身与原形并无区别,病因在患者,尤其是在患者的思维和情感方面。因此,准确地说是替身妄想,而非替身错觉。例如,一患者认为不仅家人已被别人替换,而且自己也被另一个年轻人代替,姓名、年龄、身份都不同,只是外貌一样。患者把自己的大部分衣服扔进垃圾箱,称这些衣服不是她自己的,而是复制品。甚至一英国患者诉说他所居住的小镇被替换成亚洲的某个地方。此征多见于精神分裂症、情感性精神病、更年期精神病、器质性精神病,偶见于意识模糊、癫痫
或癔病患者。

\subsection{Ganser综合征}

1898年德国精神病学家Ganser报道了一组病例,当患者在回答提问时,出现一类奇特的朦胧(twilight)状态,作者认为属于癔病性质。该状态以“近似回答”为主要特征。后人将这类患者称为刚塞综合征(Ganser
syndrome)。它的主要临床特征为近似回答、意识蒙眬及事后遗忘。此征发生前往往有明显精神创伤,多见于监禁状态下的囚犯,以男性为主。目前国际分类归入“其他分离(转换)性障碍”(1CD-10)之中。我国归为瘾症性精神障碍范畴(CCMD2-R)。像大多数癔病性朦胧状态一样,Ganser综合征的持续时间很短,患者一般在数日或数周后,类似症状便自然消失,不需特殊治疗。过去有人认为电休克治疗为有效手段,特别是难以判断是否为诈病时。但用如此剧烈的躯体疗法,也有人觉得难以接受,尚需商榷。也曾有人选用镇静安定药,如异戊巴比妥(阿米妥)、地西泮等,认为对加速患者的好转常能奏效。各种针对性心理治疗理应考虑,但疗效尚无验证。必须强调的是,与对待其他应激情况后发生的分离性障碍一样,必须对好转后的刚塞综合征患者进行跟踪随访。

\subsection{人格解体综合征}

法国医师Krishaber(1873年)最早描述了这一症状;Dugas(1898年)首次应用了人格解体(depersonalization)这一术语;三浦认为这是对自己各种体验缺乏主观感觉。一般说来,人格解体是指对自我和周围现实的一种不真实的感觉而言。对自我的不真实的感觉即指狭义的人格解体(广义者是既包括继发、原发两类,也包括现实解体的内容),它可单独产生面对周围现实的这种感觉,又称非真实感。患者感到自己躯体与周围环境以及其本人产生一种似乎是不真实的疏远的感觉。例如一位患者声称“我的脑子变得不是自己的”、“我的精神和灵魂已不存在于世界上了”;有些患者感到自己丧失了与他人的情感共鸣,不能产生正常的情绪或感觉。人格解体见于正常人的疲劳状态,神经症、抑郁症、精神活性物质或其他躯体、脑器质性因素所致。精神分裂症患者具有人格解体者常伴有被动感,且多变而不固定,而神经症、抑郁症的人格解体则单一且较固定。

流行病学资料表明,约70%的人都发生过短暂的人格解体,且男女性别间无显著差异,是作为个体生活中一种偶然出现被隔离的体验,不具有病理学意义。

大多数患者起病初期为突然出现症状,只有少数患者较缓且逐渐起病。该综合征主要发病年龄为15~30岁,30岁以后起病较少见,很少发生于老年或中年以后。大约50%的成人,在他们经历中都曾有过短暂的人格解体发作;约1/3暂时性发病的人格解体体验者是处于生命受到威胁之时。近40%的患者需住院治疗。

人格解体的病期变异较大,可非常短暂(数秒钟),也可持续数年。慢性病程者可缓解后复发或恶化加重,大多数复发与实际存在的应激事件有关,如战争期间、创伤性事故、暴力犯罪的受害者等等。

随访研究表明,半数以上的患者呈慢性病程,多数患者的症状表现呈稳定病程,但功能活动未受明显影响,而且在多次发作之间可有症状消失的缓解期。患者有时可伴有急性焦虑发作,过度换气表现比较多见。

人格解体综合征的治疗较少引起关注,迄今尚无充分的证据表明药物治疗的特异性,但抗焦虑药对继发焦虑反应是有效的。心理治疗的各种方法未被科学验证过。既然作为神经症的一种特殊类型,有人尝试心理分析疗法或提出“领悟力定向”心理治疗方法,不妨一试,但对能否消除症状不能肯定。必要时可根据患者的人格、人际关系及生活环境等评价,选择不同的适应指征,给予干预。

\subsection{恶性综合征}

恶性综合征是指一组以急骤、高热、意识障碍、肌强直、木僵缄默及多种自主神经症状如大量出汗、心动过速、尿潴留等为主要临床特征的临床综合征。许多精神药物如抗精神病药、抗抑郁药、锂盐等均可引起这种严重的药物不良反应,且死亡率较高,预后不良。

一般急性起病,通常发生在开始用药的1~2周内,90%以上的病例在48小时内出现典型的临床表现,情绪不稳、激动、兴奋、不眠,有腹泻、呕吐、脱水等消化道症状,临床表现以发热、强直、震颤和自主神经系统症状最具特性,持续高热超过39℃,80%以上血白细胞总数增高,中性粒细胞比例增高伴核左移,血中CPK活性升高最明显。急性肾衰竭、血栓塞、肺栓塞、继发感染、弥散性血管内凝血等并发症是致死主要原因。如高度怀疑或已确诊,应立即终止抗精神病药的使用,对症处理和支持治疗是基础,给予抗生素以预防感染,加强护理,防止压疮。硝苯呋海因和溴隐亭认为是该综合征的特效药物。