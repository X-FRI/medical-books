\chapter{心理因素相关生理障碍}

心理因素相关生理障碍(physiological disorders related to psychological
factors)是指一组以心理社会因素为主要病因,临床主要表现为进食、睡眠及性行为异常的生理、行为障碍。与本组障碍发生、发展、病程及预后等密切相关的因素有:①生活事件和处境;②社会文化背景;③人格特点及经历等。心理因素相关生理障碍包括进食障碍、睡眠障碍及性功能障碍等。

\section{进 食 障 碍}

进食障碍(eating
disorder)是指以进食行为异常为主要临床表现,常伴有情绪障碍、显著的体重改变和/或生理功能紊乱的一组综合征,主要包括神经性厌食、神经性贪食及神经性呕吐。本精神障碍多见于青年女性,可单独表现为神经性厌食、神经性贪食及神经性呕吐,也可为上述障碍的混合表现。

\subsection{神经性厌食}

神经性厌食(anorexia
nervosa)是一种患者自己故意限制饮食,使体重降至明显低于正常的标准为特征的进食行为异常。患者为此有意严格限制饮食,并常采取过度运动、引吐、导泻等方法以减轻体重。患者常有对“肥胖”的强烈恐惧以及对体型体重的过度关注,甚至已经明显消瘦仍自认为太胖,即使医生进行解释也无效。部分患者用胃胀不适、食欲下降等理由来解释其限制饮食的行为。患者常有营养不良、代谢和内分泌紊乱,女性可出现闭经,男性可有性功能减退,青春期前的病人性器官呈幼稚型。有的患者可有间歇发作的暴饮暴食。本病最常见于青少年女性,年龄多为13~20岁,极少见于青少年男性。

\subsubsection{病因}

1.社会文化因素 在发病中起着很重要的作用。现代社会文化观念中,存在着“瘦”的文化压力,把女性身体苗条作为体型好、自信、自我约束及成功的代表。大量的媒体信息和营销策略营造出节食促进成功这样的氛围,减肥、追求苗条成为社会时尚,受到公众的推崇,这无疑给女性带来极大的压力。而在某些职业中患病率明显高于普通人群的现象也支持这一观点,如芭蕾舞演员、时装模特患病率高于普通人群4~5倍。

2.家庭心理因素 对本病发病影响的家庭心理因素研究中,不少学者提出了新的理论假设。有的人提出多和家人在一起吃饭,对吃饭高度重视,家中吃饭气氛积极,家庭进餐环境组织良好的青少年患本病的可能性较小。父母和家庭环境影响孩子对食物的态度和孩子对饱足的评价,孩子有问题的进食、体重波动和对食物的关注与父母难以控制孩子进餐有关,这些因素反过来导致个体随后发展成进食障碍。患者的家庭环境常具有内部冲突多、管制严、情感交流少、不同意见不能表达、少娱乐等特点。有的人提出患者以进食行为表示对父母过度控制、过度保护的反抗;或以节食为手段达到对父母的反控制,以此作为解决家庭冲突的一种方法。有人认为该病的发生与青少年的性心理发育不同步有关,患者对日益丰满的身材难以接受,希望拒绝成熟,停留在儿童时期。

3.性格特征 患者性格特征的研究结果虽不完全一致,但仍表明多数患者有自我取向和社会取向于完美主义、不成熟、依赖性强、追求与众不同、自我评价差等特点。

4.生物学因素 遗传学研究显示,家族有抑郁症、酒精依赖、肥胖或进食障碍的人群中,进食障碍发生的危险性明显升高。家系研究发现,患者的一级亲属中该病的发病率高于普通人群,患者先证者母亲和姊妹患该病的危险性在4%左右,终身患病率大约是普通女性人群的8倍。双生子研究显示单卵双生子患病的一致率约为55%,而双卵双生子为5%,表明单卵双生子的同病率明显高于双卵双生子。但许多学者提出,他们共同生活的家庭和社会环境所起的作用也不容忽视。下丘脑功能的异常也是人们研究的主要问题,主要表现为下丘脑-垂体-肾上腺轴亢进和下丘脑-垂体-性腺功能低下,有1/4左右的神经性厌食患者体重下降前出现闭经。更多的研究认为内分泌和代谢的异常可能继发于营养不良。

5.神经生化因素 研究发现患者可有血中生长激素、皮质激素水平升高以及血中雌二醇水平、性腺激素水平降低等。甲状腺功能检查通常表现为正常甲状腺功能病态综合征,T3、T4水平可减低,但促甲状腺激素通常是正常的或有轻度降低。

\subsubsection{流行病学}

90%以上的神经性厌食症患者是女性,女性的终生患病率为0.5%~1.0%,其发病的两个高峰年龄为12~15岁和17~21岁,在青春期前和40岁以后很少出现。近年来,15~24岁的女性进食障碍的发病率似乎有增高的趋势,在西方,神经性厌食已成为15~24岁女性中仅次于肥胖和哮喘的最常见疾病,在我国患病率也呈上升趋势。

\subsubsection{临床表现}

患者的核心症状是恐怖“肥胖”和对体型、体重的过度关注。多数患者存在体像障碍,即使已经骨瘦如柴仍认为自己胖。患者对进食持有特殊的态度和行为,有意限制进食。有些人对食谱有严格的挑选,进食时速度缓慢,常将食物分为小块送入口中细嚼慢咽,以充分享受美味的快感。患者为使体重减少或避免体重增加,常采用过度运动、服药、自我诱吐等行为。患者的运动强度多与体力极不相称,使人感到似在自我折磨、自我惩罚。有的患者可有间歇发作的暴饮暴食,产生暴食之后再去诱吐。有些患者服用泻药、减肥药、利尿剂或食欲抑制剂以减轻体重。患者常有营养不良、代谢和内分泌紊乱。女性可出现闭经或初潮不来,男性可有性功能减退,青春期前的患者性器官常呈幼稚型。

精神症状:患者常有焦虑、抑郁、恐惧及冷漠、亢奋等,情感稳定性较差,可有敏感、多疑及认为自己过胖的超价观念。患者对身体的感知觉存在体像障碍,在病情严重进入衰竭状态时还可能出现幻觉、错觉等。有部分患者可有自我认同障碍、过分自尊等;有一些患者可能会出现对未来失去信心,或反常的内疚和自责等;可有社会性隔离,常拒绝与家人共同进食,对食物进行高度选择,拒绝进食碳水化合物及高脂肪食物,进食的主要物质多为水果和蔬菜等,有时甚至不吃;可出现意志亢进,为了节食和控制体重,采取过度活动等;对消瘦及异常进食行为缺乏自知力,拒绝就医及接受治疗。

体格及实验室检查:患者可有地塞米松抑制试验阳性、低体温、低血压、低血糖、心动过缓、水电解质紊乱、血细胞减少等;可能存在心电图异常及骨质疏松等表现;脑CT检查发现,患者在长期饥饿时,脑脊液间隙扩大(脑沟和脑室扩大);神经影像研究发现,患者的额叶和顶叶皮层代谢和灌注降低。这些改变随着体重的回升而恢复正常。

\subsubsection{诊断}

1.CCMD-3的诊断标准

(1)明显体重减轻,比正常平均体重减轻15%以上,或者Quetelet体重指数为17.5或更低,或在青春期前不能达到所期望的躯体增长标准,并有发育延迟或停止。

(2)自己故意造成体重减轻,至少有下列1项:①回避“导致发胖的食物”;②自我诱发呕吐;③自我引发排便;④过度运动;⑤服用厌食药或利尿药等。

(3)常可有病理性怕胖:指一种持续存在的异乎寻常的害怕发胖的超价观念,并且患者给自己制定一个过低的体重界限,这个界值远远低于其病前医生认为是适度的或健康的体重。

(4)常有下丘脑垂体性腺轴的广泛内分泌紊乱。女性表现为闭经(停经至少已3个连续月经周期,但妇女如用激素替代治疗可出现持续阴道出血,最常见的是使用避孕药),男性表现为性欲丧失或性功能低下。可有生长激素升高,皮质醇浓度上升,外周甲状腺素代谢异常及胰岛素分泌异常。

(5)症状至少已3个月。

(6)可有间歇发作的暴饮暴食(此时只诊断为神经性厌食)。

(7)排除躯体疾病所致的体重减轻(如脑瘤、肠道疾病例如克隆病或吸收不良综合征等)。

说明:

①正常体重期望值可用身高厘米数减105,得出正常平均体重千克数;或用Quetelet体重指数=体重千克数/身高米数的平方进行评估。

②有时厌食症可继发于抑郁症或强迫症,导致诊断困难或在必要时需并列诊断。

2.临床类型 并非所有的神经性厌食患者均能完全符合CCMD-3的诊断标准。ICD-10还另外列了一项非典型神经性厌食的临床类型,表现为缺乏神经性厌食的一个或多个关键特征,如闭经或显著的体重下降,但除此之外却表现出相当典型的临床相;亦可为存在全部特征症状但程度较轻的患者。可在综合性医院的精神科会诊中或社区医疗保健工作中见到这类患者。

\subsubsection{鉴别诊断}

神经性厌食需要与躯体疾病所致的体重减轻及其他精神障碍的继发或伴发症状鉴别。需充分排除器质性疾病、多种精神障碍如抑郁症等症状后,才能作出诊断。本病也有可能与抑郁、强迫或人格障碍形成共病,诊断时可根据病情特点同时给予若干诊断。

1.脑器质性疾病 垂体肿瘤可能表现为异常厌食,而对肥胖特别在意的青少年女性,症状可能更加明显。但是,脑器质性疾病患者可有其他症状,如头痛、头昏以及脑神经症状,头颅CT、MRI等检查有助于鉴别。有一些垂体微腺瘤可能缺乏主观症状,仅表现为厌食,诊断时需特别小心。有条件的应把头颅CT、MRI等检查作为常规检查,以避免误诊误治。

2.躯体性疾病 青年人躯体因素包括慢性消耗性疾病,肠道疾患,如克隆病或吸收不良综合征等所致的体重下降。这些疾病虽然同样也有厌食症状,但缺乏神经性厌食的其他表现,如自己故意造成体重减轻、害怕发胖的超价观念及对身体的体象障碍等。此外,各种检查也有助于鉴别。

3.抑郁症 厌食是抑郁症的一个常见的伴随症状,同时,神经性厌食可有焦虑、抑郁、恐惧及冷漠、亢奋、情感稳定性较差等情绪症状,因此,在诊断神经性厌食时需与抑郁症充分地鉴别。一般情况下,抑郁症还同时具有发作性病程、早醒、晨重夕轻、自责、自我评价下降等特点,鉴别诊断不是非常困难。但是,不典型表现的神经性厌食和抑郁症的鉴别诊断困难,有时甚至难以鉴别。

4.神经性贪食症 体重在正常范围或超重,通过抑郁心境、自我反思引起发作性暴食,常伴有自发引吐。本病常与厌食症同时存在。

5.与其他精神障碍共病 由于神经性厌食可有焦虑、抑郁、恐惧及冷漠、亢奋、情感稳定性较差等表现,根据CCMD-3及目前的共病理论,神经性厌食可与抑郁、强迫症或人格障碍等形成共病,诊断时可根据病情特点同时给予若干个诊断。

6.正常节食 是通过节制饮食以达到身材苗条和减轻体重的目的,但无体象障碍和内分泌紊乱,一旦达到理想体重,便能适可而止。

\subsubsection{治疗}

1.治疗原则 多数病人以门诊治疗为主,当出现严重的营养不良、恶病质或有严重的自伤、自杀行为时,需强制住院治疗。治疗原则是首先纠正营养不良,同时或稍后开展心理治疗以及药物治疗来增加饮食,逐渐恢复患者的正常体重和正常的生活。治疗方法以心理治疗为主,必要时辅以小剂量抗焦虑药、抗抑郁药物或抗精神病药治疗。

2.治疗方案

(1)营养支持治疗:首先加强营养,增加体重,恢复身体健康,帮助患者恢复正常的饮食习惯。体重太轻,明显营养不良者,应给予高热量饮食。呕吐、拒食者应考虑通过静脉予以肠外营养治疗。根据患者病情及胃肠道的适应过程,每种营养液或膳食都从低浓度、少量开始,逐渐增量,使摄食量与体重稳步增加。为了保证治疗计划的贯彻执行,需要适当地帮助患者自我监督并遵守治疗计划。

(2)心理治疗:心理治疗的主要目的在于改善与他人的关系和提高自我效能。治疗方法包括家庭治疗、认知行为治疗及小组治疗等。家庭治疗是以“整个家庭”为干预对象的治疗形式,全面了解家庭中的互动模式、成员关系和情感表达等,并针对不同的问题和家庭情况施以不同的干预手段,以解决家庭和个体的冲突。家庭治疗并不是一种单一的治疗手段,而是一个兼容并蓄的体系。认知行为治疗的目的一方面在于改变患者对体型、体重及进食的态度和行为;另一方面对患者改变对自己及与他人的关系有帮助。小组治疗及患者自助小组治疗也有一定的疗效。

(3)药物治疗:药物治疗不能明显增加患者的体重或改善患者的病理心理,因此药物治疗常与心理治疗联合应用,尤其在改变患者的进食态度和行为上必须配合心理干预,至今没有一种公认的特效药物:①抗精神病药:对顽固抵抗体重增加、存在严重强迫思维或表现出妄想性否认的患者可使用第二代抗精神病药(特别是奥氮平、利培酮、喹硫平);②碳酸锂:碳酸锂联合行为治疗3~4周,比安慰剂有效;③抗抑郁药:SSRIs类药物可提高食欲、增加体重,缓解抑郁、焦虑、失眠等精神异常,恢复正常的生理功能,体重恢复后减少复发。

如果是共病,则应按要求正规治疗。

(4)其他治疗:由于患者常常存在明显的消瘦,可能存在负氮平衡及电解质紊乱等,需加强支持性治疗。对于有骨质疏松、闭经者进行对症治疗。

\subsubsection{预后}

本病常为慢性迁延性病程,缓解和复发可周期性交替。国外的随访发现,大约45%的患者可以痊愈;大约30%的患者躯体情况有改善,仍遗留进食和心理问题;大约25%的患者预后不佳,并且几乎不能恢复到正常体重;其中5%~10%的患者死于并发症。预后与以下因素密切相关:①社会文化;②生活事件和处境;③人格特点、应对方式及经历;④继发性营养不良,内分泌、代谢及躯体功能紊乱等。与预后良好的相关因素有:发病年龄小、病程短、不隐瞒症状、不幼稚,对自己评价发生改变。预后不良者多是:父母矛盾突出,有暴食、诱吐、服泻药的情况,有行为的异常,如强迫症状、癔症、抑郁等。

\subsection{神经性贪食}

神经性贪食(bulimia
nervosa)是一种以反复发作、不可控制、冲动性的摄食欲望及暴食行为为特征的进食障碍。患者有担心发胖的恐惧心理,常采取自我诱吐、使用泻剂或利尿剂、禁食、过度锻炼等方法以消除暴食引起发胖的极端措施。可与神经性厌食交替出现,两者具有相似的病理心理机制,性别、年龄分布相似。多数患者是神经性厌食的延续者,发病年龄较神经性厌食晚。本症并非神经系统器质性病变所致的暴食,也不是癫痫
、精神分裂症等精神障碍继发的暴食。

\subsubsection{病因}

1.生物学因素 临床和实验证据表明体重大的女性患本症的可能性大。家族研究发现患者的一级亲属情感障碍发生率也高,情感不稳定可能是一个生物遗传因素,同卵双生子的本症并发率为22.9%,超过一般总体的危险性8倍以上。患者可能存在5-羟色胺和肾上腺素能系统失衡,即下丘脑去甲肾上腺素系统功能亢进,或下丘脑5-羟色胺系统功能低下,或两者均发生异常,而导致发作性暴食行为。

2.社会文化和家庭因素 随着社会愈来愈移向苗条的价值标准,再加上传媒的影响,追求健康时尚的兴起,女性角色的转移,女性的贪食症问题在当代日趋严重。家庭交互作用的研究发现患者家庭中的冲突、被抛弃、被忽视、敌意以及忽视情感需要的情况更严重。母亲的节食行为,尤其是对女儿体重、外貌的过度关注,是女儿进食发生障碍的显著危险因素。

3.心理学因素 早期的创伤,如性或躯体虐待是神经性贪食的一个危险因素。患者的青春期适应有困难,处理心理冲突的能力差,比神经性厌食的患者更易冲动、易怒;害怕离开家庭;焦虑、抑郁发生率高,自杀的危险性更高;心理动力学研究提出,患者将与母亲分离的冲突带入对食物的摄取矛盾中。患者多为完美主义者,追求体型的苗条,追求成就感。

\subsubsection{流行病学调查}

从20世纪80年代起,神经性贪食症的患病率逐渐上升,在年轻女性中患病率为1%~3%,有5%~10%或更多的年轻女性有部分症状。发病以女性为主,占98%~100%,起病平均年龄为18~20岁,大多数起病3~5年后,才转诊到精神科就诊。

\subsubsection{临床表现}

患者常常出现反复发作,一次进食大量食物,吃得又多又快,故称为暴食。患者的一次进食量远远超过正常人,可达到正常量的数倍之多,如某患者一次吃进1~1.5kg(2~3斤)馒头和两碗菜;另有一患者将其母亲为6口人准备的饭菜全部吃光。多数患者喜欢选择食用平时严格控制的“发胖”食物,如蛋糕、面食等,患者有不能控制的饮食感觉,自己明知不对却无法控制。一旦开始吃,患者很难主动停止,常常吃到难受为止。患者往往过分关注自己的体重和体型,存在担心发胖的恐惧心理。在发作期间,为避免因暴食带来的体重增加常反复采用不适当的代偿行为如:自己诱发呕吐、滥用泻药、间歇进食、使用厌食剂等。严重的患者常常边吃边吐,可以持续数小时,直到累得筋疲力尽才罢休。患者的暴食行为常常是偷偷进行,在公共场合尽量不吃或少吃,但常为此痛苦不堪而回避他人。因需要大量的食物,患者常常有骗钱和偷窃行为。

情绪障碍比神经性厌食的患者更突出,情绪波动性大,易产生不良情绪,如愤怒、焦虑不安、抑郁、孤独感等。患者常用暴食排解不良情绪,但很快就被自我放纵导致的内疚感、食物会引起发胖的恐惧感和腹部胀满的痛苦感包围,常常在诱吐后情绪才能平静下来。患者常为自己的行为而懊恼、自责,并感到无奈。

神经性贪食症和其他精神障碍的共病率很高,24%~88%的患者伴有心境障碍,12%的患者为双相障碍,伴随人格障碍使患者的症状更加复杂化。神经性贪食症患者也常伴有分离性症状、性冲突、性障碍及挥霍、偷窃、自残等冲动行为。

尽管许多患者在寻求治疗之前,贪食行为已持续数年,但最常见、首发的并发症为龋齿和胃肠道出血等体征。神经性贪食患者的体重可以是正常的。随着病情的进展,代谢紊乱逐渐明显,50%的患者有脱水征象及电解质紊乱,25%的患者有代谢性碱中毒及低氯血症,14%的患者有明显低血钾、低血钠。患者由于有反复贪食、引吐、导泻的行为,严重地影响了社会功能。极少数患者还可能因食管、胃肠道、心脏等并发症而出现生命危险。

\subsubsection{诊断}

1.CCMD-3的诊断标准为

(1)患者存在一种持续的、难以控制的进食和渴求食物的优势观念,并且患者屈从于短时间内摄入大量食物的贪食发作。

(2)至少用下列一种方法抵消食物的发胖作用:①自我诱发呕吐;②滥用泻药;③间歇禁食;④使用厌食剂、甲状腺素类制剂或利尿药。如果是糖尿病患者,可能会放弃胰岛素治疗。

(3)常有病理性怕胖。

(4)常有神经性厌食既往史,两者间隔数月至数年不等。

(5)发作性暴食至少每周2次,持续3个月。

(6)排除神经系统器质性病变所致的暴食,及癫痫
、精神分裂症等精神障碍继发的暴食。

说明:有时本症可继发于抑郁症,导致诊断困难或在必要时需并列诊断。

2.临床类型与神经性厌食一样,并非所有的神经性贪食患者均完全符合CCMD-3的诊断标准。DSM-4另外还列了一项非典型神经性贪食的临床类型,表现为体重正常甚至超重,却伴有暴食后呕吐或导泻的典型周期的一类患者。

\subsubsection{鉴别诊断}

1.神经性厌食 神经性贪食与神经性厌食的诊断主要是侧重不同。前者强调贪食的频度(每周2次以上)和持续的时间(3个月以上),而后者则重视体重减轻的程度(比正常体重减轻15%以上)和节食引起的内分泌失调(女性闭经,男性性功能减退)。如已明确诊断为神经性厌食或交替出现的经常性厌食与间歇性暴食症状者,均应诊断为神经性厌食症。

2.颞叶癫痫
 可出现暴食行为,病史、体检以及脑电图、CT等各项实验室检查可发现有器质性病变基础。同时患者还可有抽搐史或精神自动症,而且这类患者缺乏控制体重的不恰当行为。

3.Kleine-Levin综合征 又称周期性嗜睡贪食综合征,男性多见,表现为发作性嗜睡(不分日夜)和贪食,持续数天。患者醒了就大吃,吃了就睡,可有定向障碍、躁狂样冲动。患者无催吐、导泻等控制体重行为,亦无对身体外形或体重不满的表现。

\subsubsection{治疗}

1.首先,治疗干预的目标是营养状况的恢复及正常进食行为模式的重建,打破由于营养不良引起的躯体和心理后遗症的影响以及所形成的持续进食障碍行为模式的恶性循环。远期目标是寻找和帮助解决与贪食有关的心理、家庭、社会问题,以预防复发。治疗方法以心理治疗为主,必要时辅以小剂量抗抑郁药物或抗精神病药物治疗。

2.治疗方案

(1)恢复正常进食规律:同神经性厌食一样,需要适当的监督和管理。

(2)心理治疗:认知行为治疗是最主要的治疗方法,主要包括认知重建,纠正歪曲的信念和贪食行为,自我监督,解决问题的技能和放松训练等,认知行为治疗能够减轻贪食行为,改善抑郁情绪,改变病人对体型、体重、减肥及企图通过诱吐方式控制体形与体重的不恰当看法,对巩固疗效、预防复发有一定意义。小组治疗及患者自助小组也有一定的疗效。个别治疗及家庭治疗亦可采用。

(3)药物治疗:双盲对照研究表明抗抑郁药治疗神经性贪食症有效。氟西汀是第一个报道治疗神经性贪食的抗抑郁药,它能减少贪食症状,改善焦虑及抑郁心境。三环类抗抑郁药也能改善患者的症状,但由于存在镇静等其他副作用,已较少使用。氟哌啶醇对控制患者的暴食行为有明显的疗效。

如果是共病,则应按要求正规治疗。

(4)躯体支持治疗:由于患者可能存在负氮平衡及电解质紊乱等内科并发症,需加强支持性治疗及相应的对症处理。

\subsubsection{预后}

本病呈慢性病程,多数患者有神经性厌食的病史,症状可迁延数年。未经治疗患者中,1~2年后25%~30%的症状如发作性进食、导泻剂滥用等,可自行缓解。在住院和门诊治疗的病人随访研究中发现,2年后的缓解率分别为38%和46%。在无电解质紊乱或代谢低下等严重内科并发症时,对患者的生命没有严重伤害。

\subsection{神经性呕吐}

神经性呕吐(psychogenic
vomiting)又称心因性呕吐,是一种以自发或故意诱发反复呕吐为特征的精神障碍。呕吐物为刚吃进的食物,不伴其他的明显症状,呕吐常与心理社会因素有关。患者可有害怕发胖和减轻体重的想法,但体重无明显减轻,无器质性病变。本病女性多于男性,常常发生于成年早期。

患者表现为进食后在无明显的恶心或其他不适的情况下,突然出现呕吐,但呕吐不影响下次进食的食欲,一段时间内反复发作。呕吐常与心情不愉快、心理紧张、内心冲突等心理社会因素有关,以后可在类似情况下反复发作。部分患者常具有自我中心、易受暗示、易感情用事、好夸张、做作等癔症样个性特征。

CCMD-3的诊断标准为:

1.自发的或故意诱发的反复发生于进食后的呕吐,呕吐物为刚吃进的食物。

2.体重减轻不显著(体重保持在正常平均体重值的80%以上)。

3.可有害怕发胖或减轻体重的想法。

4.呕吐几乎每天发生,至少已持续1个月。

5.排除躯体疾病导致的呕吐,以及癔症或神经症等。

治疗可用小剂量抗焦虑药、抗抑郁药(氟西汀)、抗精神病药(舒必利)以及行为治疗等。

\section{睡 眠 障 碍}

睡眠障碍(sleep
disorders)是指各种心理社会因素等引起的睡眠与觉醒障碍,包括睡眠的发动和维持困难(失眠)、白天过度睡眠(嗜睡)、24小时睡眠-觉醒周期紊乱(睡眠-觉醒节律障碍)和某些发作性睡眠异常情况(睡行症、夜惊、梦魇等)。

\subsection{失眠症}

失眠症(insomnia)是指睡眠的始发和维持发生障碍,导致睡眠的质量处于相当长时间的不满意状况,其他症状均继发于失眠。失眠的表现有多种形式,包括难以入睡、睡眠不深、易醒、多梦、早醒、醒后不易再睡、醒后不适感、疲乏,或白天困倦。失眠可引起患者焦虑、抑郁或恐惧心理,并导致精神活动效率下降,妨碍社会功能。失眠的焦虑和恐惧心理可形成恶性循环,从而使症状持续存在。一般人群中失眠症的患病率为10%~20%,男、女差别不大。

不能把一般认为的正常睡眠时间作为诊断失眠症的主要标准,因为有些人只需要很短时间的睡眠就感到全身舒适、头脑清晰、精力充沛,并且不认为自己是失眠者。如果有失眠申诉,但自身感觉良好、精力充沛,不能诊断为失眠症。

\subsubsection{病因}

失眠是精神科常见的症状,原因很多,常见原因有:

1.心理因素 各种因素引起的焦虑、紧张、恐惧、悲观、抑郁、思虑过度等。

2.社会因素 社会适应能力差,人际关系紧张,个人工作、生活方面的挫折和失败,如工作方面不顺利、单位不景气、面临下岗,学习方面的困难、考试成绩欠佳、考前焦虑,日常生活中的失恋、婚姻困扰、亲人亡故、个人损失等,均可造成心理问题,引起失眠。

3.家庭环境 因素家庭关系紊乱、家庭不和、家庭成员酗酒、亲子关系不佳等。

4.家庭中睡眠环境 灯光太亮、噪音、震动、卧室温度不良及环境变迁难以适应、吸血昆虫的骚扰,不舒服的床铺、被褥;对睡眠环境缺乏安全感,如对自然灾害的恐惧、害怕鬼神等。

5.睡眠节律紊乱 昼夜轮班,时差反应、经常熬夜等。

6.物质或药物因素 睡前饮用浓茶、咖啡、兴奋剂及滥用安眠药物等。

7.躯体因素 躯体不适、过饥、过饱、疼痛、慢性躯体疾病等。

8.对睡眠持有不正确的态度 如认为每晚必须有8小时睡眠,当自己睡眠不足时,便担心睡眠不够。失眠与个体的性格特征、既往经历、对失眠的易感性和应付能力,以及身体状况均有关。另外,失眠与媒体的一些不正确的宣传也有关。

9.其他因素 继发于某些精神障碍,如各类神经症、情感性精神障碍、精神分裂症、躯体及器质性精神障碍等。

\subsubsection{临床表现}

失眠是最早被认为也是患病率最高的睡眠疾病。失眠的临床表现有各种形式,难以入睡是最常见的主诉,其次是维持睡眠困难和早醒,还有多梦、睡眠不深、醒后不易再睡、醒后感到疲乏或缺乏清醒感、白天思睡,患者往往为上述症状的混合表现。在其最新定义中,特别强调两点,首先患者必须有白天不适的主诉,但无法通过实验室检查或睡眠时间的长短来衡量;其次,在适合的环境中及充足睡眠时间后症状仍不缓解。失眠分为原发性失眠和继发性失眠两大类,原发性失眠约占失眠病人的15%,可能与社会、环境及个体因素有关。继发性失眠占失眠病人的绝大多数,其原因主要是:①任何影响中枢神经系统的躯体疾病,以及所造成的痛苦和不适。②酒精、咖啡、浓茶或药物等使中枢神经系统兴奋性增高。③精神疾病,特别是焦虑症和抑郁症几乎均有失眠。④在当今社会,药物或毒品成癮占慢性失眠的15%左右。失眠者常试图用服药或饮酒来应付自己的紧张情绪,失眠往往引起患者白天不同程度地感到未能充分休息和恢复精力,因而躯体困乏,精神萎靡,注意力减退,思考困难,反应迟钝。患者如果反复失眠,就会对失眠越来越恐惧,并担心其后果,久而久之,就形成了恶性循环,使得失眠者的问题持续存在,甚至越来越严重。

\subsubsection{诊断}

临床上失眠症的诊断具有相当大的挑战性,详细的病史、个人史、生活史、相关疾病史、药物史、精神状态评估、人格特质及行为模式评估、躯体及神经系统检查、适当的实验室检查都是用以排除其他因素造成的失眠状态的条件。

主观指标:①主诉睡眠生理功能障碍;②白天疲劳、头胀、头昏;③仅有睡眠量减少而无白天不适,不视为失眠。

客观指标:①睡眠潜伏期延长(30分钟以上);②实际睡眠时间减少(不足6小时);③觉醒时间增多(每夜30分钟以上)。

CCMD-3诊断标准为:

1.症状标准

(1)几乎以失眠为唯一的症状,包括难以入睡、睡眠不深、多梦、早醒,或醒后不易再睡,醒后不适感、疲乏,或白天困倦等。

(2)具有失眠和极度关注失眠结果的优势观念。

2.严重标准 对睡眠数量、质量的不满引起明显的苦恼或社会功能受损。

3.病程标准 至少每周发生3次,且至少已有1个月。

4.排除标准 排除躯体疾病或精神障碍症状导致的继发性失眠。

说明:如果失眠是某种躯体疾病或精神障碍(如神经衰弱、抑郁症)症状的一个组成部分,不另诊断为失眠症。

\subsubsection{鉴别诊断}

1.精神症状所致的失眠精神紧张、焦虑、恐惧、兴奋等可引起短暂失眠,主要为入眠困难及易惊醒,精神因素解除后失眠即可改善。许多精神障碍的患者常常有失眠的表现,易误诊为失眠症。如神经衰弱患者常诉说入睡困难,睡眠不深、多梦,但脑电图记录上显示睡眠时间并不减少,而觉醒的时间和次数有所增加。这类患者除失眠外,常有头痛、头晕、健忘、乏力、易兴奋、易衰弱、易激动等症状。抑郁症的失眠多表现为早醒或睡眠不深,脑电图描记显示觉醒时间明显延长,需要详细了解抑郁症状加以鉴别。躁狂症表现为入睡困难,甚至整夜不眠,此外还有躁狂症状。精神分裂症因受妄想影响可表现入睡困难、睡眠不深,此外有精神病性症状。详细地了解病史,细致的精神检查将有利于鉴别诊断。

2.躯体因素引起的失眠 各种躯体疾病引起的疼痛、瘙痒、鼻塞、呼吸困难、气喘、咳嗽、尿频、恶心、呕吐、腹胀、腹泻、心悸等均可引起入睡困难和睡眠不深。

3.大脑弥散性病变引起的失眠 慢性中毒、内分泌疾病、营养代谢障碍、脑动脉硬化等各种因素可引起大脑弥散性病变。失眠常为其早期症状,表现为睡眠时间减少、间断易醒、深睡期消失,病情加重时可出现嗜睡及意识障碍。

\subsubsection{治疗}

治疗原则为消除干扰因素,恢复睡眠节律;查找引发病因、疾病;采用心理疗法或行为矫治疗法;综合医疗仪器诊治和中西医药疗法。

1.一般治疗 进行适当的睡眠卫生宣教,应给患者讲解正确的睡眠卫生知识,如养成规律的作息时间。养成适当的运动习惯,避免睡前从事除了性爱以外的剧烈运动。很重要的一点是不要把情绪带到床上。对于失眠的病人应尽可能提供一个良好的睡眠环境,如:布置好卧室,选用软硬适度的床和床上用品;睡眠环境安全、舒适、安静和令人放松;避免外界的光线、噪音、气味等刺激;保持室内适当的温度和湿度等。减少使用酒精、咖啡、浓茶等刺激性物质及一些中枢神经兴奋药物(如左旋多巴类药物和外源性皮质类固醇激素等)。调整睡眠周期,睡眠时间的长短存在个体差异,有些人需要的睡眠时间就比别人少,特别是老年人。培养良好的睡眠习惯,失眠病人应努力恢复正常的睡眠与觉醒周期,废除过多的白天小睡。

2.心理及行为治疗 帮助患者理解睡眠是一种自然的生理过程,消除对失眠的焦虑和恐惧。心理治疗着重于让患者了解自我内心冲突与失眠的关系而加以修正。不少患者对睡眠有较高期望,他们过分关注自己的睡眠,夸大地认为自己睡眠时间严重不足,致使脑力、体力无法充分恢复。施行认知疗法时,帮助患者对失眠引起的症状及苦恼有一个客观的正确的理解和认识,以减少消极情绪。行为治疗的技巧包括渐进性肌肉松弛法、自我催眠法、超觉静坐及生物反馈法等。行为训练包括:刺激控制、放松训练、暗示及自我暗示、认知、支持、森田疗法。用这些方法可使心情平静、身体及肌肉松弛而改善睡眠。

3.药物治疗 经过一般性治疗和心理治疗,仍不能改善失眠症状者,需要采取药物治疗。理想的催眠药物应具有下列特点:吸收快,加速入眠,改善异常的睡眠时相,影响正常的生理睡眠时相轻,作用短,体内消除快,无蓄积作用,清醒后无药物延续作用。事实上,现在的可用药物没有一种完全符合上述理想的安眠药物。一旦临床上评估有药物治疗必要时,必须根据患者的年龄、身体功能状况、从事的工作及睡眠困扰的形态给予适当剂量的药物,包括中药和西药。药物治疗失眠有5个原则:①使用最低有效剂量;②间断给药(如每周2~4次);③短期服药(连续服药不超过3~4周);④逐渐停药,特别是半衰期较短的药物,停药更要缓慢,且要因人而异;⑤注意停药后的失眠反弹,减药要慢。常用治疗失眠的药物有:苯二氮䓬
类(地西泮、阿普唑仑等)、非苯二氮䓬
类(佐匹克隆、唑吡坦、扎来普隆)、抗抑郁药(曲唑酮、米氮平等)、褪黑素等。此类药物的主要药理作用是普遍降低大脑皮层的兴奋性,特别是降低边缘系统的兴奋性,使脑的氧化代谢普遍减慢,突触的传递过程受到抑制,减少自发活动,松弛骨骼肌,产生镇静催眠和抗焦虑作用,促进睡眠。

苯二氮䓬
类药物的催眠作用在首次应用时较明显,连续使用2周后就会产生药效学耐受,而无法维持相同的疗效。更突出的问题是,无论是高剂量还是治疗量的苯二氮䓬
类药物的连续应用均可产生依赖性。当骤停苯二氮䓬
类药物时会出现紧张焦虑、激动不安、反跳性失眠、知觉过敏(畏光、听觉过敏、痛觉过敏)和全身症状(如厌食、不适和体重减轻)等撤药症状。

非苯二氮䓬
类药物的镇静催眠作用与苯二氮䓬
类药物相似,但依赖性和戒断症状较轻,被认为是苯二氮䓬
类药物的合适替代品。这类药物具有入睡快、延长睡眠时间,明显增加深睡眠,基本不改变正常睡眠生理结构,醒后无宿醉感,不易产生耐药性和依赖性等特点。

曲唑酮、米氮平在治疗抑郁、焦虑的同时帮助恢复正常睡眠结构,无需联合苯二氮䓬
类药物,避免了苯二氮䓬
类药物依赖的危险,是较满意的镇静性抗抑郁药。

褪黑素在调节动物的昼夜节律和季节节律以及机体睡眠觉醒节律方面具有重要作用,可用来治疗由于生理节律紊乱引起的周期性失眠,如飞行时差、轮班工作等。

4.非药物治疗如音乐疗法、气功疗法、体育疗法、物理疗法,以及生物反馈,如水疗、光疗、电治疗、磁疗、针灸、按摩等。

\subsubsection{预后}

失眠是一种常见的现象,原因很多,既可是一种症状,也可是一种疾病。养成良好的睡眠习惯,消除紧张,保持放松,必要时去看医生,经过综合治疗,相信大多数患者失眠可望缓解。不论失眠及伴随症状多么严重,一般地说,失眠只是大脑的兴奋和抑制功能暂时失去平衡的结果。尽管失眠也常常是某些疾病的伴随症状,但失眠本身并不能反映身体内部有什么器质性病变,更不会转变为精神病或其他疾病。只要认真找出失眠症的原因,针对病因进行适当的锻炼和休养,再配合必要的中西药物,失眠是可以消除的。

\subsection{嗜睡症}

嗜睡症(hypersomnia)是指白天睡眠过多,不是由于睡眠不足、药物、酒精、躯体疾病所致,也不是某种精神障碍(如神经衰弱、抑郁症)症状的一部分,本症并非常见,目前缺乏相应患病率报道,病因不清。患者表现为在安静或单调环境下,经常困乏思睡,有时不分场合甚至在需要十分清醒的情况下,也出现不同程度、不可抗拒的入睡。睡眠觉醒后可出现短暂的意识模糊状态,心率及呼吸节律增快。过多的睡眠常引起患者显著的痛苦或社交、职业等其他重要功能的受损。患者可出现认知和记忆功能障碍,表现为记忆减退,思维能力下降,学习新鲜事物出现困难,甚至意外事故发生率增多。患者常被误认为懒惰、不求上进,造成严重的心理压力,使患者情绪低落。

\subsubsection{诊断}

CCMD-3诊断标准为:

1.症状标准

(1)白天睡眠过多或睡眠发作。

(2)不存在睡眠时间不足。

(3)不存在从唤醒到完全清醒的时间延长或睡眠中呼吸暂停。

(4)无发作性睡病的附加症状(如猝倒症、睡眠瘫痪、入睡前幻觉、醒前幻觉等)。

2.严重标准 患者为此明显感到痛苦或影响社会功能。

3.病程标准 几乎每天发生,且至少已有1个月。

4.排除标准 不是由于睡眠不足、药物、乙醇、躯体疾病所致,也不是某种精神障碍症状的一部分。

\subsubsection{鉴别诊断}

患者出现每天睡眠时间过多或睡眠发作持续一个月以上,需要排除由于发作性睡病及其附加症状所致的睡眠过度综合征,例如猝倒、睡眠瘫痪、入睡前幻觉及醒前幻觉。嗜睡症患者的脑电图检查为正常睡眠脑波。睡眠过度还需与睡眠呼吸暂停综合征相鉴别。此类患者白天呼吸正常,睡眠时可有阻塞性呼吸暂停发作,常伴有极强的鼾音及呼吸暂停现象。患者体形肥胖,呼吸不规则,睡眠常因呼吸暂停而中断,血氧饱和度明显降低,又称之为匹克威克综合征。

\subsubsection{治疗}

嗜睡症的治疗较棘手。

1.刺激-控制治疗 定时唤醒,有计划地打瞌睡。

2.兴奋药的应用 哌醋甲酯、苯丙胺、匹莫林,宜从小剂量开始,逐渐加量,直至症状减轻。使用兴奋剂后,可能会加重夜间睡眠障碍,可适当加用短效的安眠药。

3.SSRI类药物 有一定效果,如氟西汀、帕罗西汀、舍曲林。

4.心理治疗 包括支持性心理治疗、认知行为心理治疗。

\subsubsection{预后}

预后多不良。

\subsection{睡眠-觉醒节律障碍}

睡眠-觉醒节律障碍(sleep-wake rhythm
disorders)是指睡眠-觉醒节律与要求的不符,导致对睡眠质量的持续不满状态,患者对此有忧虑和恐惧心理,并引起精神活动效率下降,影响社会功能。本病不是任何一种躯体疾病或精神障碍症状的一部分。如果睡眠-觉醒节律障碍是某种躯体疾病或精神障碍(如抑郁症)症状的一个组成部分,不另诊断为睡眠-觉醒障碍。我国尚无确切有关睡眠-觉醒节律障碍的流行病学调查资料。本病多见于成年人,儿童期或青少年期发病者少见。生活节律失常和心理社会压力是本病常见的病因。

有些睡眠-觉醒障碍患者由于生活节律的改变,引起白天醒觉不完全,可表现为记忆差、懒散、不能很好地进行学习,给工作和生活都带来影响,有的还会导致车祸和意外事故的发生。在老年人中,往往由于影响到认知功能而被误认为是痴呆等。

\subsubsection{临床表现}

1.Jet
Lag(时区变更)综合征 在跨多个时区后出现失眠、工作障碍、胃肠症状或其他症状。个体对时区变更的敏感性不同,严重程度不同。本病与跨时区的速度和数量有关,速度越快,数量越多,不适感越明显。

2.轮班睡眠障碍 与特殊的工作安排有关,可有失眠和(或)过度睡眠。其并发症包括胃肠症状、心血管症状、酒精滥用、家庭和社会生活的紊乱、信心不足和效率下降、经常旷工。

3.睡眠延迟综合征 主要睡眠时间比预定的时间晚数小时,表现为早上很难在预定的时间醒来,常见于青春期。

4.睡眠提前综合征 主要睡眠时间比预定的睡眠时间提前,患者常抱怨傍晚时困倦,入睡早,早醒,老年人容易出现。

5.非24小时睡眠觉醒综合征 是一种慢性疾病,患者在正常环境中每天入睡时间延长30分钟至2小时。患者处于没有时间提示的一个暂时隔离的环境中,在大约25小时的周期中自由的休息和活动,其作息时间和现实生活中的时间同步或不同步地不断轮转。在普通人群中很少见,而在盲人中发病率为40%。可能是由于视上核与环境不完全同步所致,但人格障碍也可能涉及这一机制。

\subsubsection{诊断}

CCMD-3诊断标准为:

1.症状标准

(1)患者的睡眠觉醒节律与所要求的(即与患者所在环境的社会要求和大多数人遵循的节律)不符。

(2)患者在主要的睡眠时间段失眠,而在应该清醒时间段嗜睡。

2.严重标准 明显感到苦恼或社会功能受损。

3.病程标准 几乎每天发生,并至少已1个月。

4.排除标准 排除躯体疾病或精神障碍(如抑郁症)导致的继发性睡眠-觉醒障碍。

\subsubsection{鉴别诊断}

个体睡眠被剥夺后,因长时间没有睡眠,会不分场合迅速入睡,而睡眠-觉醒节律障碍患者并没有睡眠剥夺史,即在正常生活状态下出现睡眠-觉醒节律障碍。老年痴呆患者常常出现白天嗜睡、晚上不睡、漫游、日落后意识障碍等症状,需与本症鉴别。

\subsubsection{治疗}

治疗方法主要是逐步调整患者入睡和觉醒的时间,养成良好的睡眠习惯,以恢复患者的正常睡眠节律。为防止复发,常需结合药物巩固效果。

1.使生物钟与昼夜周期位相一致,维持并强化治疗。

2.相应的时间暗示 睡眠延迟综合征可延迟上床时间,提前2~3小时唤醒。

3.强光 睡眠提前综合征可暴露于夜光中。

4.合理安排日常作息时间和工作、社交活动。

5.药物治疗 褪黑素与强光同时治疗有利于改变和建立生物钟的周期位相;咪达唑仑和三唑仑有利于快速入睡,对于应该入睡而不能入睡的患者有显著疗效。

\subsection{睡行症}

睡行症(somnambulism)又称梦游症,是指一种在睡眠中出现的以行走或其他异常行为或活动为特征的睡眠障碍,通常在非快速眼动睡眠的慢波期出现。主要表现为睡眠中起床,漫无目的地行走,做一些简单刻板的动作,少数可表现为较复杂的行为,如在睡眠中做饭、进食、驾车等。患者活动可自行停止,一般不说话,询问也不回答,多能自动回到床上继续睡觉。不论是即刻苏醒或次晨醒来均不能回忆。患者在发作时对环境只有简单的反应,易发生磕碰、摔倒等意外伤害,并且意识混乱,不易被唤醒,可能会做出攻击行为或产生危害他人的严重后果。本病多见于儿童少年,儿童期患病率为1%~17%,11~12岁为发病高峰期,此后随年龄的增加患病率下降。本症没有痴呆或癔症的证据,可与癫痫
并存,但应与癫痫 发作鉴别。

\subsubsection{诊断}

CCMD-3诊断标准为:

1.症状标准

(1)反复发作的在睡眠中起床行走。发作时睡行者表情茫然、目光呆滞,对别人的招呼或干涉行为相对缺乏反应,要使患者清醒相当困难。

(2)发作后患者自动回到床上继续睡觉或躺在地上继续睡觉。

(3)尽管在发作后的苏醒初期可有短暂意识和定向障碍,但几分钟后,即可恢复常态,不论是即刻苏醒或次晨醒来均完全遗忘。

2.严重标准 不明显影响日常生活和社会功能。

3.病程标准 反复发作的睡眠中起床行走数分钟至半小时。

4.排除标准

(1)排除器质性疾病(如痴呆、癫痫
等)导致的继发性睡眠觉醒节律障碍,但可与癫痫
并存,应与癫痫 鉴别。

(2)排除癔症。

说明:睡行症可与夜惊并存,此时应并列诊断。

\subsubsection{鉴别诊断}

1.精神运动性癫痫
 精神运动性癫痫
发作很少只在晚上发作,发作时也对环境刺激无任何反应,可有吞咽、摸索等无意义动作,脑电图可有癫痫
样放电。有时同一患者也可两者并存。

2.神游症 发作始于清醒状态,发作持续时间长(数小时到数天),警觉程度高,能完成复杂的、有目的的行为,如骑车、旅游,发作醒来身处异地。多见于成年人。

\subsubsection{治疗}

1.对因治疗 儿童睡行症是神经系统发育不全、不完善所致,是生理性的,大多数在15岁前后自行消失,无需特殊处理。但成年睡行症则可能是病态的,应排除是否是癫痫或癔症。

2.室内安全措施 为防止患者撞墙,寝室内不应该放置带锐角的家具;为防止患者跌伤,楼梯上应装有铁门;为防止患者坠窗,窗户上应装有护栏。

3.发作时处理 发作时引导患者上床,不要唤醒患者,因为非但不能叫醒,反而延长发作时间,也不要强拉患者上床,否则可激起攻击行为。

4.平时预防 睡前服用苯二氮䓬
类药物和三环类抗抑郁药,可降低睡眠深度,减少发作。3周为1个疗程。但长期使用可发生耐受,断药后可反跳性加重。预防睡眠过深是减少发作的关键,因此需预防睡眠不足、过度疲劳、精神紧张和饮酒。

\subsubsection{预后}

大部分患者可以自行缓解,尤其是儿童睡行症。

\subsection{夜惊}

夜惊(night
terrors)是指一种常见于幼儿的睡眠障碍,主要为睡眠中突然惊叫、哭喊,伴有惊恐表情和动作,以及心率加快、呼吸急促、出汗、瞳孔扩大等自主神经兴奋症状。通常在夜间睡眠后较短时间内发作,每次发作持续1~10分钟。清醒后对发作时的体验完全遗忘。诊断本症应排除热性惊厥和癫痫
发作。

\subsubsection{诊断}

CCMD-3诊断标准为:

1.在一声惊恐性尖叫后从睡眠中醒来,呈反复发作,不能与环境保持适当接触,并伴有强烈的焦虑、躯体运动及自主神经功能亢进(如心动过速、呼吸急促及出汗等),持续1~10分钟,通常发生在睡眠初1/3阶段。

2.对别人试图干涉夜惊发作的活动相对缺乏反应,若干涉几乎总是出现至少几分钟的定向障碍和持续动作。

3.事后遗忘,即使能回忆,也有限。

4.排除器质性疾病(如痴呆、脑瘤、癫痫
等)导致的继发性夜惊发作,也需排除热性惊厥。

说明:睡行症可与夜惊并存,此时应并列诊断。

\subsubsection{鉴别诊断}

梦魇只是普通的噩梦,可发生在睡眠的任何时间,易被唤醒,对梦的经过能够回忆。

\subsubsection{治疗}

1.对因治疗 小儿夜惊是中枢神经系统发育不全所致,无需追究其原因。成人夜惊可能有病理性原因,如人格障碍或偏头痛等,应充分评估并予以相应治疗。

2.发作前处理 偶尔发作无需处理,经常发作可用苯二氮䓬
类药物或丙咪嗪。

3.发作时处理 限制患者运动,防止跌倒和撞伤。

\subsubsection{预后}

虽然是良性的,但可能有暴力性行为,导致自伤、伤人,或环境损害,偶尔会有司法问题。

\subsection{梦魇}

梦魇(nightmares)是指在睡眠中被噩梦突然惊醒,一旦醒来就变得清醒,对梦境中的恐怖内容能清晰回忆,并心有余悸,梦境内容与白天的活动、恐惧或所担心的事情有一定的联系。女性比男性多见。通常在夜间睡眠的后期发作,发生于快眼动睡眠阶段。在儿童中发病率为20%左右,成人为5%~10%。11岁以前经常梦魇多为生长过快所致,未必有病理意义。12岁以后经常梦魇多有精神病理问题,可有精神分裂症、分裂性人格。其他方面,如精神创伤性生活事件、焦虑、害怕、恐怖、抑郁、不安全感、内疚、发热和突然停用苯二氮䓬
类药物,均可使其发作。

\subsubsection{诊断}

CCMD-3诊断标准为:

1.从夜间睡眠或午睡中惊醒,并能清晰和详细地回忆强烈恐惧的梦境,这些梦境通常危及生存、安全或自尊。一般发生在睡眠的后半夜。

2.一旦从恐怖的梦境中惊醒,患者能迅速恢复定向和完全清醒。

3.患者感到非常痛苦。

\subsubsection{鉴别诊断}

本病需与夜惊鉴别(表\ref{tab13-1})。

\begin{table}
\centering
\caption{梦魇和夜惊的鉴别诊断}
\label{tab13-1}
\begin{tabular}{ccc}
\toprule
项目 & 梦魇 & 夜惊\\
\midrule
家族遗传性 & 低(7.1\%) & 高(96\%)  \\
睡眠位相 & 快波睡眠相 & 慢波睡眠4相\\
发作时间 & 入睡后3~6小时 & 入睡后1~2小时\\
声音 & 偶有不可理解的声音 & 持续性尖叫\\
运动 & 很少 & 在床上无目的地打滚、踢打 \\
对呼唤反应性 & 呼之易醒 & 呼之无反应 \\
事后回忆 & 能 & 不能 \\
自主神经症状 & 不明显 & 明显 \\
发作后再入睡 & 困难 & 容易 \\
\bottomrule
\end{tabular}
\end{table}

\subsubsection{治疗}

一般无需治疗,发作频繁者,应检查有无心血管系统疾病、哮喘、消化系统疾病和精神疾病。对有精神疾病的患者要治疗精神疾病,对无精神疾病的患者应解除促发因素。氯丙嗪或阿普唑仑虽能减少梦魇发生率,但停药后梦魇会反跳性加重,不能解决根本问题。对于有创伤性生活事件者,心理治疗有效。

\subsubsection{预后}

往往与患者人格、创伤性生活事件有关。

\section{非器质性性功能障碍}

非器质性性功能障碍(nonorganic sexual
dysfunction)是指一组与心理社会因素密切相关的性功能障碍。性功能障碍有各种表现形式,即个体不能进行自己所期望的性活动,包括兴趣缺乏、快感缺乏,不能产生为有效的性行为所必需的生理反应(如勃起),或不能控制或体验到高潮等。有些类型的功能障碍(如性欲缺乏)男女都可发生。不过,女性主诉性主观体验不满意较多见(如快感或兴趣缺乏),而缺乏特异性反应的较少见。男女在性功能障碍的体验上可能存在一定的差异,如一旦女性的性反应的一个方面受到了影响,其他方面也很可能会受损。女性不能体验到性高潮,那么她也常会觉得无法享受调情的其他乐趣,并因此丧失大部分性欲。而男性尽管主诉无法产生特异性反应如勃起或射精,却仍有性欲存在。

常见的非器质性性功能障碍有性欲减退、阳痿、早泄、性乐高潮缺乏、阴道痉挛、性交疼痛等。非器质性性功能障碍常与心理社会因素密切相关。它不包括器质性疾病、药物及衰老所致的性功能障碍,不是其他精神障碍症状的一部分。

症状标准:成年人不能进行自己所希望的性活动。

严重标准:对日常生活或社会功能有所影响。

病程标准:符合症状标准至少已3个月。

排除标准:不是由于器质性疾病、药物、酒精及衰老所致的性功能障碍,也不是其他精神障碍症状的一部分。

说明:可以同时存在一种以上的性功能障碍。

\subsection{病因}

由于性是一种正常的生理功能,性生活是成人的一种正常需要,而性反应是一种心身过程。心理及躯体过程通常都在性功能障碍的发病中起作用。与性活动相关的情况有:

1.性生活情况 如性生活方式、初次性交时的感受、性交的频率、是否有过快感、性交环境、彼此的性爱好情况(方式、类型、频率、时间)、是否长期存在其他性满足方式等。

2.对性生活的认识与态度 如对性的了解程度、接受性教育的情况、家庭对性的态度、是否存在与性交相关的迷信及配偶对性生活的态度等。

3.夫妻关系及感情 夫妻关系及感情对性生活影响很大,很多患性功能障碍的患者实际上只是夫妻关系及感情有问题的一种表现,应详细了解。需了解的内容有:夫妻关系及感情如何、是否有婚外情、彼此的性吸引力如何等。

4.其他方面影响 影响性生活的因素很多,除了现实与性生活关系密切的因素外,应激、对健康和身体的担心、童年生活经历及性创伤等也可能影响性生活的满意度。

5.情绪状态及稳定性 是否有抑郁、焦虑、恐惧等。

\subsection{诊断}

\subsubsection{诊断依据}

1.病史 成年人不能进行自己所希望的性活动已达3个月以上,且不是由于器质性疾病、药物、酒精及衰老所致的性功能障碍,也不是其他精神障碍症状的一部分。

2.临床表现 有性欲减退、阳痿、早泄、性高潮缺乏、阴道痉挛、性交疼痛等,亦可同时存在一种以上的性功能障碍。

3.精神检查 主要包括对性生活的认识与了解程度;情绪状态及稳定性,是否有抑郁、焦虑、恐惧等;性生活的设定情况,如方式、类型、频率、时间等。

4.体格检查及实验室检查 进行一般常规的体格检查,应重点注意患者有无性器官畸形等。实验室检查,如性激素检查等。

5.辅助检查 脑电图、头颅CT、MRI及性激素等检查有助于排除器质性疾病,LES、MMPI、SAS、SDS、HAMA、HAMD等心理评定工具有助于进一步客观地了解患者情况。

\subsubsection{诊断标准}

CCMD-3非器质性性功能障碍的诊断标准及临床类型为:

1.性欲减退(lack or less of sexual
desire) 指成年人持续存在性兴趣和性活动的降低,甚至丧失。

(1)符合非器质性性功能障碍的诊断标准。

(2)性欲减低,甚至丧失,表现为性欲望、性爱好及有关的性思考或性幻想缺乏。

(3)症状至少已持续3个月。

2.阳痿(impotence)

(1)男性符合非器质性性功能障碍的诊断标准。

(2)性交时不能产生阴道性交所需的充分阴茎勃起(阳痿),至少有下列1项:①在做爱初期(阴道性交前)可充分勃起,但正要性交时或射精前,勃起消失或减退;②能部分勃起,但不充分,不能够性交;③不产生阴茎的膨胀;④从未有过性交所需的充分勃起;⑤仅在没有性交时,产生过勃起。

3.冷阴(failure of female genital response)

(1)女性符合非器质性性功能障碍的诊断标准。

(2)性交时生殖器反应不良,如阴道湿润差和阴唇缺乏适当的膨胀,至少有下列1项:①在做爱初期(阴道性交前)有阴道湿润,但不能持续到使阴茎舒适地进入;②在所有性交场合,都没有阴道湿润;③某些情况下可产生正常的阴道湿润(如和某个性伙伴性交,或手淫过程中或并不打算进行阴道性交时)。

4.性乐高潮障碍(orgasmic dysfunction)

(1)符合非器质性性功能障碍的诊断标准。

(2)从未体验到性乐高潮(原发性)或有一段性交反应相对正常,然后发生性乐高潮障碍(继发性)。性乐高潮障碍可进一步分为:①普遍性性乐高潮障碍,发生于所有的性活动中和与任何性伙伴在一起时。②男性的境遇性性乐高潮障碍,至少有下列1项:性乐高潮仅发生于睡眠中,从不发生于清醒状态;与性伙伴在一起时从无性乐高潮;与性伙伴在一起时出现性乐高潮,但不是阴茎在进入或保持在阴道内的时候。③女性在某些情况下可有性乐高潮,但明显减少。

5.早泄(premature ejaculation)

(1)符合非器质性性功能障碍的诊断标准。

(2)不能推迟射精以充分享受做爱,并至少有下列1项:①射精发生在进入阴道前或刚刚进入阴道后;②在阴茎尚未充分勃起进入阴道的情况下射精;③并非因性行为节制继发阳痿或早泄。

6.阴道痉挛(vaginismus)

(1)符合非器质性性功能障碍的诊断标准。

(2)阴道周围肌群的痉挛阻止了阴茎进入阴道或阴茎进入阴道时不舒服,至少有下列1项:①原发性阴道痉挛,是指从未有过正常反应。②继发性阴道痉挛,是指一段性活动的反应相对正常,然后发生阴道痉挛;当不进行阴道性交时,可产生正常的性反应;对任何性接触的企图都恐惧,并力图避免阴道性交。

7.性交疼痛(dyspareunia)

(1)符合非器质性性功能障碍的诊断标准。

(2)男性在性活动过程中感到疼痛或不舒服。

(3)女性在阴道性交的全过程或在阴茎插入很深时发生疼痛,不能归于阴道痉挛或阴道湿润差。

说明:器质性病变所致的性交疼痛应根据原发疾病分类。

\subsection{鉴别诊断}

由于性功能障碍的病因较多,其中一部分可能是器质性疾病或其他精神障碍的症状,因此需充分排除器质性疾病、药物、酒精及衰老所致的性功能障碍和其他精神障碍所致性功能障碍。

\subsection{治疗}

\subsubsection{治疗原则}

非器质性性功能障碍的治疗以减少或消除症状、恢复正常的性生活及体验为主。治疗方法以心理治疗为主,必要时辅以中、西药对症治疗。

\subsubsection{治疗方案}

1.性知识教育 多数性功能障碍的患者可能存在性知识缺乏或认识不当,必要的教育是治疗性功能障碍的基础。

2.心理治疗 各种心理治疗方法都可能有效,如精神分析、认知行为、催眠治疗、婚姻家庭治疗以及性行为训练等,要根据患者的情况及存在问题的特点选择合适的心理治疗方法。性功能障碍治疗的前提是建立恰当的治疗关系,消除患者的顾虑,取得患者及其配偶的充分配合。婚姻治疗和夫妻交流训练有可靠的疗效。

3.药物治疗 药物治疗的使用原则是仅在症状严重、配合心理治疗或患者的确存在一些需要处理的问题时才使用。不应让患者寄全部希望于药物,否则既抑制了患者自身的调节能力,同时也会因为药物的不良反应出现其他问题。药物的种类包括增强性功能、神经调节药、小剂量抗抑郁药和抗焦虑药等。