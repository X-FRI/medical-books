\chapter{精神活性物质与非依赖物质所致精神障碍}

\section{概  述}

精神活性物质(psychoactive
substance)是指来自体外,可影响精神活动,并可导致成瘾的化学物质,又称成瘾物质(addictive
substance)、物质(substance)或药物(drug)。

根据药理特性,精神活性物质可分为以下几类:酒类、阿片类物质、大麻类物质、镇静催眠药或抗焦虑药、中枢神经兴奋药、致幻剂、烟草和挥发性溶剂等。

精神活性物质和毒品不是同一个概念。毒品(illicit
drug)通常指能使人成瘾并在社会上禁止使用的化学物质。我国刑法规定:毒品为“鸦片、海洛因、吗啡、大麻、可卡因以及国务院规定管制的其他能够使人形成瘾癖的麻醉药品和精神药品”。由此可见,毒品是精神活性物质,但精神活性物质并不都是毒品。

20世纪80年代末期,国际毒潮开始侵袭中国,毒品(当时主要为海洛因)滥用在我国死灰复燃。值得注意的是,近年来吸食冰毒(甲基苯丙胺)、摇头丸(亚甲二氧甲基苯丙胺)、K粉(氯胺酮)等新型毒品的人员比例正在迅速上升。与阿片、海洛因等传统毒品相比,新型毒品的精神依赖性更强,但由于其躯体依赖性相对较弱,不少吸毒者对其危害认识不足。这些新型毒品的滥用,多发生在娱乐场所,所以又被称为“俱乐部药(club
drugs)”或“舞会药(party
drug)”。滥用新型毒品者大多是一些青少年,由于对其危害性认识不足,这些新型毒品的滥用呈蔓延态势,而且在有的地区,新型毒品的滥用已超过了海洛因等传统毒品。

吸食毒品严重损害滥用者个体的身心健康,影响其工作、事业和前途。吸毒可导致机体免疫功能下降,肌注或静脉注射毒品往往引起感染性并发症,吸毒甚至可危及生命。公用注射器和针头可传播各种传染性疾病,严重危害自己和他人的身体健康,造成严重的公共卫生问题。采用注射途径吸毒是我国艾滋病传播的主要途径。吸毒还会严重损害家庭关系,家庭成员中一旦出现了吸毒者,家庭经济不堪重负,往往导致其他家庭成员沾染上毒品,以致家庭破裂。吸毒妇女一旦怀孕,还会影响到胎儿的正常发育,造成胎儿宫内发育迟缓和早产,并可能通过母婴途径患上传染性疾病,新生儿还会出现毒瘾发作。毒品问题还会诱发大量的违法犯罪活动,影响社会稳定和经济发展。据统计,大多数省市发生的抢劫、盗窃等侵财性案件中,30%以上是吸毒人员所为,一些毒情严重地区甚至接近70%。

总之,精神活性物质的滥用不仅严重损害吸毒者个人的身心健康,而且还会传播疾病,降低国民素质,扰乱社会安宁,引发各种犯罪活动,因此,精神活性物质的滥用是人类共同面临的严重公共卫生问题。

不同类型的精神活性物质所导致的精神障碍在临床表现和严重程度上各不相同。根据国际疾病分类(ICD-10)和中国精神障碍分类与诊断标准(CCMD-3),精神活性质所致精神障碍包括急性中毒、有害使用、依赖综合征、戒断状态、伴有谵妄(包括震颤谵妄)的戒断状态、精神病性障碍、遗忘综合征、残留性和迟发性精神病性障碍以及其他精神和行为障碍。

\subsection{基本概念}

\subsubsection{急性中毒}

急性中毒(acute
intoxification)是指使用某些物质后引起短暂意识障碍或认知、情感、行为障碍,往往与剂量有关,且不符合依赖综合征、戒断综合征或精神病性障碍的诊断标准。

\subsubsection{滥用和有害使用}

物质滥用(drug
abuse)是指反复大量使用与医疗目的无关的、具有依赖特性的药物或化学物质,并导致躯体或心理健康损害,社会功能受损。滥用强调的是不良后果,滥用者不一定有明显的耐受性增加或戒断症状。如有明显戒断症状,就表明形成了依赖。物质滥用是一种适应不良性用药方式,在DSM-Ⅳ中有此诊断类别。有害使用(harmful
use)与物质滥用同义,在CCMD-3和ICD-10中使用这一名称。

\subsubsection{耐受}

耐受(tolerance)是指长期使用某种药物以后,个体对该药的反应性下降,为了取得与初始用药时相同的药效,就必须增加该药的用量。一般来说,躯体依赖形成后,个体对该药物也会产生耐受。耐受的产生往往导致用药方式的改变。例如初始用药是采用烫吸,因欣快感产生耐受,为了取得原先或预期的效果,则会改为静脉注射。

\subsubsection{依赖综合征}

药物依赖(drug dependence)俗称药物成瘾(drug
addiction)。依赖综合征是指对使用某种物质有强烈的渴求,对使用该物质的自控能力下降,反复使用该种物质,以取得特定的心理效应,且避免减量或停药后出现戒断症状的一种行为障碍。

药物依赖包含躯体依赖和心理依赖两个方面。躯体依赖(physical
dependence)又称生理依赖,是指长期使用成瘾物质后造成机体的一种病理性的稳态,一旦撤药,这种稳态难以维持,导致以戒断症状为主要表现的生理功能严重紊乱。心理依赖(psychological
dependence)又称精神依赖性,俗称“心瘾”,由于用药后产生欣快感、舒适感,多次用药后导致心理上对所用药物的渴求(craving)或持续的觅药行为(drug
seeking
behavior)。虽然滥用者能够认识到使用药物对个人的身体、前途、家庭等均产生极其有害的影响,但由于对药物的渴求非常强烈,以致不择手段获取药物。心理依赖是导致复吸的重要原因,如何消除心理依赖是戒毒治疗过程中面临的最大难题。

\subsubsection{戒断综合征}

戒断综合征(abstinence syndrome)又称撤药综合征(withdrawal
syndrome),指反复使用,通常是长期和(或)高剂量使用某种精神活性物质后,在减量或停药时出现的一组症状。戒断症状的出现是躯体依赖形成的标志。戒断综合征的出现和病程有时限性,而且与所使用的物质的类型有关。阿片类物质、酒类、巴比妥类物质的戒断症状较明显。

\subsection{病因学理论}

物质滥用和依赖的原因和机制非常复杂,不是单一因素所引起,而是与生物、社会和心理等多方面的因素有关。

\subsubsection{社会环境因素}

社会环境因素,包括家庭环境,是物质滥用和依赖形成的一个重要因素,药物的可获得性是物质滥用和依赖的一个前提,同伴的诱惑往往也是吸毒的一个常见原因。社会文化背景、社会观念对物质滥用也有重要影响,尤其是在烟、酒的滥用中起着重要作用。环境因素与复吸有密切关系。有些人经强制戒毒后已经脱瘾,但一回到原来的生活环境,就会出现对毒品的强烈渴求,导致复吸。

\subsubsection{心理因素}

心理因素中研究得最多的是个性。采用三维人格问卷进行调查,发现药物依赖者的“好奇寻求(novelty
seeking)”分值较高。“好奇寻求”分值较高者有冲动、爱探索、易激动、不守法、易分心等特点。除个性因素外,精神状态也与物质滥用有关,精神创伤、不良生活事件、情绪不稳、抑郁、焦虑等均有可能导致物质的滥用和依赖的产生,酗酒或吸食毒品已成为有些人缓解压力的一种方式。

\subsubsection{遗传生物学因素}

生物学因素在物质依赖的形成中发挥重要作用。酒精依赖在不同的种族以及不同的家庭中发病率有差异,提示遗传素质与酒依赖有关。家系研究和双生子研究也提示,海洛因等阿片类物质的依赖有一定的遗传易感性。

\subsection{依赖形成的神经学基础}

对物质依赖的形成机制还不完全明了。个体对某种物质产生依赖,表明该物质的使用能产生愉快感,即奖赏效应(rewarding
effect)。产生愉快感的神经学基础是大脑中存在奖赏系统(rewarding
system),奖赏系统通过产生愉快感而调节和控制行为活动。奖赏系统中最重要的是中脑边缘多巴胺系统(mesolimbic
dopamine
system),该系统神经元胞体起源于腹侧被盖区,其纤维投射到嗅球、杏仁核、隔区、前额叶皮质,尤其是大部分纤维投射到伏隔核。强大的奖赏效应可能是毒瘾形成、维持、复发及强迫性觅药行为的基础。多巴胺是产生奖赏效应的重要神经递质,不同物质导致奖赏效应的机制不同,但最终均增加脑内的DA水平。

物质依赖的本质是中枢神经系统产生的代偿性适应,包括细胞功能和形态的适应,是长期用药后产生的病理状态的平衡。因此,现在医学界公认,物质依赖是一种慢性脑病。

\section{阿片类物质所致精神障碍}

\subsection{概述}

阿片(Opium)又称鸦片(俗称大烟),是罂粟(Asian poppy
plant)未成熟的蒴果浆汁干燥物。阿片原产于西亚和欧洲,隋唐时传入我国,是最古老的麻醉镇痛药。古希腊希波克拉底(Hippocrates)就已用阿片治病。明朝李时珍在《本草纲目》中对阿片的药理作用也有描述。18世纪中叶开始,英国等欧美国家向中国大量输入鸦片以抵消贸易逆差,攫取高额利润。众所周知,鸦片战争失败后,阿片在我国泛滥成灾,严重摧残了国民的身心健康。1949年以前,我国吸毒人员众多,约有2000多万人。中华人民共和国成立后,由于党和政府的重视,经过几年的努力,才禁绝了阿片在我国的流行。20世纪80年代后期以来,以阿片类物质为主的毒品滥用又在我国流行。

阿片类含20多种生物碱,按化学结构不同可分为菲类和异喹啉类。菲类生物碱主要有吗啡和可待因,有镇痛、镇咳和致欣快作用;异喹啉类生物碱主要有罂粟碱,有解痉作用。吗啡(Morphine)是阿片类最主要的生物碱。

阿片类物质包括天然来源的鸦片以及其中所含的有效成分,如吗啡、可卡因(甲基吗啡),也包括人工合成或半合成的化合物,如海洛因(heroin)、二氢埃托啡(dihydroetor-phine)、哌替啶(dolentin)、美沙酮(methodone)等。

目前滥用的阿片类物质主要是海洛因,其产地主要在东南亚的缅甸、泰国、老挝三国交界的“金三角”地区和西南亚的巴基斯坦、阿富汗、伊朗交界的“金新月”地区。海洛因俗称“白粉”,是吗啡的衍生物,化学名是二乙酰吗啡(diacetylmorphone),通常是一种白色或棕色的粉末,易溶于水,并可很快通过血脑屏障,在中枢迅速水解为吗啡。海洛因的致欣快作用比吗啡强,戒断症状也比吗啡重,因此其成瘾性更强。

\subsection{药理作用和成瘾机制}

阿片类物质通过与阿片受体作用产生效应。阿片受体主要有μ,δ和κ受体等,每种受体又有不同的亚型。兴奋μ和δ受体引起欣快和满足,产生奖赏效应,导致依赖的产生;兴奋κ受体则引起厌恶反应。

物质依赖是长期使用成瘾物质后所引起的一种脑病。与躯体依赖有关的脑区有中脑导水管周围灰质、内侧丘脑、下丘脑、杏仁核、黑质、苍白球以及中缝核和蓝斑等中脑和脑桥结构,其中,蓝斑是脑内最大的去甲肾上腺素能核团,在阿片类物质躯体依赖的形成和戒断症状的表达中发挥重要作用。对药物的渴求及心理(精神)依赖则与主要中脑边缘多巴胺系统有关。伏隔核是中脑边缘多巴胺系统中的一个关键核团。腹侧被盖区的DA神经元兴奋后,其神经末梢释放到伏隔核、前额叶皮质等脑区的DA增多,引起奖赏效应。

阿片类物质使DA水平增加,可能是通过作用于GABA中间神经元间接作用于腹侧被盖区内的DA能神经元。腹侧被盖区是中脑边缘DA系统中DA神经元胞体所在的部位。阿片类物质作用于GABA能神经元上的μ受体,抑制GABA神经元的活性,减少GABA的释放,取消对DA神经元的紧张性抑制,使释放到伏隔核的DA水平增加。伏隔核DA水平增加是阿片类物质产生奖赏效应的关键。

阿片类物质依赖的本质是中枢神经系统产生的代偿性适应,这种代偿性适应不仅发生在阿片受体作用系统,还发生在非阿片受体作用系统(如NMDA受体作用系统),现在越来越多的研究对非阿片受体作用系统在阿片类物质依赖中的作用进行探索,非阿片受体作用系统在阿片类物质依赖的形成中可能也起着重要作用。

cAMP通路上调是阿片类物质依赖的主要分子基础之一。阿片类药物急性作用通过抑制腺苷酸环化酶(AC)而抑制蓝斑(LC)神经元的放电,而慢性作用则引起AC-cAMP系统的上调。代偿性适应的本质是基因的表达发生了改变,基因表达的关键是基因转录的调控。药物依赖对基因表达的调控主要涉及两类转录因子家族:一类是cAMP反应元结合蛋白(cAMP
response element-binding protein,
CREB)和CREB样蛋白;另一类是快反应基因(immediate early
genes),如c-fos, c-jun。

\subsection{临床表现}

以海洛因为主的阿片类物质的滥用方式有烫吸、皮下注射、肌内注射、静脉注射等方式。初吸者大多采用烫吸,又称“追龙(chasing
dragon)”,即将海洛因放在锡纸上,下面加热使之化为烟雾,再用吸管吸入。近年来,不少吸毒者在滥用毒品后不久就开始肌注或静注毒品,一是为了追求快感,二是为了“节省”毒品。不少吸毒者能熟练地自己把海洛因注射到体表静脉甚至股静脉内。使用后很快起效,持续数小时。用药后即体验到一种欣快感,持续数分钟,随之进入一种无忧无虑的松弛状态,感觉所有的烦恼、焦虑、痛苦均烟消云散,心情非常宁静、舒适。所产生的欣快感通常被描述成是一种“飘”的感觉,但长期吸毒后往往很难再次体验到初始吸毒时的快感。

\subsubsection{急性中毒}

阿片类物质急性中毒见于蓄意自杀或误用过量的药物。新中国成立前吞食鸦片是一种常见的自杀方式。海洛因中毒大多为意外静脉注射过量药物所致,其原因通常有:初次使用时对自身的耐受情况不了解,使用了其他海洛因依赖者所使用的剂量;为追求快感盲目提高剂量或改变用药途径;所使用的海洛因纯度改变,如由原先的低纯度海洛因改为使用高纯度的海洛因,而剂量又未变,则容易发生意外中毒;停用一段时期后耐受性下降,此时再次静脉注射原剂量海洛因造成过量等。

海洛因中毒的特征性表现是昏迷、呼吸抑制、针尖样瞳孔三联征。就诊时多半已深昏迷,呼吸极慢、浅表,有时仅每分钟2~4次,面色发绀,皮肤湿冷、心率缓慢、血压下降、肺水肿、下颚松弛致舌后坠阻塞呼吸道等,可因呼吸衰竭死亡。

\subsubsection{依赖综合征}

阿片类物质反复使用后即可形成依赖,停药或减量后会出现戒断症状。随着依赖的形成,个体对药物的存在产生适应,即产生了耐受。为了获得快感和避免戒断症状的产生,吸毒者不断增加剂量,并出现人格和行为特征的改变。此时,吸食毒品已成了他们的“首要工作”,为了吸毒可以放弃一切,可以不顾个人的前途、身体健康、家庭责任和道德法律。长期吸毒后生活方式发生改变,工作能力减退,性格变得乖戾,谎话连篇。为了获得毒资可以不择手段,甚至违法犯罪。在吸毒人群中,女性卖淫、男性偷盗的现象非常多见。长期用药后躯体方面的改变有食欲不振、身体衰弱、性欲和性功能减退等。

\subsubsection{戒断综合征}

长期使用阿片类物质后停药或减量即出现戒断症状。戒断综合征的出现是判定成瘾的重要临床标志。戒断症状的轻重与用药种类、剂量大小、用药次数、用药持续的时间等有关。用药次数越多,持续时间越长,则戒断症状越重。

戒断症状一般在停药8~12小时即出现,于36~72小时达高峰。主要表现为:对药物产生强烈的渴求,伴难以克制的觅药行为,情绪恶劣、烦躁不安,激惹易怒;躯体方面出现流涕、打哈欠、软弱无力、瞳孔扩大、体毛竖起(皮肤出现鸡皮疙瘩)、出汗、恶心、呕吐、腹痛、腹泻、肌肉和骨骼疼痛、自发射精、血压上升、脉搏加快、发热、失眠等。有人诉戒断时全身还有虫爬感,不堪忍受,称宁可自杀也不愿忍受这种痛苦。3天后戒断症状逐渐减轻,7~10天后主要症状基本消失。躯体情况差者突然停药有时会导致死亡。

躯体戒断症状消失后仍有失眠、焦虑、情感脆弱、心境恶劣、全身不适、乏力等,称为稽延性戒断症状(protracted
abstinence
syndrome)。即使躯体症状完全消失,对药物的心理渴求仍长期存在。

\subsubsection{躯体并发症}

阿片类物质滥用后可引起食欲减退、营养不良、体重下降、性功能减退、月经不规则、停经、免疫功能下降等。免疫功能下降再加上用药方式不卫生,特别是肌内或静脉注射时共用注射器及针头,因此并发各种感染非常多见。常见的继发感染有细菌性心内膜炎、呼吸道感染、肺部感染、肝炎、皮下脓肿、蜂窝组织炎、血栓性静脉炎、败血症等。吸毒还是感染艾滋病(AIDS)的重要途径。吸毒者可能通过以下两种途径感染艾滋病病毒:一是吸毒者之间共用被污染的不洁注射器,在静脉内注射毒品,致使艾滋病病毒在吸毒者中传播;二是吸毒者往往有不洁性行为,这也是传播艾滋病的原因。长期使用皮下、肌内注射方式者,往往身上布满针眼。长期静脉注射者,局部皮肤可见条索状瘢痕。

孕妇滥用阿片类物质对胎儿会造成严重不良影响,可出现死胎、早产、低体重儿。由于胎儿在母体内已形成依赖,胎儿在出生2~4天后即可出现戒断症状,表现激惹不安、高声哭叫、呼吸快、睡眠障碍、鼻塞、打哈欠、喷嚏、呕吐、腹泻、发热、四肢震颤,甚至痉挛发作。

\subsection{阿片类药物依赖的治疗}

阿片类药物过量中毒者的治疗,除一般抢救措施外,可静脉注射特效解毒药纳洛酮。纳洛酮是阿片受体拮抗剂,可反复使用,直至呼吸增快、瞳孔扩大、神志清醒。由于阿片类药物过量中毒者一般被送至综合医院急诊科抢救,因此,本节仅涉及阿片类药物依赖的治疗。

现代医学认为,成瘾的形成与生物、心理以及社会等诸方面因素密切相关。因此,现在全球公认的戒毒治疗方案不应仅仅着眼于躯体症状一个方面,而应从吸毒成瘾的机制出发,按照生物-心理-社会模式进行全面考虑。阿片类药物依赖的治疗包括脱毒(detoxification)、预防复吸,以及社会心理康复和回归社会三个完整的过程。

\subsubsection{脱毒治疗}

脱毒治疗主要采用药物治疗方法使吸毒者顺利渡过急性戒断反应期,减轻躯体戒断症状,减轻吸毒者生理上的痛苦。阿片类物质成瘾的脱毒治疗主要方法是替代治疗,辅以非替代治疗。替代治疗就是以具有相似药理作用的药物替换所滥用的毒品,然后再逐渐减少剂量,直至停药。目前主要采用美沙酮替代递减疗法和丁丙诺啡(buprenorphine)脱毒治疗。非替代治疗是以非阿片类药物来减轻戒断症状,主要使用可乐定(clonidine)。

1.美沙酮替代递减疗法 美沙酮是人工合成的阿片类镇痛药,药理作用与吗啡相似,口服后吸收好,生理利用度达90%,口服后3小时血药浓度达到峰值,半衰期约15小时。美沙酮口服后能有效地抑制戒断症状24~32小时。美沙酮久用也可成瘾,但成瘾的形成较慢,戒断症状也较轻。美沙酮于20世纪60年代中期即用于阿片类物质依赖的治疗。美沙酮替代递减疗法是目前最常用的脱毒治疗方法。该方法适用于各种阿片类药物的戒毒治疗,尤其是海洛因依赖。美沙酮替代递减疗法的使用原则是:单一用药,逐日递减,先快后慢,只减不加,停药时应坚决。一般而言,静脉注射海洛因每天在1g左右者,美沙酮的初始剂量为每日40~60mg;如是烫吸,则美沙酮的初始剂量可为每日25~40mg。首次剂量后应密切观察戒断症状控制的程度以及对美沙酮的耐受情况,在第2天调整剂量。3天后开始以每日5~10mg的速度递减,减至每日10hg时应放慢减量速度,改为每1~3天减1mg。减量后应对情绪障碍、睡眠障碍等进行对症处理。

2.丁丙诺啡脱毒治疗 丁丙诺啡是阿片受体部分激动药,主要作用于μ受体,对μ受体的激动作用可抑制对海洛因的渴求,可用于治疗和预防海洛因依赖。丁丙诺啡有针剂和片剂两种剂型。根据海洛因戒断症状的轻、中、重程度,平均每日给予丁丙诺啡3mg、4mg、6mg,分3~4次舌下含服,最大剂量不超过每日8mg,4日后减量,第7日停药。

3.可乐定 可乐定是α\textsubscript{2}
受体激动药,激活抑制性肾上腺素能神经元,抑制戒断时去甲肾上腺素功能亢进,从而减轻戒断症状。其优点是属非阿片类药物,不产生欣快作用,无成瘾性。主要不良反应有降压、体位性低血压、晕厥等,长期使用后突然停药可出现反跳性血压升高、头痛等。由于老年人对降压作用较敏感,因此该药不适于老年人。有心脑血管疾病或肝肾功能障碍者也禁用。剂量与用法:治疗剂量一般为每日1.2~1.5mg,个别可高达每日2.0mg,分3次服用。首日剂量不宜太大,约为最高剂量的2/3,第2日增至最高量,从第5日开始每日递减20%,第11日停药。治疗的前4天尽量卧床休息,避免剧烈活动,不应突然改变体位。如连续发生体位性低血压或血压持续低于(90/50mmHg),应适当减药,可减当日剂量的1/4。

4.其他 如中草药、氯丙嗪、针刺等。

此阶段是戒毒治疗的第一步,随后应转入后两个阶段。若只进行单纯的脱毒治疗,则疗效不佳,1年内复吸率几乎达100%。

\subsubsection{防复吸治疗}

脱毒治疗只是消除或减轻了严重的戒断症状,脱毒后还存在稽延性戒断症状(如失眠、焦虑、抑郁及躯体症状),吸毒者对药物仍有强烈的心理渴求。这些因素加上环境的影响,很可能会引起复吸。如何防止复吸,是戒毒治疗的难点。因此,应特别重视脱毒后的康复、防复吸治疗。有条件者最好在脱毒治疗后脱离原来的生活环境,并由专人对其进行严密的监督。

纳屈酮(naltrexone)可用于预防复吸。纳屈酮是阿片受体拮抗剂,与阿片受体有很强的亲和力,阻断阿片类物质与受体的结合,使阿片类物质不产生欣快效应,从而降低对毒品的渴求和复吸。在开始使用纳屈酮前必须先实施脱毒治疗,在脱毒治疗完毕7~10天后方可使用纳屈酮,以免促发戒断症状。在进行纳屈酮治疗前要做纳洛酮激发试验(具体方法可参见有关参考书),试验结果为阴性时方可开始纳屈酮治疗。首次纳屈酮剂量为25mg,观察1小时后,如无戒断症状,则再加25mg,即给足首日治疗量。以后使用方法可为每日50mg,或每日100mg。

开展美沙酮维持治疗是目前预防复吸的重要途径。美沙酮于1972年就被美国FDA批准作为治疗阿片类物质成瘾的替代药物。目前,美沙酮维持治疗已成为阿片类物质依赖维持疗法中应用最为广泛的治疗方法。我国内地从2004年开始,在部分地区的吸毒人群中开展美沙酮维持治疗试点工作。美沙酮具有口服使用有效、作用时间长、可以减少或消除海洛因依赖者对阿片类药物的心理渴求、耐受性的产生缓慢、药物滥用潜力低等特点。但美沙酮在药理学本质上与海洛因一样,均为阿片受体激动剂。美沙酮维持治疗只是以一种阿片类药物替代另一种阿片类药物,因此,美沙酮维持治疗仍只是一种姑息治疗手段。但美沙酮维持治疗的意义在于:使用美沙酮口服液缓解停用海洛因后出现的戒断症状;降低维持治疗者对毒品的渴求,减少觅药和用药行为;减少注射毒品的行为并减少通过共用注射器传播血源性疾病(特别是艾滋病)的机会;减少非法毒品交易和吸毒者的违法犯罪行为;恢复病人的社会功能和家庭功能;可以与病人保持联系,以便于为他们提供防病知识、社会支持、心理辅导,鼓励他们逐渐戒除毒品。

\subsubsection{社会心理康复}

戒毒者回归社会之后,应给予接纳、照管,不予歧视,对戒毒者提供各方面的支持和帮助,使他们能作为一个正常人适应并融入正常的社会生活之中,这对预防复吸也有重要作用。

\subsection{阿片类药物滥用的预防}

毒品问题包括两方面内容:一是存在大量的吸毒人群,另一方面是社会上有从事毒品非法生产和贩运的罪恶活动。因此,减少毒品的非法供应和降低对毒品的非法需求是国际禁毒战略的两项核心内容。两大禁毒战略必须同时并举。首先要杜绝毒品的供应,如果不杜绝毒品的非法供应,就会不断产生新的吸毒者,同时,吸毒者在脱毒治疗后也会很容易复吸。降低毒品的非法供应需要公安、司法、海关、边防等部门的配合,以及全社会的共同参与和努力,同时要杜绝或降低对毒品需求。

控制毒品滥用的关键在于预防,要做到防患于未然。一级预防是对正常人群进行宣传教育,宣传毒品知识及其危害,预防毒品的使用。二级预防是对高危人群进行宣传教育,及早发现吸毒者并对其进行早期干预,缩短滥用时间。三级预防是针对吸毒人群,为吸毒者提供脱毒、康复、重返社会等服务,消除或减轻滥用毒品带来的严重危害。

\section{酒类所致精神障碍}

酒类饮料的主要成分是乙醇,是一种亲神经性物质,对中枢神经系统有重要影响。精神障碍可在一次饮酒后发生,也可由长期饮酒形成依赖后逐渐出现,或突然停饮后急剧产生症状。除精神障碍外,往往合并有躯体症状和体征。

\subsection{流行病学}

饮酒在世界各地都是普遍存在的生活习惯和社会风俗,同样也成为世界各国的社会和公共卫生问题。2001年在国内5城市的调查中酒依赖的患病率男性为6.6%,女性为0.2%,总患病率为3.8%。在西方发达国家酒的消费更大,酒精依赖发生率也更高。

\subsection{病因与发病机制}

\subsubsection{遗传因素}

家系调查发现,嗜酒有明显的家族聚集性,与酒精依赖患者有血缘关系的家庭成员中,酒精依赖的患病率高于一般人群,酒精依赖者一级亲属患酒依赖的危险性比对照组高4~7倍。双生子调查发现,双生子的饮酒行为和酒精依赖的同病率单卵双生子高于双卵双生子。寄养子研究也发现,生父为酒精依赖者的男性被寄养者的酒滥用发生率比生父为非酒精依赖者的男性被寄养者高,而患酒精依赖的生父自己抚养的儿子和被寄养出去的儿子之间,酒精依赖的发生率则无显著差异。这些研究结果提示遗传因素在酒精依赖中起重要作用。

\subsubsection{社会心理因素}

社会传统、文化习俗、经济状况、职业特点、家庭情况和个人的性格特征等均与酒精所致精神障碍的发生相关。

\subsubsection{代谢和药理作用}

乙醇经胃肠黏膜吸收,在肝内通过乙醇脱氢酶转变为乙醛,然后经乙醛脱氢酶转变为乙酸,最后代谢为水和二氧化碳。当乙醛脱氢酶缺乏时,乙醛在体内积聚,它和乙醇代谢时产生的其他毒性代谢产物对机体产生毒性作用,特别会影响中枢神经细胞,严重者导致细胞死亡。此外,乙醇也可以直接损害神经细胞。

酒精(乙醇)是亲神经物质,能迅速透过血脑屏障,进入脑内,而大脑又是对酒精最敏感的器官。酒精是中枢神经系统抑制剂,血液内酒精浓度不同,对大脑的抑制程度也不同。小剂量时抑制大脑皮质,使抑制性的控制机制受到压制,导致抑制的解除(脱抑制),出现兴奋。高浓度则导致精神运动性抑制和嗜睡,浓度更高可抑制中脑功能,干扰脊髓反射以及调节温度和控制心脏呼吸功能的延髓中枢,抑制呼吸、心跳,产生意识障碍。

\subsection{与酒类相关的精神障碍类型}

\subsubsection{急性酒精中毒}

急性酒精中毒(acute alcohol
intoxication)时的临床表现与血液内酒精浓度及作用时间有关。在早期或小剂量时,由于抑制大脑皮质,出现兴奋症状,表现欣快、言语活动增多、判断和控制能力受损、易怒、易产生攻击行为或不恰当的性行为等。血液内高浓度的酒精则引起言语不清、步态不稳、动作笨拙不协调、眼球震颤、反应迟钝、注意记忆能力下降及其他认知缺损、嗜睡等。如浓度更高,则可抑制呼吸、心跳,导致昏睡、昏迷甚至死亡。酒精与苯二氮䓬
类药或其他中枢神经系统抑制剂合用,则对中枢神经系统的抑制增强,易造成死亡。急性酒精中毒有以下类型:

1.普通醉酒(drunkenness) 为一次较大量饮酒引起的急性中毒,出现一种特殊的兴奋状态,言语增多,情绪兴奋,易激动,控制能力削弱。同时表现走路不稳,手震颤,口齿不清,此外还有心率增快,血压降低,皮肤血管扩张、面部充血,有时呕吐、眩晕等。醉酒严重者则表现嗜睡、少语。除重症外,一般能自然恢复,无后遗症。

2.病理性醉酒(pathological
drunkenness) 很小量饮酒即引起严重的精神病性发作。患者意识模糊不清,具有强烈的兴奋性及攻击行为,无单纯醉酒状态时的步态不稳、口齿不清。有时出现片断的幻觉妄想,多为恐怖内容,因而常发生攻击性行为;剧烈兴奋,持续几分钟到几小时,酣睡后结束,有完全或部分遗忘。

3.复杂性醉酒(complex
drunkenness) 患者一般均有脑器质性病史,或者有影响酒代谢的躯体疾病,在此基础上,患者对酒的敏感性增高,小量饮酒便发生急性中毒反应,出现明显的意识障碍,伴错觉、幻觉或被害妄想,显著的情绪兴奋、易激惹,攻击和破坏行为。发作通常持续数小时,缓解后对经过部分或全部遗忘。

\subsubsection{酒精有害使用}

酒精有害使用又称问题饮酒、酒精滥用,指饮酒已使家庭生活或身体健康出现问题,社会功能受到损害。由于长期大量饮酒并经常酒后滋事,给自身形象、工作、生活、人际交往、事业、前途带来一系列负面的影响。

\subsubsection{酒精依赖综合征}

酒精依赖综合征俗称“酒瘾”,是长期使用酒精后中枢神经系统所产生的适应性改变。酒精依赖的形成与饮酒量及饮酒时间有关。酒精依赖综合征主要表现为对酒精有强烈的渴求,喝酒已成为他们生活中不可缺少的一部分,而且,视饮酒为生活中最重要或非常重要的事,为了喝酒,可以不顾家庭和工作。对酒的耐受性不断增加,表现为酒量不断增加。停止饮酒或减少饮酒后则出现戒断症状,喝酒后症状消失。晨饮是酒精依赖的典型症状,经过一夜代谢,血液内酒精浓度降低,会出现戒断症状,所以醒后便要喝酒。

\subsubsection{酒精戒断综合征}

1.单纯戒断反应 长期饮酒形成酒依赖者在停止饮酒或减少饮酒数小时后即可出现戒断症状,表现为:震颤(手抖或舌震颤)、无力、厌食、恶心呕吐、烦躁不安、易怒、失眠、反射亢进,患者对饮酒有强烈的渴求。继之出现自主神经系统兴奋症状,如心动过速、血压升高,出汗、高热、心律不齐、肌肉抽动等。厌食,恶心、呕吐,会导致脱水、电解质紊乱。急性酒精戒断症状一般在最后一次饮酒后3~5天开始缓解。

2.震颤谵妄 震颤谵妄(delirium
tremens)是最严重的酒精戒断症状,一般在最后一次饮酒48~72小时后出现。表现为肢体粗大震颤、意识模糊、定向障碍、激越、躁动不安、错觉、幻觉,以及发热、大汗、心跳加快等自主神经症状。往往还出现行为紊乱、自言自语、思维不连贯等。意识水平波动,时而清醒,时而模糊。幻觉中以生动鲜明的视幻觉常见,可因此表现为极度恐惧、大声叫喊或出现冲动行为。症状缓解后,对发病经过往往不能回忆。

3.酒精性癫痫
 严重者可出现抽搐发作,一般在最后一次饮酒12~48小时后出现。

\subsubsection{酒精所致精神病性障碍}

1.酒精中毒性幻觉症(alcohol
hallucinosis) 酒精依赖者长期饮酒后,在意识清晰的状态下出现生动、丰富的幻觉,如幻听、幻视,以视幻觉多见,可引起相应的情绪和行为障碍。大多发生在突然停饮或显著减少饮酒量后,也可发生在持续饮酒的情况下,可持续数日、数周或数月。

2.酒精中毒性妄想症(alcohol delusional
disorder) 酒精依赖者长期饮酒后,在意识清晰的状态下出现的妄想状态,缓慢起病,嫉妒妄想多见。

\subsubsection{酒中毒性记忆及智力障碍}

慢性酒中毒者可出现记忆障碍(主要为近记忆损害)、虚构、定向障碍,称之为柯萨可夫综合征(Korsakov's
syndrome)。酒中毒性痴呆(alcohol
dementia)是长期大量饮酒所导致的智力减退,表现为记忆损害、痴呆、人格改变。

\subsubsection{酒精所致人格改变}

长期饮酒可导致人格发生改变,患者只对饮酒有兴趣,对亲人冷淡,对家庭无责任心,对工作无兴趣,说谎等。

\subsection{酒精依赖的治疗}

急性酒精中毒者一般被送至综合医院急诊科抢救,除一般抢救措施外,可静脉注射纳洛酮。酒精所致精神病性障碍在治疗上除戒酒外,可给予抗精神病药对症处理。

1.戒断症状的治疗 戒酒应根据患者酒精依赖和中毒的严重程度灵活掌握戒酒进度,轻者可尝试一次性戒断,对酒精依赖严重者采用递减法逐渐戒酒,避免出现严重的戒断症状而危及生命。无论一次或分次戒酒,临床上均要密切观察与监护,尤其在戒酒开始的第一周,特别注意患者的体温、脉搏、血压、意识状态和定向力,及时处理可能发生的戒断反应。

酒精与苯二氮䓬
类药药理作用相似,对单纯戒断症状,可给予苯二氮䓬
类药,如安定、氯硝安定。纳洛酮静脉滴注对缓解酒精戒断症状也有较好的疗效。酒依赖者大多数有神经系统损害以及躯体营养状态较差,应该给予神经营养药,同时补充大量维生素,加强支持治疗。对震颤谵妄者,要密切观察生命体征,严密监护,加强支持治疗。在药物治疗方面,由于口服给药往往不合作,可给予纳洛酮静脉滴注,肌内注射或静脉给予安定或氯硝安定,肌内注射氟哌啶醇等控制症状。

2.酒精依赖的康复 与阿片类物质依赖一样,酒精依赖的治疗,关键还在于如何消除心理依赖、在脱瘾后不再饮酒。在康复期可以使用的药物有以戒酒硫为代表的酒增敏药物、催吐药物等,但依从性差,实际临床意义有限。

\section{中枢兴奋药所致精神障碍}

中枢兴奋药(central nervous system
stimulants)也称精神兴奋药(psychostimulants),是指能激活或增强中枢神经活性的物质,包括苯丙胺(amphetamine)、可卡因(Cocaine)、咖啡因(caffeine)和其他黄嘌呤类物质。20世纪80年代我国滥用的毒品主要是海洛因。20世纪90年代以来,中枢兴奋药(尤其是苯丙胺类中枢兴奋药)的滥用人数呈上升趋势,苯丙胺类毒品滥用人数的增长速度已远远高于海洛因、可卡因等传统毒品。

苯丙胺类中枢兴奋药包括苯丙胺、甲基苯丙胺(methyamphetamine)、亚甲二氧甲基苯丙胺(3,4-methylenedioxymethamphetamine,MDMA)、哌醋甲酯(利他林)(methylphenidate)、苯甲马林(phenmetrazine)、芬氟拉明(fenfluramine)等。目前滥用的苯丙胺类中枢兴奋药主要有甲基苯丙胺和亚甲二氧甲基苯丙胺。

中枢兴奋药的滥用方式有鼻吸、皮下或静脉注射、抽吸。中枢兴奋药几乎总是与其他精神活性物质一起滥用,通常是酒或阿片类物质。酒可增强可卡因等中枢神经兴奋药的欣快效应并减轻其不良反应。

\subsection{药理作用}

苯丙胺类兴奋药和可卡因抑制去甲肾上腺素和多巴胺的再摄取,促进多巴胺的释放。两者有非常相似的药理效应和拟交感效应,可阻断去甲肾上腺素的再摄取,引起心动过速、高血压、血管收缩、瞳孔扩大、出汗、震颤;阻断多巴胺的再摄取,引起兴奋、厌食、刻板动作、活动过多以及性兴奋。苯丙胺有短暂的兴奋作用,使用后出现欣快、感觉精力充沛、疲劳感消失、自信心增强。大多数苯丙胺类兴奋药的精神活动效应持续时间比可卡因长,而可卡因引起心律失常、抽搐等严重的躯体并发症的危险性则比苯丙胺类兴奋药大。

\subsection{临床表现}

\subsubsection{急性中毒}

急性中毒的临床表现有兴奋、欣快、警觉性增高、焦虑、紧张、愤怒、判断力损害、刻板行为、运动困难、肌张力障碍、精力旺盛、对睡眠需求减少、厌食、恶心呕吐、体重下降。自主神经症状有心动过速、血压升高、瞳孔扩大、出汗、震颤。其他症状有胸痛、心律失常、呼吸抑制、意识模糊、抽搐等。

\subsubsection{苯丙胺性精神病}

长期滥用或使用高剂量苯丙胺类药物可引起中毒性精神病,表现为意识清晰的状态下出现丰富的幻听、幻视、错觉、关系妄想、被害妄想等。临床表现与偏执型精神分裂症十分相似,但与剂量有关,而且病程较短,停用该类物质后症状缓解。

\subsubsection{耐受和依赖}

反复使用中枢兴奋药后,个体对药物逐渐产生耐受,因此剂量也逐渐加大,以获得所期待的欣快效应。长期使用后不良反应增加,而欣快效应降低。

与阿片类物质不同,中枢神经兴奋药的戒断症状轻,突然停用中枢神经系统兴奋药不会引起严重的戒断症状,持续使用主要是对药物有心理上的渴求,而不是为了减轻或消除戒断症状。常见的戒断症状有情绪抑郁、精神运动性迟滞、感到疲劳、对睡眠的需求增多、食欲增加以及对药物的渴求。虽然中枢神经系统兴奋药的躯体戒断症状较轻,但戒断后常出现严重抑郁情绪,可导致自杀。

\subsubsection{治疗}

由于突然停用中枢神经兴奋药不会引起严重的戒断症状,因此不需要逐渐递减或进行替代治疗。对戒断症状的处理主要是采用支持治疗,对情绪抑郁等可采取对症治疗,对可能出现的自杀行为应加以防范,对出现精神病性症状者可短期使用抗精神病药。

\subsection{几种常见的中枢兴奋药}

\subsubsection{甲基苯丙胺}

甲基苯丙胺俗称冰毒,因其形状呈结晶状似冰而得名。甲基苯丙胺的滥用方式有口服、鼻吸、静脉注射及抽吸。在抽吸或静脉注射后立即就会有非常强烈的快感,持续几分钟。口服或鼻吸所产生欣快效应不如抽吸或静脉注射时强烈。甲基苯丙胺对中枢神经系统有强烈的兴奋作用,其兴奋作用和依赖性潜力比苯丙胺更强,滥用者可很快成瘾,而且滥用的频度和剂量加大。吸毒者初始滥用时精神亢奋、不知疲倦,活动明显增加,睡眠减少,食欲减退,呼吸加快,高热,心率加快,血压升高,冲动易怒,行为失控甚至发生自杀和伤人行为。大剂量应用可致抽搐、摇头、震颤、意识模糊。高热和抽搐可导致死亡。虽然甲基苯丙胺有强烈的兴奋作用,但作用过后感觉情绪抑郁、全身乏力、精神萎靡。为了再次获得快感,吸毒者就会再一次吸食,以至反复使用形成依赖。

甲基苯丙胺有神经毒性作用,损害多巴胺、5-HT及其他神经递质的神经元。长期使用后多巴胺含量下降,引起帕金森病样症状。

\subsubsection{亚甲二氧甲基苯丙胺}

亚甲二氧甲基苯丙胺(MDMA)俗称“摇头丸”、“迷魂药”、“销魂剂”,外观为白色药片,是合成的精神活性物质。社会上滥用的“摇头丸”的成分变化很大,除亚甲二氧甲基苯丙胺外,可能还含有咖啡因、海洛因、麦司可林(mescaline)等。20世纪90年代以来,“摇头丸”作为一种所谓的“俱乐部药”在西方国家的娱乐场所广为滥用。近年来,我国一些地区的歌舞厅等娱乐场所也出现了滥用“摇头丸”的现象,国内已有服用“摇头丸”后出现妄想、幻觉、行为异常等精神症状的报道。

亚甲二氧甲基苯丙胺的化学结构与中枢神经兴奋药苯丙胺和致幻药麦司可林相似,能刺激5-HT的释放,具有中枢神经兴奋和致幻作用。中枢神经兴奋效应有愉快感和自我满足感,与他人有亲近感并有想接触他人的欲望,感觉精力充沛。该药可引起幻觉、幻视、眩晕、空间定向力障碍。服用后使人极度兴奋、摇头不止,可造成行为失控,引发治安问题。大剂量滥用可引起中毒,长期滥用可导致精神障碍。

亚甲二氧甲基苯丙胺可造成持久的记忆及其他功能受损,这种损害在停药2周以后仍然存在,损害的程度与所用的剂量直接相关。PET研究发展,亚甲二氧甲基苯丙胺可损害5-HT神经元,使5-HT转运体的数量显著减少。由于5-HT神经元在情绪的调节、记忆、睡眠、食欲等方面发挥重要作用,因此,亚甲二氧甲基苯丙胺对记忆及其他功能的损害可能与其损害5-HT神经元有关。

亚甲二氧甲基苯丙胺还影响体温和心血管系统的调节。由于亚甲二氧甲基苯丙胺的滥用大多在舞厅等场所,室内温度较高,且服用后长时间极度活动,再加上该药对体温和心血管系统调节的影响,可引起脱水、心率加快、高血压、高热,甚至心肾功能衰竭死亡。

\subsubsection{可卡因}

可卡因是一种从古柯叶(古柯灌木的叶子)中提取的生物碱。古柯灌木生长于南美,从很古老的时候开始,当地土著人就咀嚼古柯叶以消除疲劳,获取兴奋效应。可卡因是一种强烈的中枢神经系统兴奋药,具有很强的成瘾性,可以说是当前所有滥用药物中成瘾性最强的。非医疗使用时可产生欣快或难以入睡,反复使用则产生依赖。可卡因的滥用主要在西方国家,是国外最常滥用的毒品之一。19世纪20年代初可卡因就已在美国流行,国内使用较少。

可卡因的使用方式有鼻吸、抽吸以及皮下或静脉注射。通过鼻腔吸入可卡因粉末是最常见和最安全的方式。20世纪70年代可卡因的滥用方式主要是鼻吸,而使用抽吸方式者少,因为抽吸加热时盐酸可卡因会被分解。注射的方式可使药物快速到达大脑,产生强烈的欣快作用。可卡因引起多巴胺水平升高导致欣快,是成瘾性很强的物质。

由于抽吸时加热可分解盐酸可卡因,20世纪80年代出现了另一形式的精炼的可卡因------克赖克(crack)。克赖克是一种可卡因游离碱,通过把可卡因溶于乙醚、氨水或碳酸氢钠,再加热去除盐酸而制成,因为抽吸时加热发出噼啪声而得名。克赖克易挥发,加热时不被破坏,所以可通过抽吸方式滥用。抽吸克赖克起效快、作用强,因此它一出现就替代了盐酸可卡因而迅速流行,使可卡因的滥用问题变得更为严重。

可卡因的药理效应与苯丙胺中枢兴奋药相似,用药后出现欣快感、精力旺盛、不知疲倦、自信心增强、对睡眠需求减少、食欲减退。欣快效应持续的时间取决于用药途径。吸收越快,欣快感越强,作用持续的时间也越短。鼻吸后产生的欣快持续15~30分钟,而抽吸则持续5~10分钟。周围效应有外周血管收缩、瞳孔扩大、体温升高、心率加快、血压升高等。

有人出现坐立不安、激越、焦虑,可出现对欣快感的耐受。高剂量或长期使用还可促发偏执。静脉注射可卡因或抽吸克赖克有时可导致死亡。死亡的原因通常是心脏骤停、呼吸抑制、癫痫
发作等。

长期大剂量滥用可卡因可引起慢性脑病综合征,表现为反应迟钝、短时记忆力下降、共济运动失调等。

可卡因精神病主要与游离碱可卡因的滥用或以注射方式使用可卡因有关,表现与苯丙胺精神病类似,可出现幻觉、妄想、刻板动作、兴奋冲动等。病程比苯丙胺性精神病短,一般可在48~72小时消失。

长期滥用造成依赖,停药几小时后即出现戒断症状,表现为对药物强烈的渴求、抑郁、无力、易激惹、焦虑不安、嗜睡、精力减退等。

\section{大麻类物质所致精神障碍}

大麻是一种草本植物,种植的区域分布很广,亚、非、美、欧及大洋洲均有种植,我国也有不少地区种植大麻。根据大麻中的有效成分四氢大麻酚含量的高低。大麻植物可分为毒品型和纤维型两种,我国种植的大麻大多属毒品型大麻。

大麻(cannabis)是大麻植物(cannabis
sativa)的生物活性制剂。根据大麻植物的品种、气候、土壤、种植和制作方法的不同,大麻制剂的活性成分和效应有很大差异,最通常方式是把它掺入香烟中抽吸或用烟斗吸入,或制成雪茄吸入,也可混入食物中食用或泡茶饮用。玛利华纳(marijuana)是大麻植物的粗制品,大麻的一种浓缩树脂制品叫“哈希什”(hashish),一种黏稠的黑色液体制品叫“哈希油”(hash
oil)。

在美国,大麻是最常用的毒品,约有33%的成年人使用过大麻,通常还是青少年除酒外第一个使用的非法药物。近年来,大麻滥用在我国也有流行。

大麻中至少有400多种化学物质,其中具有精神活性效应的物质统称大麻类物质或大麻素(cannabinoids)。大麻类物质中活性最强的是四氢大麻酸(delta-9-tetrahydrocannabinol,
Δ-9-THC)。不同制品之间Δ-9-THC的含量不同。玛利华纳含Δ-9-THC
4%~8%,哈希什含Δ-9-THC
5%~12%。Δ-9-THC与脑部大麻素受体结合产生效应。该受体的数量在不同脑区分布不一,在影响愉快、记忆、思维、注意、感觉和时间、运动协调的脑区有许多该受体。

大麻有致幻觉和镇静作用,产生的效应个体间存在差异,与所使用的剂量、使用者的期望、情绪以及社会环境有关。小剂量及中等剂量时引起欣快感、幸福感、放松、情绪脱抑制(emotional
disinhibition),四肢有轻飘飘的感觉。欣快之后出现倦睡、镇静。可出现感知歪曲,如感觉时间变慢、产生错觉等,但程度比麦角酰二乙胺(LSD)所导致的轻。大剂量时则产生LSD样作用,如幻觉、思维紊乱、偏执、惊恐发作、激越等。

吸食大麻可损害认知和操作功能,可影响驾驶及其他复杂的、需技巧的操作,容易引起交通事故,影响学习成绩、工作、社会技能和日常生活,还会引起抑郁、快感缺乏、焦虑、人格改变。

滥用大麻对身体健康会造成不良影响,对心血管系统最明显的作用是心动过速,心跳加快及血压小幅度下降常见,球结膜血管扩张,出现红眼。抽吸大麻后的第一个小时心脏病发作的危险性增加,这与心率、血压和血液携氧能力下降有关。吸大麻者易患呼吸道和肺部疾病。肿瘤的发病危险增高,因为大麻含刺激物质和致癌物质。大麻中的致癌物质比烟草中多50%~70%,比吸烟更易致癌。长期滥用大麻影响免疫功能,还能抑制睾酮分泌,这对青少年滥用者来说,是特别值得注意的问题。

有些人在长期使用后可出现依赖,停药后会出现轻微的戒断症状,主要表现有对药物的渴求、易激惹、失眠、焦虑、不安、出汗、食欲减退、胃部不适等,在停用后近1周时达到高峰。因戒断症状轻微,无需进行脱毒治疗。

\section{镇静催眠药或抗焦虑药所致精神障碍}

镇静催眠有巴比妥类和非巴比妥类,对中枢神经系统有抑制作用,小剂量产生镇静,大剂量可引起催眠、抗惊厥、麻醉。药物过量引起急性中毒,可因抑制呼吸中枢而死亡。所有的镇静催眠药都能产生耐受和依赖,长期用药后突然停药可出现戒断症状,表现为兴奋、焦虑、震颤,甚至惊厥。该类药物目前已很少使用,因此临床上几乎见不到这类药物依赖的患者。

抗焦虑药有苯二氮䓬
类药、丙二醇类和丁螺环酮。丙二醇类目前已很少使用,非苯二氮䓬
类药丁螺环酮为5-HT\textsubscript{1A}
受体部分激动剂,其优点之一是无成瘾性。但由于丁螺环酮抗焦虑作用起效慢,不能改善睡眠,因此目前使用的抗焦虑药仍然以苯二氮䓬
类药为主,临床上所见到的抗焦虑药成瘾一般都是苯二氮䓬
类药引起。

苯二氮䓬
苯二氮䓬
类药抗焦虑药的理化特性与镇静催眠药不相同,但所产生的作用与镇静催眠药相似,即小剂量产生镇静,大剂量可引起催眠、抗惊厥。由于失眠、焦虑的患者众多,因此,苯二氮䓬
类药是目前最常用的精神药物之一,但该类药物最大的缺点是长期使用后会产生依赖,临床上苯二氮䓬
类药依赖者并不少见,戒断症状有失眠、焦虑、激越、震颤、头痛、多汗、抑郁,甚至抽搐发作。目前临床上使用的苯二氮䓬
类药均有依赖性。药物的作用时间越短,依赖性越大,形成依赖的时间越短,甚至使用通常剂量1个月后就可形成依赖。短效的苯二氮䓬
类药物(如阿普唑仑、罗拉西泮、艾司唑仑)戒断症状出现快,在停药1~2天后即出现戒断症状,而且症状较重;长效的苯二氮䓬
类药物长期使用停药,1~2周后出现戒断症状,持续时间长,但程度轻。

苯二氮䓬
类药用于失眠时,严格掌握适应证,搞清楚失眠的原因,防止滥用。在用于失眠时,大多数情况不应超过几周。撤药宜缓慢,以避免发生戒断症状。

对抗焦虑药所致精神障碍,首先要从预防做起。虽然目前使用的抗焦虑药仍然以苯二氮䓬
类药为主,但由于精神药理学的发展,现在可供选择的抗焦药的种类比以前多,除苯二氮䓬
类药物外,还有无成瘾性的丁螺环酮。另外,选择性5-HT再摄取抑制药(如帕罗西汀paroxetine)、文拉法辛(venlafaxine)和米氮平(mirtazapine)、曲唑酮(trazodone)等抗抑郁药也有抗焦虑作用,可作为抗焦虑药使用。对于失眠,还可以选择米氮平、酒石酸唑吡坦、曲唑酮等。因此,现在有条件从源头抓起,大幅度减少医源性苯二氮䓬
类药物依赖的形成。

\section{致幻药所致精神障碍}

致幻药(hallucinogen)又称迷幻药(psychedelics)、拟精神病药(psychotomimetics),是一类在不明显损害记忆和意识的情况下,产生类似于功能性精神病症状的精神活动物质。

\subsection{分类}

致幻药大体上可分为两类:吲哚烷基胺类和苯乙胺类。

\subsubsection{吲哚烷基胺类}

结构上与5-HT有关,如麦角酰二乙胺(lysergic acid diethylamide,
LSD)、二甲色胺(dimethytryptamine,
DMT)、二乙色胺(diethyltryptamine)、塞洛西宾(psilocybin)。

\subsubsection{苯乙胺类}

结构上与儿茶酚胺有关,如仙人球毒碱(mescaline)、苯环己哌啶(phencyclidine,
PCP)和氯胺酮(ketamine)。另外,亚甲二氧甲基苯丙胺(MDMA)、大麻也有致幻作用,也可列入此类。

\subsection{LSD所致精神障碍}

致幻药中最常见的是LSD,它是5-HT受体促动药,市售的有片剂、胶囊,偶有液体制剂。LSD的效应难以预测,取决于使用者个性、情绪、期望以及剂量和使用时的环境。摄入LSD后,通常在30~90分钟后产生效应。躯体方面的表现有瞳孔扩大、视物模糊、发热、心率加快、血压升高、出汗、食欲丧失、口干、震颤、动作不协调等。精神方面的效应有感知觉、情绪、思维等方面的改变,持续8~14小时。最显著的体验是感知觉的歪曲或增强。视觉形象变得异常鲜明、色彩丰富、轮廓清晰,对很平常的事也觉得很惊奇,常见错觉和幻觉。患者也常出现不同感觉形式的混淆,即共感(synaesthesia),如声音被感知为是看到的,颜色被体验为是听到的。可出现定向障碍、体像改变、人格解体,对时间、空间的感知发生改变。情绪变得异常强烈,表现多样,可同时体验到多种情绪或两种互不协调的情绪,并快速波动。思维障碍方面有牵连、偏执。

其他常见的不良反应有“倒霉之旅(bad
trip)”和闪回(flashback)。“倒霉之旅”:表现为极度惊恐,害怕自己会失控、发疯或死去,要求立即得到治疗。发作通常不超过2小时。应使这些患者在安静的环境内休息,并向其保证这种异常的体验是暂时的,是药物引起的。严重者可给予苯二氮䓬
类药物,避免使用抗精神病药。闪回:即停止用药后仍自发地再次体验到以前摄入致幻药时出现过的视觉歪曲、躯体症状、自我界限丧失或强烈的情绪,可以精确地重复既往吸入致幻药时的症状。闪回现象为发作性,持续数秒至数分钟,也可更长。闪回现象有时因疲劳、饮酒或大麻中毒而促发。闪回现象比较常见,可在停LSD数天内或1年后出现,估计可发生于25%以上的致幻药使用者中。随着时间的推移,闪回发作的次数迅速减少,发作时的强度也迅速降低。

慢性中毒可出现持久的焦虑、抑郁和精神病性症状。

\subsection{氯胺酮所致精神障碍}

氯胺酮俗称K粉,是一种白色结晶粉末,易溶于水。氯胺酮滥用主要是在一些娱乐场所,吸食方式为鼻吸、卷入香烟中抽吸或溶于饮料后饮用。目前,氯胺酮的滥用问题日益严重。

氯胺酮是一种分离性麻醉药,对丘脑-新皮质系统有抑制作用,选择性地阻断痛觉,而对边缘系统则有兴奋作用,造成痛觉消失而意识还部分存在的分离状态。

服用氯胺酮后会出现“去人格化(depersonalization)”、“去真实感(derealization)”、“人体形象(body
imagery)”改变、梦境、幻觉以及恶心、呕吐等。有些梦境或幻觉是“愉悦性”的,有些则是不愉快的、痛苦的。氯胺酮的梦幻作用是导致滥用产生的主要因素,这种梦幻作用因滥用者个体精神状况和滥用场景不同而有差异。

氯胺酮是苯环己哌啶的衍生物,可阻断NMDA(N-methyl-D-aspartate)受体,使正常个体产生阳性症状、阴性症状和认知缺陷,与精神分裂症的症状几乎难以区别,如给予精神分裂症患者NMDA受体阻断药,则可加重病情。

氯胺酮依赖的处理主要是对症治疗,可给予镇静催眠药,如果出现精神病性症状,可给予抗精神病药。

\section{烟草所致精神障碍}

烟草(tobacco)是世界上使用最普遍的物质,全球烟民已达12亿以上。我国是世界上最大的烟草生产和消费大国。据统计,我国有烟民3.5亿人,其中青少年烟民有5000万。吸烟严重损害人类的身体健康,我国每年100万的死亡人群中,有3/4是死于与烟草使用有关的疾病。吸烟是烟草使用最常见的方法,其他有吸雪茄、用烟斗吸烟、咀嚼烟草等。

\subsection{临床表现}

1.急性效应 烟草中的主要精神活性成分是烟碱,即尼古丁(nicotine)。吸入一口烟后,其中的烟碱可通过肺快速吸收,在几秒钟内到达大脑,作用持续长达30分钟。烟草使用后可产生多种效应,例如情绪改善、肌肉松弛、焦虑减轻、食欲下降,认知方面有注意力、记忆增强。由于烟碱的代谢速度快,脑中烟碱的含量很快降低,吸烟者在一支烟抽完后30~45分钟,就很想再抽一支。吸烟者往往在饭后、应激和焦虑时即很想抽烟。渴求的程度个体间有差异,戒烟的能力个体间也有明显不同。

2.耐受和依赖 烟草的依赖性潜力很强。像其他精神活性物质一样,烟草依赖的核心症状是难以克制的觅药行为。对烟草已形成依赖者,在停用数小时后即可出现戒断症状,表现为对烟草的渴求、抑郁、焦虑、易激惹、愤怒、注意力不集中、头痛、食欲增加、心跳加快、睡眠障碍、血压升高。对烟草的渴求在停用后24小时达到高峰,然后逐渐下降,持续数周,但可被有关刺激所诱发。

研究发现,烟碱的耐受和依赖还与应激和焦虑有关。一方面,应激和焦虑时个体往往采取吸烟的方式舒缓情绪;另一方面是应激和焦虑时肾上腺皮质激素分泌增多,肾上腺皮质激素可降低烟碱的效应,因此需要较多的烟碱才能取得同等的效应。

\subsection{作用机制}

烟草中的主要精神活性成分是烟碱。烟碱有高的成瘾性,它对中枢神经系统既有兴奋作用也有镇静作用。摄入烟碱后立即会产生兴奋,引起肾上腺皮质释放肾上腺素,刺激中枢神经系统及其他内分泌腺,随之出现抑郁,导致滥用者寻找更多的烟碱。

研究发现,烟碱与可卡因、海洛因、大麻一样,可提高多巴胺的水平。研究发现,烟碱成瘾的关键成分是一种特殊的分子------烟碱胆碱能受体的β\textsubscript{2}
亚单位。该分子与烟碱的奖赏效应有关。

烟碱与脑中的胆碱能受体结合发挥作用。已发现中枢神经系统有几种烟碱的胆碱能受体,激活这些受体与烟碱的强化效应及降低食欲有关。对外周烟碱受体的刺激则引起自主神经作用。短期使用烟草增加脑血流量,长期使用则降低脑血流量。

\subsection{戒烟方法}

实施戒烟可采取逐渐减量的方法,同时结合药物治疗和心理治疗。给予心理支持及技能训练,以应对高风险情形下可能出现的吸烟行为,并保持长期不吸。

关于烟草依赖的药物治疗,一些文献和参考书中介绍的方法很多,例如,国外有尼古丁香口胶(nicotine
chewing gums)、尼古丁透皮贴剂(nicotine transdermal
patch)、尼古丁鼻雾剂和吸入剂等。烟碱口香糖是获FDA通过的治疗烟碱依赖的药物,是烟碱的替代治疗。成功率各家报道相差较大。另一方法是使用烟碱皮贴,它释放相对恒定量的烟碱。其他还有烟碱喷雾剂和吸入剂。烟碱口香糖和烟碱皮贴等都是替代治疗方法,有助于缓解戒断症状。其他药物还有安非拉酮(bupropion)、去甲替林、可乐定等。如出现情绪不稳可用抗抑郁药和抗焦虑药。

\section{非依赖性物质所致精神障碍}

非依赖性物质所致精神障碍(mental disorder caused by non-dependence
substance)指来自体外的某些物质,虽不产生心理或躯体依赖,但可影响个体的精神状态,如产生摄入过量所致的中毒症状(以往称中毒性精神障碍)或突然停用所致的停药综合征(如反跳现象)。

常见的引起精神障碍的非依赖性物质是医用药物、有机化合物、一氧化碳、重金属及有毒食物等。这些物质进入体内后直接作用于中枢神经系统,引起认知损害、情感障碍、精神病性症状、人格改变和社会功能受损。临床表现一般可分为急性中毒和慢性中毒两类。短期内摄入较大剂量有毒物质后产生急性精神障碍,中毒较轻时主要表现为脑衰弱综合征,严重时则出现意识障碍,表现为急性脑病综合征。长期摄入小剂量有毒物质后产生慢性中毒,起病缓慢,临床表现较轻,但持续时间长。慢性中毒在不同阶段表现不同,早期一般表现为脑衰弱综合征,发展阶段则出现感知障碍、情感障碍和思维障碍,而在后期则可有智能障碍和人格改变,表现为慢性脑病综合征。

值得临床注意的是,随着化工工业的发展、农业生产中农药的广泛使用以及家居装饰、美容减肥药、抗感冒药等非处方药物的使用,导致精神障碍的非依赖性物质的种类也有所增多。临床诊断时必须获得有此类物质被摄入体内的证据,且有理由推断精神障碍系该物质所致,除残留性或迟发性精神障碍之外,精神障碍的发生应在非依赖物质直接效应所能达到的合理期限之内。

\subsection{非依赖性药物所致精神障碍}

\subsubsection{肾上腺皮质激素所致精神障碍}

肾上腺皮质激素按其生理作用可分为糖皮质激素和盐皮质激素。糖皮质激素在临床广泛使用,但不良反应也较多,可引起精神症状。一般认为,精神症状的产生与激素的剂量和疗程无关,而与激素的种类、患者的病前性格、既往精神病史和躯体功能状态有关。糖皮质激素中,地塞米松引发精神症状的可能性最大,以下依次为可的松、泼尼松、氢化可的松。

地塞米松可提高中枢神经系统的兴奋性,从而引起精神症状。精神症状常发生在用药后数日或2个月内。一般起病较急,病程较短,精神症状的程度轻重不一。情感障碍较突出,可出现失眠、欣快、易怒、言语动作增多,少数可表现为焦虑、抑郁,也可出现行为紊乱、幻觉、妄想等分裂样症状和意识障碍。

肾上腺皮质激素所致精神障碍的诊断有时会有困难。例如,临床上并不少见的系统性红斑狼疮患者在使用激素治疗的过程中出现精神症状,而此时系统性红斑狼疮又处于活动期,那么要判定精神症状是由系统性红斑狼疮导致(例如狼疮性脑病),还是激素所致则有一定困难,此时需要纵向观察、进行分析判断。

治疗:逐渐减药或停药,或改用其他激素,如因躯体疾病不能停用激素,可继续小剂量使用,并同时使用地西泮或小剂量抗精神病药控制精神症状。

\subsubsection{异烟肼所致精神障碍}

异烟肼是治疗结核病的首选药,较常见的不良反应有周围神经病、眩晕和失眠等,量大可引起视神经炎,诱发惊厥,甚至引起中毒性脑病和精神症状。发病机制可能是异烟肼的结构与维生素B\textsubscript{6}
相似,而与维生素B\textsubscript{6}
竞争同一酶系或两者结合成腙随尿排出体外,导致维生素B\textsubscript{6}
缺乏。

精神障碍表现为意识障碍、幻觉妄想状态、记忆障碍、躁狂或抑郁状态。

治疗:立即停药,改用其他抗结核药,大量补充B族维生素。控制精神症状,可使用地西泮或抗精神病药物。

\subsubsection{阿托品类生物碱所致精神障碍}

阿托品是M胆碱受体阻断药,可兴奋延脑和高位神经中枢。阿托品中毒时,除口干、面部潮红、皮肤干燥、瞳孔扩大、视物模糊、体温升高等抗胆碱能反应外,还可以出现精神症状,表现为兴奋、言语动作增多、躁动不安,较重者可出现谵妄,有恐怖性错觉、幻觉。严重中毒时则由兴奋转为抑郁,出现昏迷和呼吸麻痹。

治疗:毒扁豆碱是拟胆碱药,能对抗阿托品的中枢和周围抗胆碱能作用,宜及早使用。可给予0.5~2mg肌注或缓慢静脉注射,必要时重复,对躁动不安者可使用抗焦虑药。不宜使用有抗胆碱能反应的抗精神病药。

\subsubsection{抗感冒药所致精神障碍}

抗感冒药的种类繁多,除中药成分外,主要含有金刚烷胺、伪麻黄碱、马来酸氯苯那敏及对乙酰氨基酚等。由于此类药品属于非处方药,易于购买,常因超剂量服用或延长服用疗程而导致不良反应。急性精神障碍主要表现为睡眠障碍、烦躁焦虑、幻觉、妄想,严重者可有意识障碍。金刚烷胺导致精神障碍可能与该药能增加DA的释放,使中脑边缘系统多巴胺功能增强有关。伪麻黄碱能促进去甲肾上腺素的释放,产生拟交感效应。

治疗:立即停用抗感冒药,给予支持治疗及对症处理,精神症状严重者也可给予小剂量抗精神病药。

\subsubsection{减肥药所致精神障碍}

市场上减肥药物品种较多,有西药,也有中药,根据作用机理,这些药物大致可分为四类:摄食抑制剂、影响消化吸收的药物、加速代谢的激素类药物及其他药物。摄食抑制剂较常用的主要有拟儿茶酚胺类递质药物(如苯丙胺及其衍生物)和拟5-羟色胺能药物(如芬氟拉明、右旋芬氟拉明),加速代谢的激素类药物中含甲状腺激素、生长激素等,能引起精神症状。减肥药所致精神障碍轻度表现为失眠、焦虑,严重者可出现言语性幻听为主的精神病性症状、行为紊乱等。

治疗:立即停用减肥药,对精神症状可使用小剂量抗焦虑及抗精神病药物。

\subsection{有机化合物所致精神障碍}

\subsubsection{苯中毒所致精神障碍}

苯是一种化工原料和有机溶剂,是常见的工业毒物,属中等毒类,常温下容易挥发,主要以蒸气状态经呼吸道吸入,皮肤也可少量吸收。

急性中毒:主要影响中枢神经系统,由于苯的亲脂性,易附着于神经细胞表面,抑制生物氧化,影响神经递质,麻醉中枢神经系统。轻者表现醉酒状态、头晕、头痛、恶心、呕吐等;重者可出现意识障碍、肌肉痉挛、抽搐。

慢性中毒:主要影响造血系统和神经系统。早期神经系统最常见的表现为脑衰弱综合征和自主神经功能紊乱。造血系统最常见的表现为白细胞和血小板减少,继而出现各种出血症状,如鼻衄、皮下黏膜出血甚至内脏出血等而导致贫血。

治疗:对急性中毒患者应迅速将其移至空气新鲜处,注意呼吸情况。除给氧、给予呼吸兴奋药等一般措施外,还可以使用葡萄糖醛酸内酯,它与苯代谢产物酚结合物促进排毒。对精神障碍和慢性中毒可采取对症治疗。

\subsubsection{甲醇中毒所致精神障碍}

甲醇是一种工业原料,为无色易挥发的液体,气味与乙醇相似,可经呼吸道、消化道吸收,也可经皮肤部分吸收。在体内代谢和排泄均缓慢,有明显蓄积作用。甲醇主要作用于神经系统,有明显的麻醉作用。对视神经和视网膜有特殊的选择毒性作用,造成视神经萎缩。甲醇的代谢产物抑制细胞的有氧代谢,造成酸中毒。

临床表现:急性中毒早期呈醉酒状态,步态不稳,并有头昏、头痛、乏力、视物模糊、表情淡漠、失眠等,严重时出现谵妄、昏迷、呼吸衰竭,甚至死亡。可有双眼疼痛、视力下降、复视,甚至永久失明。少数患者出现精神症状,表现为恐惧、多疑、兴奋或抑郁、幻觉等。

治疗:对误服甲醇者应立即以5%NaHCO\textsubscript{3}
250mg静滴纠正酸中毒。积极防治水肿,可用20%甘露醇加地塞米松静滴。对于神经损害和慢性中毒者应给予B族维生素、神经营养药、血管扩张药、糖皮质激素等。

\subsubsection{有机磷农药中毒所致精神障碍}

有机农药可经消化、呼吸道和皮肤粘膜吸收,进入体内后迅速分布到全身各器官,肝脏中含量最高,脑内含量取决于其通过血脑屏障的能力。

中毒机制:有机磷的毒性作用,主要由于其与胆碱酯酶迅速结合,形成不易解离的磷酰化胆碱酯酶,后者不能催化水解乙酰胆碱导致乙酰胆碱蓄积,从而抑制神经传导,产生中毒症状。有些有机磷农药具有迟发性神经毒性作用,其机制可能是有机磷抑制神经鞘酯酶,并使其老化。

1.急性中毒 急性中毒的临床表现与体内胆碱酯酶活性受抑制的程度相平行,潜伏期的长短与药物的种类、摄入的剂量和途径有关。临床症状分为三类:

(1)毒覃碱样症状:过量的乙酰胆碱作用于胆碱能节后纤维,导致平滑肌和腺体高度兴奋引起大汗淋漓、支气管分泌物增多、肺水肿、呼吸困难等症状。

(2)烟碱样症状:过量的乙酰胆碱作用神经肌肉接头,引起肌束震颤、肌肉痉挛、抽搐,严重者呼吸肌麻痹。

(3)神经精神症状:中毒轻者主要表现脑衰弱综合征,可出现头痛、头晕、乏力、失眠、注意力不集中、焦虑、抑郁或欣快等;中毒重者则出现言语不清、定向障碍、意识障碍,严重者出现脑水肿、中枢性呼吸衰竭。精神症状有抑郁或欣快、焦虑、躁动不安、幻觉、妄想等。

2.慢性中毒 多见于职业性接触者,主要表现为脑衰弱综合征,如头痛、头晕、食欲不振、乏力,也可出现情绪低落、焦虑、易激怒等。

治疗:中毒者应立即脱离现场,清除毒物,口服者彻底洗胃,及早使用足量的特效解毒剂。阿托品可减轻或消除毒覃碱样症状(对烟碱样症状无效)和中枢神经症状。使用原则是早期、足量、反复给药,直到阿托品化,再减量维持。胆碱酯酶复能剂可使被抑制的胆碱酯酶恢复活性,可与阿托品合用。控制精神症状采取对症治疗,可选用地西泮、小剂量抗精神病药。

\subsection{一氧化碳中毒所致精神障碍}

一氧化碳(CO)中毒多由于工业生产时防护不当,家庭使用煤炉、燃气热水器时通风不良或煤气泄漏等造成。CO是一种无色无味的气体,含碳物质不完全燃烧时均可产生CO。

中毒机制:CO经呼吸道吸入后立即进入血液,与血红蛋白(Hb)结合成碳氧血红蛋白(HbCO)。CO与Hb的亲和力比氧与Hb的亲和力大200倍,而解离又比HbO\textsubscript{2}
慢3600倍。HbCO不仅不能携氧,还可以抑制HbO\textsubscript{2}
的解离。另外,高浓度的CO还可与细胞色素氧化酶中的铁结合,抑制组织的呼吸过程,使组织缺氧。中枢神经系统对缺氧最敏感,CO中毒后神经元内的ATP迅速耗尽,钠泵转运丧失能源,钠聚集在细胞内,而引起细胞内水肿。同时,缺氧引起血脑屏障通透性增加,引起细胞间水肿,从而导致脑血肿和脑血循环障碍。缺氧和脑血循环障碍可促使血栓形成、缺血性脑软化或广泛的脱髓鞘改变。脑组织的这些病理变化的形成需要一定的时间。因此,一部分急性CO中毒者在抢救苏醒后一段时间表现正常,即“假性痊愈”,在病理改变形成后又出现多种神经精神症状的迟发性脑病。

临床表现:急性中毒轻症者有头痛、头昏、乏力、恶心、呕吐、视物模糊等。急性期患者颜面充血,呈樱桃红色,四肢皮肤潮红,初期血压上升,后下降,可有心律失常。严重者可出现意识障碍、去大脑皮质状态,甚至呼吸衰竭死亡。

少数患者在抢救苏醒后经2~60天的假愈期,以后可出现迟发性脑病,表现为:①痴呆状态、谵妄状态、去大脑皮质状态;②锥体外系症状,出现帕金森综合征;③锥体系神经损害,如偏瘫,小便失禁、病理反射阳性等;④大脑皮质局灶性功能障碍,如失语、失明或继发性癫痫
;⑤头颅CT可发现病理性低密度区,EEG高度异常。

长期在CO浓度高的环境中工作可引起慢性中毒,主要表现为脑衰弱综合征,如多痛、头晕、记忆力下降、乏力、个性改变等。

治疗:急性中毒者应立即移至空气新鲜处,要积极纠正缺氧和防治脑水肿。应尽早行高压氧治疗。急性中毒后2~4小时即可出现脑水肿,24~48小时达到高峰并持续多天,应使用高渗脱水、利尿药、糖皮质激素等,并给予能促进脑血液循环和细胞代谢的药物,维持呼吸循环功能。经抢救苏醒后应卧床休息密切观察2周,预防并发症的发生。对精神症状可给予适量地西泮、小剂量抗精神病药。

\subsection{重金属所致精神障碍}

\subsubsection{铅中毒所致精神障碍}

铅是一种嗜神经性及溶血性毒物,以无机铅中毒多见。铅可经呼吸道、皮肤、消化道吸收。急性铅中毒多系口服可溶性铅无机化合物和含铅药物引起。慢性铅中毒为常见职业病,多见于在工作中长期吸入铅烟、铅尘的工人。四乙铅是有机铅化合物,为无色油状液体,主要用于汽油抗爆剂。

中毒机制:铅吸收后分布于全身各组织,最后约95%的铅稳定地沉淀于骨骼系统,器官中的铅以肝、肾中为高,主要经肾排出。铅在体内的代谢与钙相似,能促进钙排出的因素同样也促进铅的排出。铅损害中枢神经系统是由于其阻碍GABA的功能,降低细胞色素C的浓度,加速多巴胺释放,减少细胞外Ca\textsuperscript{2+}
的浓度,影响Ach的释放,最终引起各种行为和神经效应的改变。严重中毒引起神经细胞退行性改变,导致脑病。

四乙铅进入体内后迅速转变为毒性更大的三乙铅,主要抑制脑内的葡萄糖氧化和单胺氧化酶。四乙铅还抑制胆碱酯酶活力,轻者使大脑皮质功能失调和自主神经功能紊乱,严重时损害神经细胞,出现脑水肿和弥漫性脑损伤。

临床表现:急性中毒后出现恶心、呕吐、腹痛、血压升高。四乙铅易引起精神失常,表现为兴奋、感觉异常、幻觉、行为异常等。严重者发生中毒性脑病,出现痉挛、抽搐、昏迷、脑水肿。慢性中毒一般表现为脑衰弱综合征。

诊断:必须了解铅接触史,结合临床表现和尿铅、血铅等实验室检查结果,排除其他疾病后才能作出诊断。

治疗:误服铅化物者应立即采取洗胃、导泻等措施。驱铅治疗可使用依地酸钙钠或二巯基丁二酸钠。对中毒性脑病应积极改善脑缺氧和脑水肿。控制精神症状可使用地西泮或小剂量抗精神病药。

\subsubsection{汞中毒所致精神障碍}

汞俗称水银,为银白色的液态金属,常温下即蒸发。汞中毒以慢性多见,主要发生在生产活动中吸入汞蒸汽或汞化合物粉尘所致。大剂量吸入或摄入汞化合物即可发生急性中毒。金属汞和汞化合物蒸汽经呼吸道进入体内,金属汞在胃肠道几乎不吸收,但可由皮肤吸收,汞化合物也可经消化道吸收。

中毒机制:汞蒸汽脂溶性高,易透过肺泡壁含脂质的细胞膜。进入体内后被氧化成Hg\textsuperscript{2+}
。Hg\textsuperscript{2+}
与巯基亲和力高,它们结合后使与巯基有关的酶失去活性,阻碍了细胞代谢,使细胞发生损害。Hg\textsuperscript{2+}
可通过血脑屏障,中枢神经系统最易受到损害,也可以损害肾脏。

临床表现:精神障碍多见于慢性汞中毒,主要表现为头昏、头痛、记忆下降、失眠等脑衰弱综合征,以后可出现性格改变和情感障碍,表现为急躁、自制力差、易冲动、孤僻、胆怯、无故哭笑、焦虑不安、抑郁等。严重时出现多疑、幻觉、智能减退、震颤、共济失调等中毒性脑病表现。

诊断:根据接触史和脑衰弱综合征等为主的神经精神症状可以诊断。

治疗:脱离中毒环境,改善劳动条件,加强防护。驱汞治疗可肌内注射二巯基丙醇(BAL)或二巯基丙磺酸钠,这类解毒药分子中有活性的巯基,能与血液和组织中的巯基毒物起反应,形成无毒化合物由尿排出。对精神症状一般可给予抗焦虑药物。

\subsection{食物所致精神障碍}

此类精神障碍多见于食用有毒蕈类后引起。蕈类俗称蘑菇,有些蕈类含有毒素,且用一般方法难以破坏其毒性,误食后可引起中毒。不同蕈类所含毒素种类不同,可引起相应的毒性反应,可分为四型:①胃肠炎型;②溶血型;③中毒性肝炎型;④神经精神型。不同毒蕈所致中毒在开始时均有胃肠道症状。

神经精神型多由误食毒蝇伞、豹斑毒伞和牛肝蕈等引起。毒蝇伞、豹斑毒伞的毒素为毒蕈碱。中毒后,副交感神经兴奋症状突出,如多汗、流涎、流泪、心率减慢、瞳孔缩小等。精神症状可为欣快、话多,少数严重者出现幻觉、谵妄、呼吸抑制等。误食牛肝蕈后,有人可出现小人国幻觉、谵妄、幻听、被害妄想等精神症状。

治疗:除催吐、洗胃、导泻等措施外,对毒蝇伞、豹斑毒伞中毒者应用阿托品,不仅能改善胃肠道症状,还可以消除蕈碱样症状,改善心脏传导。轻者给予阿托品0.5~1mg皮下注射,每隔30分钟至6小时1次,严重者1~2mg,每隔15~30分钟1次,病情好转后应调整剂量或停药。精神症状明显者可用小剂量的抗精神病药。